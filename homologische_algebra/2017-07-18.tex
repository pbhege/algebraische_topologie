Ist
\begin{equation*}
  \begin{tikzcd}
    0 \arrow{r}{}
      & A \arrow{r}{f}
      & B \arrow{r}{g}
      & C \arrow{r}{}
      & 0
  \end{tikzcd}
\end{equation*}
exakt, so ist
\begin{equation*}
  \begin{tikzcd}[arrows=leftarrow]
    ? \arrow{r}{}
      & \Hom(A,G) \arrow{r}{f^*}
      & \Hom(B,G) \arrow{r}{g^*}
      & \Hom(C,G) \arrow{r}{}
      & 0
  \end{tikzcd}
  \tag{$*$}\label{eqn:kes_unvollstaendig}
\end{equation*}
ist exakt, aber $f^*$ i.A.\ nicht surjektiv.
Naheliegend könnte man~\eqref{eqn:kes_unvollstaendig} mit
\begin{equation*}
  \coker f^* := \Hom(A,G) / {\im f^*}
\end{equation*}
und
\begin{equation*}
  \nu\colon \Hom/A,G) \to \coker f^*
\end{equation*}
fortsetzen, was aber so aussieht, dass es von zu vielen Wahlen abhängt.
\begin{defn}
  Seien $A$ und $G$ abelsche Gruppen und $S(A)$ die Standardauflösung von $A$.
  Dann nennt man
  \begin{align*}
    \Ext(A,G) &:= \coker i^* = \Hom(R,g) / {\im i^*}, \\
    i^* & \colon \Hom(F,G) \to \Hom/R,G)
  \end{align*}
  das \emph{Extensionsprodukt} (kurz: Ext-Produkt) \emph{von $A$ und $G$}.
\end{defn}
\begin{kommentar}
  Für die Standardauflösung $S(A)\colon \begin{tikzcd} 0 \arrow{r}{} &R\arrow{r}{i} &F\arrow{r}{\pi} &A\arrow{r}{}&0\end{tikzcd}$ von $A$ wird dann also mit der kanonischen Projektion $\nu\colon \Hom(R,G) \to \Ext(A,G)$ die folgende Sequenz exakt:
  \begin{equation*}
    \begin{tikzcd}
      0 \arrow{r}{}
        & \Hom(C,G) \arrow{r}{g^*}
        & \Hom(B,G) \arrow{r}{f^*}
        & \Hom(A,G) \arrow{r}{\nu}
        & \Ext(A,G) \arrow{r}{}
        & 0
    \end{tikzcd}
  \end{equation*}
\end{kommentar}
\emph{Frage}: Wie ist das mit anderen freien Auflösungen?
\begin{lemma}
  Seien $A$ und $A'$ abelsche Gruppen und
  \begin{equation*}
    \begin{tikzcd}[row sep=tiny]
      S\colon 0  \arrow{r}{}  & R \arrow{r}{f}  & F \arrow{r}{g}  & A \arrow{r}{} & 0 \\
      S'\colon 0 \arrow{r}{}  & R \arrow{r}{f'} & F \arrow{r}{g'} & A \arrow{r}{} & 0
    \end{tikzcd}
  \end{equation*}
  freie Auflösungen sowie $h\colon A \to A'$ ein Homomorphismus.
  \begin{equation*}
    \begin{tikzcd}
      0 \arrow{r}{} & R   \arrow{d}{\alpha}
                          \arrow{r}{f}  & F   \arrow{d}{\beta}
                                              \arrow{r}{g}  & A   \arrow{d}{h}
                                                                  \arrow{r}{} & 0 \\
      0 \arrow{r}{} & R'  \arrow{r}{f'} & F'  \arrow{r}{g'} & A'  \arrow{r}{} & 0 
    \end{tikzcd}
    \tag{$*$}\label{eqn:frei_aufl_hom}
  \end{equation*}
  \begin{enumerate}[(a)]
    \item 
      Dann existieren Homomorphismen $\alpha \colon R \to R'$ und $\beta \colon F \to F'$, die das Diagramm~\eqref{eqn:frei_aufl_hom} kommutieren lassen.
      \begin{equation*}
        \begin{tikzcd}
      0 \arrow{r}{} & R   \arrow[swap]{d}{\tilde\alpha - \alpha}
                          \arrow{r}{f}  & F   \arrow{d}{\tilde\beta - \beta}
                                              \arrow{dl}{\varphi}
                                              \arrow{r}{g}  & A   \arrow{d}{h-h = 0}
                                                                  \arrow{r}{} & 0 \\
      0 \arrow{r}{} & R'  \arrow[swap]{r}{f'} & F'  \arrow{r}{g'} & A'  \arrow{r}{} & 0 
        \end{tikzcd}
        \tag{$**$}\label{eqn:frei_aufl_homotopie}
      \end{equation*}
    \item
      Sind $\tilde\beta \colon F \to F'$ und $\tilde\alpha \colon R \to R'$ weitere Homomorphismen, die~\eqref{eqn:frei_aufl_hom} kommutieren lassen, so existiert ein $\varphi \colon F \to R'$ mit
      \begin{equation*}
        f' \circ \varphi = \tilde\beta - \beta, \qquad \varphi \circ f = \tilde\alpha - \alpha.
      \end{equation*}
  \end{enumerate}
\end{lemma}
\begin{proof}
  \begin{enumerate}[(a)]
    \item 
      Sei ${(e_i)}_{i \in I}$ eine Basis von $F$.
      Betrachte dann $h \circ g (e_i) \in A'$.
      Da $f'$ surjektiv ist, existiert $x_i \in F'$ mit $f'(x_i) = h \circ g(e_i)$.
      Definiere dann $\beta \colon F \to F'$ durch $\beta(e_i) := x_i$.
      Dann gilt für alle $i \in I$:
      \begin{equation*}
        g' \circ \beta(e_i) = g'(x_i) = h \circ g(e_i)
      \end{equation*}
      und somit $g' \circ \beta = h \circ g$.

      Für alle $x \in R$ ist
      \begin{align*}
        g' \circ (\beta \circ f) (x)
          & = (g' \circ \beta) \circ f (x) \\
          & = h \circ \underset{=0}{\underbrace{g \circ f(x)}}
      \end{align*}
      und somit $\im (\beta \circ f) \subseteq  \ker g' = \im f'$.
      Bezeichne mit $f'\colon R' \to \im f'$ auch den Homomorphismus $f'$, wenn man der Wertebereich auf $\im f'$ einschränkt.
      Dann ist (dieses) $f'$ ein Isomorphismus, da $f'$ injektiv und (nun auch) surjektiv ist.
      Setze dann
      \[
        \alpha \colon R \to R',\quad \alpha := {(f')}^{-1} \circ \beta \circ f.
      \]
      Dann ist offenbar
      \begin{equation*}
        f' \circ \alpha = \beta \circ f.
      \end{equation*}
    \item
      Für  alle $x \in F$ ist
      \begin{equation*}
        g' \circ (\tilde\beta - \beta) (x) = (g' \circ \tilde \beta - g' \circ \beta) (x) = h\circ g(x) - h \circ g(x) = 0
      \end{equation*}
      und somit $\im (\tilde\beta - \beta) \subseteq \ker g' = \im f'$.
      Setze daher $\varphi \colon F \to R'$ mit $\varphi := {(f')}^{-1} \circ (\tilde \beta - \beta)$, dann gilt $f' \circ \varphi = \tilde \beta - \beta$.
      Es ist aber auch:
      \begin{align*}
        f' \circ (\tilde\alpha - \alpha)
          & = f' \circ \tilde \alpha - f' \circ \alpha \\
          & = \tilde \beta \circ f - \beta \circ f \\
          & = (\tilde \beta - \beta) \circ f \\
          & = f' \circ \varphi \circ f.
      \end{align*}
      Da $f'$ injektiv ist, folgt $\tilde \alpha - \alpha = \varphi \circ f$.
  \end{enumerate}
\end{proof}
\begin{prop}
  \label{thm:hom_freier_aufloesungen}
  Seien $A$, $A'$ abelsche Gruppen, $S$ und $S'$ freie Auflösungen von $A$ und $A'$,
  \begin{equation*}
    \begin{tikzcd}
      S\colon\,0 \arrow{r}{} & R   \arrow{d}{f}
                              \arrow{r}{j}  & F   \arrow{d}{}
                                                  \arrow{r}{}   & A   \arrow{d}{h}
                                                                      \arrow{r}{} & 0, \\
      S'\colon 0 \arrow{r}{} & R'  \arrow{r}{j'} & F'  \arrow{r}{}   & A'  \arrow{r}{} & 0,
    \end{tikzcd}
  \end{equation*}
  und sei $h\colon A \to A'$ Homomorphismus.
  Dann gibt es genau einen Homomorphismus
  \begin{equation*}
    \Phi(h;S,S') \colon \coker {(j')}^* \to \coker j^*,
  \end{equation*}
  sodass gilt: Sind $g \colon F \to F'$ und $f \colon R \to R'$ Homomorphismen derart, dass $(f,g,h) \colon S \to SÄ$ Homomorphismus zwischen freien Auflösungen ist, d.h.~\eqref{eqn:frei_aufl_hom} kommutiert, so kommutiert auch (für jede abelsche Gruppe $G$):
  \begin{equation*}
    \begin{tikzcd}
      0 \arrow{r}{}   & \Hom(A,G)   \arrow{r}{}   & \Hom(F,G) \arrow{r}{j^*}        & \Hom(R,G) \arrow{r}{\nu}  & \coker j^* \arrow{r}{}      & 0 \\
      0 \arrow{r}{}   & \Hom(A',G)  \arrow{u}{h^*}
                                    \arrow{r}{}   & \Hom(F',G)\arrow{u}{g^*}
                                                              \arrow{r}{{(j')}^*}  & \Hom(R',G) \arrow{u}{f^*}
                                                                                                \arrow[swap]{ul}{\alpha^*}
                                                                                                \arrow{r}{\nu} & \coker {(j')}^* \arrow{u}{\Phi}
                                                                                                                              \arrow{r}{} & 0
    \end{tikzcd}
  \end{equation*}
\end{prop}
\begin{kommentar}
  Das Besondere ist nicht so sehr die Existenz und Eindeutigkeit von $\Phi$, denn es gibt nur einen Kandidaten
  \begin{equation*}
    \Phi([\varphi]) = [f^*(\varphi)]  \qquad \text{für } \varphi \in \Hom(R',G), 
  \end{equation*}
  sondern mehr die Unabhängigkeit von $f$ (und $g$).
\end{kommentar}
\begin{proof}[Beweis von Proposition~\ref{thm:hom_freier_aufloesungen}]
  Wegen $f^* \circ {(j')}^* = j^* \circ g^*$ ist
  \begin{equation*}
    f^* (\im {(j')} ^*) \subseteq \im j^*
  \end{equation*}
  und somit $\nu \circ f^* (\im {(j')}^* = 0$, also existiert eindeutiges $\Phi \colon \coker {(j')}^* \to \coker j^*$ mit $\Phi \circ \nu' = \nu \circ f^*$, also
  \begin{equation*}
    \Phi([\varphi]) = [f^*(\varphi)] \quad \text{für alle $\varphi \in \Hom(R',G)$}.
  \end{equation*}
  Ist $(\tilde f, \tilde g)$ eine weitere Wahl, die~\eqref{eqn:frei_aufl_hom} kommutieren lässt, und $\alpha \colon F \to R'$ mit $\alpha \circ j = \tilde f - f$ (und $j' \circ \alpha = \tilde g - g$), so ist $j^* \circ \alpha^* = {\tilde f}^* - f^*$ und somit ${\tilde f}^* (\varphi) - f^*(\varphi) \in \im j^*$.
  Macht daher $\tilde\Phi$~\eqref{eqn:frei_aufl_hom} kommutativ (mit $\tilde f$ statt $f$ und $\tilde g$ statt $g$), so ist
  \begin{equation*}
    \tilde\Phi ([\varphi]) = [f^*(\varphi)] = [{\tilde f}^*(\varphi)] = \Phi ([\varphi]) \quad \text{für alle $\varphi \in \Hom(R',G)$}.
  \end{equation*}
\end{proof}
\begin{korollar}
  Definiert man eine Kategorie \C\ dadurch, dass man als Objekte freie Auflösungen $S$ von abelschen Gruppen $A$ betrachtet und als Morphismen solche von $A$ nach $A'$,
  \begin{equation*}
    \begin{tikzcd}
      S\colon 0 \arrow{r}{}  & R \arrow{r}{j}  & F \arrow{r}{} & A \arrow{d}{h}
                                                              \arrow{r}{}   & 0 \\
      \phantom{S\colon}
      0 \arrow{r}{}  & R'\arrow{r}{j}  & F'\arrow{r}{} & A'\arrow{r}{}   & 0,
    \end{tikzcd}
  \end{equation*}
  so ist die Zuordnung (für gegebene $G$) $\Phi \colon \C \to \Ab$ auf Objekten $\Phi(S) := \coker j^*$ und auf Morphismen $h \mapsto \Phi(h;S,S')$, wie in der Proposition, funktoriell:
  \begin{enumerate}[(a)]
    \item 
      $\Phi(\id;S,S) = \id$
    \item
      $\Phi(h' \circ h;S,S'') = \Phi(h;S,S') \circ \Phi(h';S',S'')$ für alle $h,h',S,S',S''$ etc.
    \end{enumerate}
\end{korollar}
\begin{proof}
  \begin{enumerate}[(a)]
    \item 
      Ist $h = \id\colon S \to S$, so kann $f = \id$ und $g = \id$ gewählt werden und man erhält: $\Phi = \id$.
    \item
      Sind $(f,g)$ bzw.\ $(f',g')$ für $h \colon S \to S'$ bzw.\ $h'\colon S' \to S''$ gewählt, so kann man für $(f'',g'')$ die Wahlen $f'' = f' \circ f$ und $g'' = g' \circ g$ treffen, weil das Diagramm~\eqref{eqn:neu} dann als Zusammensetzung von~\eqref{eqn:} und~\eqref{eqqn:} kommutiert.
      \todo{Fill the gaps}
      Somit gilt
      \begin{align*}
        \Phi(h' \circ h; S,S'')
          & = [{(f' \circ f)}^* (\varphi)]\\
          & = [f^* \circ {(f')}^* (\varphi)]\\
          & = \Phi(h;S,S') \circ \Phi(h';S',S'').
      \end{align*}
  \end{enumerate}
\end{proof}

