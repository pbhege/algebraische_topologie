Ist
\begin{equation*}
  \begin{tikzcd}
    0 \arrow{r}{}
      & A \arrow{r}{f}
      & B \arrow{r}{g}
      & C \arrow{r}{}
      & 0
  \end{tikzcd}
\end{equation*}
exakt, so ist
\begin{equation*}
  \begin{tikzcd}[arrows=leftarrow]
    ? \arrow{r}{}
      & \Hom(A,G) \arrow{r}{f^*}
      & \Hom(B,G) \arrow{r}{g^*}
      & \Hom(C,G) \arrow{r}{}
      & 0
  \end{tikzcd}
  \tag{$*$}\label{eqn:kes_unvollstaendig}
\end{equation*}
ist exakt, aber $f^*$ i.A.\ nicht surjektiv.
Naheliegend könnte man~\eqref{eqn:kes_unvollstaendig} mit
\begin{equation*}
  \coker f^* := \Hom(A,G) / {\im f^*}
\end{equation*}
und
\begin{equation*}
  \nu\colon \Hom/A,G) \to \coker f^*
\end{equation*}
fortsetzen, was aber so aussieht, dass es von zu vielen Wahlen abhängt.
\begin{defn}
  Seien $A$ und $G$ abelsche Gruppen und $S(A)$ die Standardauflösung von $A$.
  Dann nennt man
  \begin{align*}
    \Ext(A,G) &:= \coker i^* = \Hom(R,g) / {\im i^*}, \\
    i^* & \colon \Hom(F,G) \to \Hom/R,G)
  \end{align*}
  das \emph{Extensionsprodukt} (kurz: Ext-Produkt) \emph{von $A$ und $G$}.
\end{defn}
