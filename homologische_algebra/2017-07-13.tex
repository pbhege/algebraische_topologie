\setcounter{prop}{3}
\begin{prop}
  Sei $G$ eine abelsche Gruppe und
  \begin{equation*}
    \begin{tikzcd}
      A \arrow{r}{f}  & B
        \arrow{r}{g}  & C
        \arrow{r}{}   & 0
    \end{tikzcd}
  \end{equation*}
  eine exakte Sequenz abelscher Gruppen. Dann is auch die induzierte Sequenz
  \begin{equation*}
    \begin{tikzcd}
      \Hom(A,G) & \Hom(B,G) \arrow{l}{f^*}
                & \Hom(C,G) \arrow{l}{g^*}
                & 0         \arrow{l}{}
    \end{tikzcd}
  \end{equation*}
  exakt. $\Hom(-,G)$ ist \emph{links-exakt}.
\end{prop}
\begin{proof}
  \begin{enumerate}[(i)]
    \item 
      Exaktheit bei $\Hom(C,G)$.
      Zeige $g^*$ ist injektiv.
      Sei $\varphi \in \Hom(C,G)$ mit $g^*(\varphi) = \varphi \circ g  = 0$.
      \begin{align*}
        \begin{tikzcd}[ampersand replacement=\&]
          B \arrow{rd}{0} \arrow[twoheadrightarrow]{r}{g}  \& C \arrow{d}{\varphi} \\
                                        \& G
        \end{tikzcd}
        & \qquad \implies \varphi = 0
      \end{align*}
      %\begin{equation*}
        %\begin{tikzcd}
          %B \arrow{rd}{0} \arrow[twoheadrightarrow]{r}{g}  & C \arrow{d}{\varphi} \\
                                        %& G
        %\end{tikzcd}
      %\end{equation*}
      %Dann folgt $\varphi = 0$.
    \item
      Exaktheit bei $\Hom(B,G)$:
      \begin{enumerate}[(a)]
        \item 
          $\im g^* \subseteq \ker f^*$, also $f^* \circ g^* = 0$.
          Aber $f^* \circ g^* = {(g \circ f)}^* = 0^* = 0$.
        \item
          $\ker f^* \subseteq \im g^*$:
          Sei $\varphi \colon B \to G \in \ker f^*$, $0 = f^*(\varphi) = \varphi \circ f$.
          \begin{equation*}
            \begin{tikzcd}
              B \arrow[twoheadrightarrow]{r}{g}
                \arrow[twoheadrightarrow]{rd}{\pi}
                \arrow{d}{\phi}
                & C \\
              G & B/{\ker g}  \arrow{u}{\overline g}
                              \arrow{l}{\overline \varphi}
            \end{tikzcd}
          \end{equation*}
          Dann ist $\ker g = \im f \subseteq \ker \varphi$ und daraus folgt die eindeutige Existenz eines $\overline \varphi \colon B/{\ker g} \to G$ mit $\overline \varphi \circ \pi = \varphi$.

          Ebenso induziert $g$ einen Morphismus $\overline g\colon B/{\ker g} \to C$ mit $\overline g \circ \pi = g$.
          Außerdem ist $\overline g$ injektiv und surjektiv, also ein Isomorphismus und somit
          \begin{equation*}
            \varphi = \overline \varphi \circ \pi = \overline \varphi \circ {\overline g}^{-1} \circ g = g^* (\overline \varphi \circ {\overline g}^{-1}).
            \end{equation*}
      \end{enumerate}
  \end{enumerate}
\end{proof}
\begin{kommentar}
  Man sagt, dass der kontravariante Funktor $\Hom(-,G) =: F$ links-exakt ist.
  Beachte, dass $F$ allerdings i.A.\ nicht exakte Sequenzen
  \begin{equation*}
    \begin{tikzcd}
      0 \arrow{r}{}   & A
        \arrow{r}{f}  & B
        \arrow{r}{g}  & C
        \arrow{r}{}   & 0
    \end{tikzcd}
  \end{equation*}
  in exakte Sequenzen überführt.
  \begin{equation*}
    \begin{tikzcd}
      0         & \Hom(A,G) \arrow{l}{}
                & \Hom(B,G) \arrow{l}{f^*}
                & \Hom(C,G) \arrow{l}{g^*}
                & 0         \arrow{l}{}
    \end{tikzcd}
  \end{equation*}
  \begin{tikz}[overlay]
    \draw[dashed] (3.1,.65) rectangle +(2.2,1);
  \end{tikz}
  ist also i.A.\ nicht exakt.
\end{kommentar}
\begin{erinnerung}
  eine exakte Sequenz abelscher Gruppen
  \begin{equation*}
    \begin{tikzcd}
      0 \arrow{r}{}
        & A \arrow{r}{f}
        & B \arrow[bend left]{r}{\theta}
        & C \arrow[bend left]{l}{r}
            \arrow[loop above]{r}{id_C}
            \arrow{r}{}
        & 0
    \end{tikzcd}
  \end{equation*}
  mit $g \circ r = id_C$ \emph{spaltet}.
  Äquivalent:
  \begin{equation*}
    \begin{tikzcd}
      0 \arrow{r}{}
        & A \arrow[bend left]{r}{f}
            \arrow[loop above]{r}{id_A}
        & B \arrow[bend left]{l}{l}
            \arrow{r}{g}
        & C \arrow{r}{}
        & 0
    \end{tikzcd}
  \end{equation*}
  mit $l \circ f = id_A$.

  In diesem Fall gilt $B \cong A \oplus C$.
\end{erinnerung}
\begin{zusatz}
  Ist
  \begin{equation*}
    \begin{tikzcd}
      0 \arrow{r}{}   & A
        \arrow{r}{f}  & B
        \arrow{r}{g}  & C
        \arrow{r}{}   & 0
    \end{tikzcd}
  \end{equation*}
  exakt und spaltet, so ist auch
  \begin{equation*}
    \begin{tikzcd}
      0         & \Hom(A,G) \arrow{l}{}
                & \Hom(B,G) \arrow{l}{f^*}
                & \Hom(C,G) \arrow{l}{g^*}
                & 0         \arrow{l}{}
    \end{tikzcd}
    \tag{$*$}\label{eqn:hom_spaltung}
  \end{equation*}
  exakt und spaltet.
\end{zusatz}
\begin{proof}
  ist $l \colon B \to A$ linksinvers zu $f$, $l \circ f = id_A$, so ist $id_{\Hom(A,G)} = id_{A}^* = {(l \circ f)}^* = f^* \circ l^*$, also ist $f^*$ surjektiv.
  Außerdem ist nun $l^*$ offenbar rechtsinvers zu $f^*$, also eine Spaltung von~\eqref{eqn:hom_spaltung}.
\end{proof}
\begin{defn}
  Sei $A$ eine abelsche Gruppe.
  Dann heißt eine kure exakte Sequenz
  \begin{equation*}
    \begin{tikzcd}
      0 \arrow{r}{}
        & R \arrow{r}{\alpha}
        & F \arrow{r}{\beta}
        & F \arrow{r}{}
        & 0
    \end{tikzcd}
  \end{equation*}
  eine \emph{freie Auflösung}, wenn $F$ eine frei abelsche Gruppe ist.
\end{defn}
\begin{kommentar}
  Als Untergruppe (vie $\alpha$) von $F$ ist $R$ selbst eine frei abelsche Gruppe.
  Ist ${(e_i)}_{i \in I}$ eine Basis von $F$, so ist $\varepsilon = {(\beta(e_i))}_{i \in I}$ ein Erzeugendensystem von $A$.
  Und ist ${(r_j)}_{j \in J}$ eine Basis von $R$, so erzeugt ${(\alpha(r_j))}_{j \in J}$ die Relationen von $\varepsilon$ (\emph{Relationen auf $\varepsilon$}: $f \in F$ mit $\beta(f) = 0$).
\end{kommentar}
\begin{beispiel}
  \begin{enumerate}
    \item 
      ist $A$ selbst frei, s okann man $F = A$ und $\beta = id_A$ wählen (dann $R = \triv$).
    \item
      Ist $A = \Z_4$, so ist
      \begin{equation*}
        \begin{tikzcd}
          0 \arrow{r}{}
            & \Z \arrow{r}{\cdot 2}
            & \Z \arrow{r}{\pi}
            & \Z_2 \arrow {r}{}
            & 0
        \end{tikzcd}
      \end{equation*}
      eine freie Auflösung.
    \item
      Ist $A$ beliebig, so betrachte $A$ als menge und setze $F = \F(A)$ und $\pi\colon F \to A$ der Homorphismus, der auf der Basis ${(i(a))}_{a \in A}$ durch $\pi(i(a)) = a$ gegeben ist.
      Natürlich ist dann $\pi(2 \cdot a) = \pi(1\cdot (2a)) = 2a$ und $\pi(0_A) = \pi(0_{\F(A)}) = 0_A$.
      Ist $R = \ker \pi$ und $j\colon R \hookrightarrow F$ die Inklusion, so ist
      \begin{equation*}
        \begin{tikzcd}
          0 \arrow{r}{}
            & R \arrow{r}{j} 
            & F \arrow{r}{\pi}
            & A \arrow{r}{}
            & 0
        \end{tikzcd}
      \end{equation*}
      offenbar exakt (weil $\pi$ surjektiv ist).
      Das ist die \emph{Standardauflösung} $S(A)$ von $A$:
      \begin{equation*}
        \begin{tikzcd}
          S(A):
          0 \arrow{r}{}
            & R \arrow{r}{j} 
            & F \arrow{r}{\pi}
            & A \arrow{r}{}
            & 0
        \end{tikzcd}
      \end{equation*}
  \end{enumerate}
\end{beispiel}
