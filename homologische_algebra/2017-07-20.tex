\begin{kommentar}
  \begin{enumerate}[(a)]
    \item 
      Ist $h\colon A \to A'$ Isomorphismus, so ist
      \begin{equation*}
        \Phi(h;S,S')\colon \coker {(j')}^* \to \coker j^*
      \end{equation*}
      ebenfalls Isomorphismus, denn $\Phi(h^{-1};S',S) = {\Phi(h;S,S')}^{-1}$.
    \item
      Ist insbesondere $S(A)$ die Standardauflösung von $A$, $S$ eine beliebige freie Auflösung von $A$ und $h = \id_A$, so ist also das induzierte
      \begin{equation*}
        \Phi(\id;S,S(A))\colon \coker i^* = \Ext(A,G) \to \coker j^*
      \end{equation*}
      ein (kanonischer) Isomorphismus.
      Daher hängt der Isomorphietyp von $\Ext(A,G)$ nicht von der Wahl der freien Auflösung von $A$ ab.
  \end{enumerate}
\end{kommentar}
\begin{beispiel}
  \begin{enumerate}[(a)]
    \item 
      ist $A$ frei-abelsch und $G$ beliebig, so ist
      \begin{equation*}
        \Ext(A,G) = \triv,
      \end{equation*}
      denn:
      \begin{equation*}
        \begin{tikzcd}
          S\colon 0 \arrow{r}{}  & 0 \arrow{r}{0}  & A \arrow{r}{\id}  & A \arrow{r}{} & 0
        \end{tikzcd}
      \end{equation*}
      ist freie Auflösung und $\coker 0^* = \triv$.
    \item
      Für zwei abelsche Gruppen $A_1$ und $A_2$ (und $G$ beliebig) ist
      \begin{equation*}
        \Ext(A_1 \oplus A_2,G) \cong \Ext(A_1,G) \oplus \Ext(A_2,G),
      \end{equation*}
      denn: Sind
      \begin{equation*}
        \begin{tikzcd}[row sep=tiny]
          S_1\colon 0 \arrow{r}{}  & R_1 \arrow{r}{j_1}  & F_1 \arrow{r}{}  & A_1 \arrow{r}{} & 0 \\
          S_2\colon 0 \arrow{r}{}  & R_2 \arrow{r}{j_2}  & F_2 \arrow{r}{}  & A_2 \arrow{r}{} & 0
        \end{tikzcd}
      \end{equation*}
      freie Auflösungen, so ist
      \begin{equation*}
        \begin{tikzcd}[row sep=tiny]
          S_1 \oplus S_2\colon 0 \arrow{r}{}  & R_1 \oplus R_2 \arrow{r}{j_1 \oplus j_2}  & F_1 \oplus F_2 \arrow{r}{}  & A_1 \oplus A_2 \arrow{r}{} & 0
        \end{tikzcd}
      \end{equation*}
      eine freie Auflösung von $A_1 \oplus A_2$ und daher ist
      \begin{align*}
        \Ext(A_1 \oplus A_2,G) \cong \coker {(j_1 \oplus j_2)}^* 
          & \cong \coker (j_1^* \oplus j_2^*) \\
          & \cong \coker j_1^* \oplus \coker j_2^* \\
          & \cong \Ext(A_1,G) \oplus \Ext(A_2,G).
      \end{align*}
    \item
      Für jede abelsche Gruppe $G$ und $n \in \N_0$ sei
      \begin{equation*}
        n G := \{ \underset{\mathclap{g+\dotsb+g \ (\text{$n$-mal})}}{\underbrace{ng}} \in G : g \in G \} \leq G
      \end{equation*}
      Denn gilt für $\Z_n := \Z / {Z_n}$:
      \begin{equation*}
        \Ext(\Z_n, G) = G/{nG},
      \end{equation*}
      denn
      \begin{equation*}
        \begin{tikzcd}
          0 \arrow{r}{} & \Z \arrow{r}{\cdot n}[swap]{=:j}  & \Z \arrow{r}{\pi} & \Z_n \arrow{r}{}  & 0
        \end{tikzcd}
      \end{equation*}
      ist eine freie Auflösung von $\Z_n$.
      Identifiziert man nun noch $\Hom(\Z,G)$ mit $G$ (vermöge $\varphi \mapsto \varphi(1)$), so erhält man, dass $j^*\colon G \to G, j^*(g) = n \cdot g$ ist.
      Daher ist
      \begin{equation*}
        \Ext(\Z_n,G) \cong \coker j^* \cong G/{nG}.
      \end{equation*}
      \emph{Bemerkung}: Beachte, dass also
      \begin{equation*}
        \Ext(\Z_n,\Z) \cong \Z/{n\Z} = \Z_n \neq \triv = \Ext(\Z,\Z_n)
      \end{equation*}
      für $n \ge 2$.
      Also ist $\Ext$ (im Unterschied zu $\operatorname{Tor}(-,-)$, gebildet ganz ähnlich wie $\Ext$ aus $\Hom$, aus $-\oplus-$) im Allgemeinen nicht symmetrisch in seinen beiden Argumenten, genauso wie $\Hom(-,-)$ auch, denn z.B.\ ist
      \begin{equation*}
        \Hom(\Z,\Z_n) \cong \Z_n \neq \triv = \hom(\Z_n,\Z) \quad \text{für $n\geq 2$}.
      \end{equation*}
    \item
      Ist insbesondere $G$ ein Körper der Charakteristik 0 (z.B.\ $G = \Q,\R,\CC$), so ist also für alle $n \in \N$
      \begin{equation*}
        n = 1 + \dotsb + 1 \ \text{($n$-mal)}
      \end{equation*}
      verschieden von $0$, also existiert $n^{-1} \in G$ und damit ist $nG = G$, also ist $g/{nG} = \triv$.
      In diesem Fall ist also
      \begin{equation*}
        \Ext(\Z_n,G) = \triv.
      \end{equation*}
    \item
      Für endlich erzeugte abelsche Gruppe $A$, wo also
      \begin{equation*}
        A \cong \Z^b \oplus \Z_{n_1} \oplus \dotsb \oplus \Z_{n_r}
      \end{equation*}
    (mit $b \in \N_0$ und $n_i \in \N^{\ge 2}$) ist, erhält man also
    \begin{equation*}
      \Ext(A,G) = \triv
    \end{equation*}
    für jeden Körper $G$ der Charakteristik 0.
  \end{enumerate}
\end{beispiel}
\begin{defn}
  Sei $G$ fest und $f\colon A \to B$ Homomorphismus abelscher Gruppen.
  Wir nennen dann
  \begin{equation*}
    f^* = \Phi(f;S(A),S(B))\colon \Ext(B,G) \to \Ext(A,G)
  \end{equation*}
  den \emph{von $f$ induzierten Homomorphismus}.
\end{defn}
\begin{kommentar}
  Es wird dann
  \begin{equation*}
    T := \Ext(-,G)\colon \Ab \to \Ab
    \end{equation*}
    ein kontravarianter Funktor,
    \begin{align*}
      T(A) & = \Ext(A,G) \\
      T(f) & = \Ext(f,G) = f^*,
    \end{align*}
    denn
    \begin{align*}
      T(\id) & = \id^* = \id \\
      T(g \circ f) & = T(f) \circ T(g)
    \end{align*}
\end{kommentar}
