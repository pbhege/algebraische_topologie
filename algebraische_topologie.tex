
\documentclass[parskip=half,a4paper]{scrartcl}

\usepackage[utf8x]{inputenc}
\usepackage[T1]{fontenc}
\usepackage[ngerman]{babel}

\usepackage{amsmath,amssymb,amsthm}
\usepackage{mathtools}

\usepackage{microtype}

\usepackage{enumerate}

\renewcommand{\labelenumi}{(\alph{enumi})}
\renewcommand{\labelenumii}{(\roman{enumii})}

\usepackage{tikz}
%\usetikzlibrary{cd}
\usepackage{tikz-cd}

\tikzset{%
  >=stealth
}
%\tikzcdset{%
  %arrow style=tikz,
  %diagrams={>={Straight Barb}}
  %%diagrams={>=stealth}
%}

%\listfiles
\usepackage{todonotes}

\newtheoremstyle{default}
{3pt}% pace above
{3pt}% Space below
{}% Body font
{}% Indent amount
{\bfseries}% Theorem head font
{.}% Punctuation after theorem head
{.5em}% Space after theorem head
{\thmnumber{(#2)} \thmname{#1}\thmnote{ (#3)}}% Theorem head spec (can be left empty, meaning ‘normal’)
\theoremstyle{default}
\newtheorem{prop}{Proposition}[section]
\newtheorem{lemma}[prop]{Lemma}
\newtheorem{erinnerung}[prop]{Erinnerung}
\newtheorem{kommentar}[prop]{Kommentar}
\newtheorem{korollar}[prop]{Korollar}
\newtheorem{beispiel}[prop]{Beispiel}
\newtheorem{zusatz}[prop]{Zusatz}
\newtheorem{defn}[prop]{Definition}
\newtheorem{bemerkung}[prop]{Bemerkung}
\newtheorem{vorbereitung}[prop]{Vorbereitung}
\newtheorem{motivation}[prop]{Motivation}
\newtheorem{satz}[prop]{Satz}

\title{Algebraische Topologie II ff}
\author{Ingo Skupin}
%\renewcommand*{\sectionformat}{\textrm\S \thesection\enskip}


\newcommand{\Hom}{\mathrm{Hom}}
\newcommand{\Ext}{\mathrm{Ext}}
\newcommand{\im}{\operatorname{im}}
%\newcommand{\ker}{\operatorname{ker}}
\newcommand{\coker}{\operatorname{coker}}
\newcommand{\id}{\operatorname{id}}
\newcommand{\triv}{(0)}
\newcommand{\del}{\partial}
\newcommand{\Z}{\ensuremath{\mathbb{Z}}}
\newcommand{\N}{\ensuremath{\mathbb{N}}}
\newcommand{\F}{\ensuremath{\mathbb{F}}}
\newcommand{\R}{\ensuremath{\mathbb{R}}}
\newcommand{\Q}{\ensuremath{\mathbb{Q}}}
\newcommand{\CC}{\ensuremath{\mathbb{C}}}
\newcommand{\C}{\ensuremath{\mathcal{C}}}
\newcommand{\Ab}{\typesetConcreteCat{Ab}}
\newcommand{\GAb}{\typesetConcreteCat{GAb}}
\newcommand{\KK}{\typesetConcreteCat{KK}}
\newcommand{\CoKK}{\typesetConcreteCat{Co\kern1.5pt\text{\textbf{-}}KK}}
\newcommand{\typesetConcreteCat}[1]{\ensuremath{\mathbf{#1}}}
% https://tex.stackexchange.com/questions/224805/i-want-a-really-small-underbrace
\makeatletter
\def\smallunderbrace#1{\mathop{\vtop{\m@th\ialign{##\crcr
   $\hfil\displaystyle{#1}\hfil$\crcr
   \noalign{\kern3\p@\nointerlineskip}%
   \scriptsize\upbracefill\crcr\noalign{\kern3\p@}}}}\limits}
\makeatother


\begin{document}
\thispagestyle{empty}
\maketitle

\clearpage

%\renewcommand*{\thesection}{§\arabic{section}}

\setcounter{section}{8}
\section{Homologische Algebra}

\setcounter{prop}{3}
\begin{prop}
  Sei $G$ eine abelsche Gruppe und
  \begin{equation*}
    \begin{tikzcd}
      A \arrow{r}{f}  & B
        \arrow{r}{g}  & C
        \arrow{r}{}   & 0
    \end{tikzcd}
  \end{equation*}
  eine exakte Sequenz abelscher Gruppen. Dann is auch die induzierte Sequenz
  \begin{equation*}
    \begin{tikzcd}
      \Hom(A,G) & \Hom(B,G) \arrow{l}{f^*}
                & \Hom(C,G) \arrow{l}{g^*}
                & 0         \arrow{l}{}
    \end{tikzcd}
  \end{equation*}
  exakt. $\Hom(-,G)$ ist \emph{links-exakt}.
\end{prop}
\begin{proof}
  \begin{enumerate}[(i)]
    \item 
      Exaktheit bei $\Hom(C,G)$.
      Zeige $g^*$ ist injektiv.
      Sei $\varphi \in \Hom(C,G)$ mit $g^*(\varphi) = \varphi \circ g  = 0$.
      \begin{align*}
        \begin{tikzcd}[ampersand replacement=\&]
          B \arrow{rd}{0} \arrow[twoheadrightarrow]{r}{g}  \& C \arrow{d}{\varphi} \\
                                        \& G
        \end{tikzcd}
        & \qquad \implies \varphi = 0
      \end{align*}
      %\begin{equation*}
        %\begin{tikzcd}
          %B \arrow{rd}{0} \arrow[twoheadrightarrow]{r}{g}  & C \arrow{d}{\varphi} \\
                                        %& G
        %\end{tikzcd}
      %\end{equation*}
      %Dann folgt $\varphi = 0$.
    \item
      Exaktheit bei $\Hom(B,G)$:
      \begin{enumerate}[(a)]
        \item 
          $\im g^* \subseteq \ker f^*$, also $f^* \circ g^* = 0$.
          Aber $f^* \circ g^* = {(g \circ f)}^* = 0^* = 0$.
        \item
          $\ker f^* \subseteq \im g^*$:
          Sei $\varphi \colon B \to G \in \ker f^*$, $0 = f^*(\varphi) = \varphi \circ f$.
          \begin{equation*}
            \begin{tikzcd}
              B \arrow[twoheadrightarrow]{r}{g}
                \arrow[twoheadrightarrow]{rd}{\pi}
                \arrow{d}{\phi}
                & C \\
              G & B/{\ker g}  \arrow{u}{\overline g}
                              \arrow{l}{\overline \varphi}
            \end{tikzcd}
          \end{equation*}
          Dann ist $\ker g = \im f \subseteq \ker \varphi$ und daraus folgt die eindeutige Existenz eines $\overline \varphi \colon B/{\ker g} \to G$ mit $\overline \varphi \circ \pi = \varphi$.

          Ebenso induziert $g$ einen Morphismus $\overline g\colon B/{\ker g} \to C$ mit $\overline g \circ \pi = g$.
          Außerdem ist $\overline g$ injektiv und surjektiv, also ein Isomorphismus und somit
          \begin{equation*}
            \varphi = \overline \varphi \circ \pi = \overline \varphi \circ {\overline g}^{-1} \circ g = g^* (\overline \varphi \circ {\overline g}^{-1}).
            \end{equation*}
      \end{enumerate}
  \end{enumerate}
\end{proof}
\begin{kommentar}
  Man sagt, dass der kontravariante Funktor $\Hom(-,G) =: F$ links-exakt ist.
  Beachte, dass $F$ allerdings i.A.\ nicht exakte Sequenzen
  \begin{equation*}
    \begin{tikzcd}
      0 \arrow{r}{}   & A
        \arrow{r}{f}  & B
        \arrow{r}{g}  & C
        \arrow{r}{}   & 0
    \end{tikzcd}
  \end{equation*}
  in exakte Sequenzen überführt.
  \begin{equation*}
    \begin{tikzcd}
      0         & \Hom(A,G) \arrow{l}{}
                & \Hom(B,G) \arrow{l}{f^*}
                & \Hom(C,G) \arrow{l}{g^*}
                & 0         \arrow{l}{}
    \end{tikzcd}
  \end{equation*}
  \begin{tikz}[overlay]
    \draw[dashed] (3.1,.65) rectangle +(2.2,1);
  \end{tikz}
  ist also i.A.\ nicht exakt.
\end{kommentar}
\begin{erinnerung}
  eine exakte Sequenz abelscher Gruppen
  \begin{equation*}
    \begin{tikzcd}
      0 \arrow{r}{}
        & A \arrow{r}{f}
        & B \arrow[bend left]{r}{\theta}
        & C \arrow[bend left]{l}{r}
            \arrow[loop above]{r}{id_C}
            \arrow{r}{}
        & 0
    \end{tikzcd}
  \end{equation*}
  mit $g \circ r = id_C$ \emph{spaltet}.
  Äquivalent:
  \begin{equation*}
    \begin{tikzcd}
      0 \arrow{r}{}
        & A \arrow[bend left]{r}{f}
            \arrow[loop above]{r}{id_A}
        & B \arrow[bend left]{l}{l}
            \arrow{r}{g}
        & C \arrow{r}{}
        & 0
    \end{tikzcd}
  \end{equation*}
  mit $l \circ f = id_A$.

  In diesem Fall gilt $B \cong A \oplus C$.
\end{erinnerung}
\begin{zusatz}
  Ist
  \begin{equation*}
    \begin{tikzcd}
      0 \arrow{r}{}   & A
        \arrow{r}{f}  & B
        \arrow{r}{g}  & C
        \arrow{r}{}   & 0
    \end{tikzcd}
  \end{equation*}
  exakt und spaltet, so ist auch
  \begin{equation*}
    \begin{tikzcd}
      0         & \Hom(A,G) \arrow{l}{}
                & \Hom(B,G) \arrow{l}{f^*}
                & \Hom(C,G) \arrow{l}{g^*}
                & 0         \arrow{l}{}
    \end{tikzcd}
    \tag{$*$}\label{eqn:hom_spaltung}
  \end{equation*}
  exakt und spaltet.
\end{zusatz}
\begin{proof}
  ist $l \colon B \to A$ linksinvers zu $f$, $l \circ f = id_A$, so ist $id_{\Hom(A,G)} = id_{A}^* = {(l \circ f)}^* = f^* \circ l^*$, also ist $f^*$ surjektiv.
  Außerdem ist nun $l^*$ offenbar rechtsinvers zu $f^*$, also eine Spaltung von~\eqref{eqn:hom_spaltung}.
\end{proof}
\begin{defn}
  Sei $A$ eine abelsche Gruppe.
  Dann heißt eine kure exakte Sequenz
  \begin{equation*}
    \begin{tikzcd}
      0 \arrow{r}{}
        & R \arrow{r}{\alpha}
        & F \arrow{r}{\beta}
        & F \arrow{r}{}
        & 0
    \end{tikzcd}
  \end{equation*}
  eine \emph{freie Auflösung}, wenn $F$ eine frei abelsche Gruppe ist.
\end{defn}
\begin{kommentar}
  Als Untergruppe (vie $\alpha$) von $F$ ist $R$ selbst eine frei abelsche Gruppe.
  Ist ${(e_i)}_{i \in I}$ eine Basis von $F$, so ist $\varepsilon = {(\beta(e_i))}_{i \in I}$ ein Erzeugendensystem von $A$.
  Und ist ${(r_j)}_{j \in J}$ eine Basis von $R$, so erzeugt ${(\alpha(r_j))}_{j \in J}$ die Relationen von $\varepsilon$ (\emph{Relationen auf $\varepsilon$}: $f \in F$ mit $\beta(f) = 0$).
\end{kommentar}
\begin{beispiel}
  \begin{enumerate}
    \item 
      ist $A$ selbst frei, s okann man $F = A$ und $\beta = id_A$ wählen (dann $R = \triv$).
    \item
      Ist $A = \Z_4$, so ist
      \begin{equation*}
        \begin{tikzcd}
          0 \arrow{r}{}
            & \Z \arrow{r}{\cdot 2}
            & \Z \arrow{r}{\pi}
            & \Z_2 \arrow {r}{}
            & 0
        \end{tikzcd}
      \end{equation*}
      eine freie Auflösung.
    \item
      Ist $A$ beliebig, so betrachte $A$ als menge und setze $F = \F(A)$ und $\pi\colon F \to A$ der Homorphismus, der auf der Basis ${(i(a))}_{a \in A}$ durch $\pi(i(a)) = a$ gegeben ist.
      Natürlich ist dann $\pi(2 \cdot a) = \pi(1\cdot (2a)) = 2a$ und $\pi(0_A) = \pi(0_{\F(A)}) = 0_A$.
      Ist $R = \ker \pi$ und $j\colon R \hookrightarrow F$ die Inklusion, so ist
      \begin{equation*}
        \begin{tikzcd}
          0 \arrow{r}{}
            & R \arrow{r}{j} 
            & F \arrow{r}{\pi}
            & A \arrow{r}{}
            & 0
        \end{tikzcd}
      \end{equation*}
      offenbar exakt (weil $\pi$ surjektiv ist).
      Das ist die \emph{Standardauflösung} $S(A)$ von $A$:
      \begin{equation*}
        \begin{tikzcd}
          S(A):
          0 \arrow{r}{}
            & R \arrow{r}{j} 
            & F \arrow{r}{\pi}
            & A \arrow{r}{}
            & 0
        \end{tikzcd}
      \end{equation*}
  \end{enumerate}
\end{beispiel}


Ist
\begin{equation*}
  \begin{tikzcd}
    0 \arrow{r}{}
      & A \arrow{r}{f}
      & B \arrow{r}{g}
      & C \arrow{r}{}
      & 0
  \end{tikzcd}
\end{equation*}
exakt, so ist
\begin{equation*}
  \begin{tikzcd}[arrows=leftarrow]
    ? \arrow{r}{}
      & \Hom(A,G) \arrow{r}{f^*}
      & \Hom(B,G) \arrow{r}{g^*}
      & \Hom(C,G) \arrow{r}{}
      & 0
  \end{tikzcd}
  \tag{$*$}\label{eqn:kes_unvollstaendig}
\end{equation*}
ist exakt, aber $f^*$ i.A.\ nicht surjektiv.
Naheliegend könnte man~\eqref{eqn:kes_unvollstaendig} mit
\begin{equation*}
  \coker f^* := \Hom(A,G) / {\im f^*}
\end{equation*}
und
\begin{equation*}
  \nu\colon \Hom/A,G) \to \coker f^*
\end{equation*}
fortsetzen, was aber so aussieht, dass es von zu vielen Wahlen abhängt.
\begin{defn}
  Seien $A$ und $G$ abelsche Gruppen und $S(A)$ die Standardauflösung von $A$.
  Dann nennt man
  \begin{align*}
    \Ext(A,G) &:= \coker i^* = \Hom(R,g) / {\im i^*}, \\
    i^* & \colon \Hom(F,G) \to \Hom/R,G)
  \end{align*}
  das \emph{Extensionsprodukt} (kurz: Ext-Produkt) \emph{von $A$ und $G$}.
\end{defn}
\begin{kommentar}
  Für die Standardauflösung $S(A)\colon \begin{tikzcd} 0 \arrow{r}{} &R\arrow{r}{i} &F\arrow{r}{\pi} &A\arrow{r}{}&0\end{tikzcd}$ von $A$ wird dann also mit der kanonischen Projektion $\nu\colon \Hom(R,G) \to \Ext(A,G)$ die folgende Sequenz exakt:
  \begin{equation*}
    \begin{tikzcd}
      0 \arrow{r}{}
        & \Hom(C,G) \arrow{r}{g^*}
        & \Hom(B,G) \arrow{r}{f^*}
        & \Hom(A,G) \arrow{r}{\nu}
        & \Ext(A,G) \arrow{r}{}
        & 0
    \end{tikzcd}
  \end{equation*}
\end{kommentar}
\emph{Frage}: Wie ist das mit anderen freien Auflösungen?
\begin{lemma}
  Seien $A$ und $A'$ abelsche Gruppen und
  \begin{equation*}
    \begin{tikzcd}[row sep=tiny]
      S\colon 0  \arrow{r}{}  & R \arrow{r}{f}  & F \arrow{r}{g}  & A \arrow{r}{} & 0 \\
      S'\colon 0 \arrow{r}{}  & R \arrow{r}{f'} & F \arrow{r}{g'} & A \arrow{r}{} & 0
    \end{tikzcd}
  \end{equation*}
  freie Auflösungen sowie $h\colon A \to A'$ ein Homomorphismus.
  \begin{equation*}
    \begin{tikzcd}
      0 \arrow{r}{} & R   \arrow{d}{\alpha}
                          \arrow{r}{f}  & F   \arrow{d}{\beta}
                                              \arrow{r}{g}  & A   \arrow{d}{h}
                                                                  \arrow{r}{} & 0 \\
      0 \arrow{r}{} & R'  \arrow{r}{f'} & F'  \arrow{r}{g'} & A'  \arrow{r}{} & 0 
    \end{tikzcd}
    \tag{$*$}\label{eqn:frei_aufl_hom}
  \end{equation*}
  \begin{enumerate}[(a)]
    \item 
      Dann existieren Homomorphismen $\alpha \colon R \to R'$ und $\beta \colon F \to F'$, die das Diagramm~\eqref{eqn:frei_aufl_hom} kommutieren lassen.
      \begin{equation*}
        \begin{tikzcd}
      0 \arrow{r}{} & R   \arrow[swap]{d}{\tilde\alpha - \alpha}
                          \arrow{r}{f}  & F   \arrow{d}{\tilde\beta - \beta}
                                              \arrow{dl}{\varphi}
                                              \arrow{r}{g}  & A   \arrow{d}{h-h = 0}
                                                                  \arrow{r}{} & 0 \\
      0 \arrow{r}{} & R'  \arrow[swap]{r}{f'} & F'  \arrow{r}{g'} & A'  \arrow{r}{} & 0 
        \end{tikzcd}
        \tag{$**$}\label{eqn:frei_aufl_homotopie}
      \end{equation*}
    \item
      Sind $\tilde\beta \colon F \to F'$ und $\tilde\alpha \colon R \to R'$ weitere Homomorphismen, die~\eqref{eqn:frei_aufl_hom} kommutieren lassen, so existiert ein $\varphi \colon F \to R'$ mit
      \begin{equation*}
        f' \circ \varphi = \tilde\beta - \beta, \qquad \varphi \circ f = \tilde\alpha - \alpha.
      \end{equation*}
  \end{enumerate}
\end{lemma}
\begin{proof}
  \begin{enumerate}[(a)]
    \item 
      Sei ${(e_i)}_{i \in I}$ eine Basis von $F$.
      Betrachte dann $h \circ g (e_i) \in A'$.
      Da $f'$ surjektiv ist, existiert $x_i \in F'$ mit $f'(x_i) = h \circ g(e_i)$.
      Definiere dann $\beta \colon F \to F'$ durch $\beta(e_i) := x_i$.
      Dann gilt für alle $i \in I$:
      \begin{equation*}
        g' \circ \beta(e_i) = g'(x_i) = h \circ g(e_i)
      \end{equation*}
      und somit $g' \circ \beta = h \circ g$.

      Für alle $x \in R$ ist
      \begin{align*}
        g' \circ (\beta \circ f) (x)
          & = (g' \circ \beta) \circ f (x) \\
          & = h \circ \underset{=0}{\underbrace{g \circ f(x)}}
      \end{align*}
      und somit $\im (\beta \circ f) \subseteq  \ker g' = \im f'$.
      Bezeichne mit $f'\colon R' \to \im f'$ auch den Homomorphismus $f'$, wenn man der Wertebereich auf $\im f'$ einschränkt.
      Dann ist (dieses) $f'$ ein Isomorphismus, da $f'$ injektiv und (nun auch) surjektiv ist.
      Setze dann
      \[
        \alpha \colon R \to R',\quad \alpha := {(f')}^{-1} \circ \beta \circ f.
      \]
      Dann ist offenbar
      \begin{equation*}
        f' \circ \alpha = \beta \circ f.
      \end{equation*}
    \item
      Für  alle $x \in F$ ist
      \begin{equation*}
        g' \circ (\tilde\beta - \beta) (x) = (g' \circ \tilde \beta - g' \circ \beta) (x) = h\circ g(x) - h \circ g(x) = 0
      \end{equation*}
      und somit $\im (\tilde\beta - \beta) \subseteq \ker g' = \im f'$.
      Setze daher $\varphi \colon F \to R'$ mit $\varphi := {(f')}^{-1} \circ (\tilde \beta - \beta)$, dann gilt $f' \circ \varphi = \tilde \beta - \beta$.
      Es ist aber auch:
      \begin{align*}
        f' \circ (\tilde\alpha - \alpha)
          & = f' \circ \tilde \alpha - f' \circ \alpha \\
          & = \tilde \beta \circ f - \beta \circ f \\
          & = (\tilde \beta - \beta) \circ f \\
          & = f' \circ \varphi \circ f.
      \end{align*}
      Da $f'$ injektiv ist, folgt $\tilde \alpha - \alpha = \varphi \circ f$.
  \end{enumerate}
\end{proof}
\begin{prop}
  \label{thm:hom_freier_aufloesungen}
  Seien $A$, $A'$ abelsche Gruppen, $S$ und $S'$ freie Auflösungen von $A$ und $A'$,
  \begin{equation*}
    \begin{tikzcd}
      S\colon\,0 \arrow{r}{} & R   \arrow{d}{f}
                              \arrow{r}{j}  & F   \arrow{d}{}
                                                  \arrow{r}{}   & A   \arrow{d}{h}
                                                                      \arrow{r}{} & 0, \\
      S'\colon 0 \arrow{r}{} & R'  \arrow{r}{j'} & F'  \arrow{r}{}   & A'  \arrow{r}{} & 0,
    \end{tikzcd}
  \end{equation*}
  und sei $h\colon A \to A'$ Homomorphismus.
  Dann gibt es genau einen Homomorphismus
  \begin{equation*}
    \Phi(h;S,S') \colon \coker {(j')}^* \to \coker j^*,
  \end{equation*}
  sodass gilt: Sind $g \colon F \to F'$ und $f \colon R \to R'$ Homomorphismen derart, dass $(f,g,h) \colon S \to SÄ$ Homomorphismus zwischen freien Auflösungen ist, d.h.~\eqref{eqn:frei_aufl_hom} kommutiert, so kommutiert auch (für jede abelsche Gruppe $G$):
  \begin{equation*}
    \begin{tikzcd}
      0 \arrow{r}{}   & \Hom(A,G)   \arrow{r}{}   & \Hom(F,G) \arrow{r}{j^*}        & \Hom(R,G) \arrow{r}{\nu}  & \coker j^* \arrow{r}{}      & 0 \\
      0 \arrow{r}{}   & \Hom(A',G)  \arrow{u}{h^*}
                                    \arrow{r}{}   & \Hom(F',G)\arrow{u}{g^*}
                                                              \arrow{r}{{(j')}^*}  & \Hom(R',G) \arrow{u}{f^*}
                                                                                                \arrow[swap]{ul}{\alpha^*}
                                                                                                \arrow{r}{\nu} & \coker {(j')}^* \arrow{u}{\Phi}
                                                                                                                              \arrow{r}{} & 0
    \end{tikzcd}
  \end{equation*}
\end{prop}
\begin{kommentar}
  Das Besondere ist nicht so sehr die Existenz und Eindeutigkeit von $\Phi$, denn es gibt nur einen Kandidaten
  \begin{equation*}
    \Phi([\varphi]) = [f^*(\varphi)]  \qquad \text{für } \varphi \in \Hom(R',G), 
  \end{equation*}
  sondern mehr die Unabhängigkeit von $f$ (und $g$).
\end{kommentar}
\begin{proof}[Beweis von Proposition~\ref{thm:hom_freier_aufloesungen}]
  Wegen $f^* \circ {(j')}^* = j^* \circ g^*$ ist
  \begin{equation*}
    f^* (\im {(j')} ^*) \subseteq \im j^*
  \end{equation*}
  und somit $\nu \circ f^* (\im {(j')}^* = 0$, also existiert eindeutiges $\Phi \colon \coker {(j')}^* \to \coker j^*$ mit $\Phi \circ \nu' = \nu \circ f^*$, also
  \begin{equation*}
    \Phi([\varphi]) = [f^*(\varphi)] \quad \text{für alle $\varphi \in \Hom(R',G)$}.
  \end{equation*}
  Ist $(\tilde f, \tilde g)$ eine weitere Wahl, die~\eqref{eqn:frei_aufl_hom} kommutieren lässt, und $\alpha \colon F \to R'$ mit $\alpha \circ j = \tilde f - f$ (und $j' \circ \alpha = \tilde g - g$), so ist $j^* \circ \alpha^* = {\tilde f}^* - f^*$ und somit ${\tilde f}^* (\varphi) - f^*(\varphi) \in \im j^*$.
  Macht daher $\tilde\Phi$~\eqref{eqn:frei_aufl_hom} kommutativ (mit $\tilde f$ statt $f$ und $\tilde g$ statt $g$), so ist
  \begin{equation*}
    \tilde\Phi ([\varphi]) = [f^*(\varphi)] = [{\tilde f}^*(\varphi)] = \Phi ([\varphi]) \quad \text{für alle $\varphi \in \Hom(R',G)$}.
  \end{equation*}
\end{proof}
\begin{korollar}
  Definiert man eine Kategorie \C\ dadurch, dass man als Objekte freie Auflösungen $S$ von abelschen Gruppen $A$ betrachtet und als Morphismen solche von $A$ nach $A'$,
  \begin{equation*}
    \begin{tikzcd}
      S\colon 0 \arrow{r}{}  & R \arrow{r}{j}  & F \arrow{r}{} & A \arrow{d}{h}
                                                              \arrow{r}{}   & 0 \\
      \phantom{S\colon}
      0 \arrow{r}{}  & R'\arrow{r}{j}  & F'\arrow{r}{} & A'\arrow{r}{}   & 0,
    \end{tikzcd}
  \end{equation*}
  so ist die Zuordnung (für gegebene $G$) $\Phi \colon \C \to \Ab$ auf Objekten $\Phi(S) := \coker j^*$ und auf Morphismen $h \mapsto \Phi(h;S,S')$, wie in der Proposition, funktoriell:
  \begin{enumerate}[(a)]
    \item 
      $\Phi(\id;S,S) = \id$
    \item
      $\Phi(h' \circ h;S,S'') = \Phi(h;S,S') \circ \Phi(h';S',S'')$ für alle $h,h',S,S',S''$ etc.
    \end{enumerate}
\end{korollar}
\begin{proof}
  \begin{enumerate}[(a)]
    \item 
      Ist $h = \id\colon S \to S$, so kann $f = \id$ und $g = \id$ gewählt werden und man erhält: $\Phi = \id$.
    \item
      Sind $(f,g)$ bzw.\ $(f',g')$ für $h \colon S \to S'$ bzw.\ $h'\colon S' \to S''$ gewählt, so kann man für $(f'',g'')$ die Wahlen $f'' = f' \circ f$ und $g'' = g' \circ g$ treffen, weil das Diagramm~\eqref{eqn:neu} dann als Zusammensetzung von~\eqref{eqn:} und~\eqref{eqqn:} kommutiert.
      \todo{Fill the gaps}
      Somit gilt
      \begin{align*}
        \Phi(h' \circ h; S,S'')
          & = [{(f' \circ f)}^* (\varphi)]\\
          & = [f^* \circ {(f')}^* (\varphi)]\\
          & = \Phi(h;S,S') \circ \Phi(h';S',S'').
      \end{align*}
  \end{enumerate}
\end{proof}



\begin{kommentar}
  \begin{enumerate}[(a)]
    \item
      Ist $h\colon A \to A'$ Isomorphismus, so ist
      \begin{equation*}
        \Phi(h;S,S')\colon \coker {(j')}^* \to \coker j^*
      \end{equation*}
      ebenfalls Isomorphismus, denn $\Phi(h^{-1};S',S) = {\Phi(h;S,S')}^{-1}$.
    \item
      Ist insbesondere $S(A)$ die Standardauflösung von $A$, $S$ eine beliebige freie Auflösung von $A$ und $h = \id_A$, so ist also das induzierte
      \begin{equation*}
        \Phi(\id;S,S(A))\colon \coker i^* = \Ext(A,G) \to \coker j^*
      \end{equation*}
      ein (kanonischer) Isomorphismus.
      Daher hängt der Isomorphietyp von $\Ext(A,G)$ nicht von der Wahl der freien Auflösung von $A$ ab.
  \end{enumerate}
\end{kommentar}
\begin{beispiel}
  \begin{enumerate}[(a)]
    \item
      ist $A$ frei-abelsch und $G$ beliebig, so ist
      \begin{equation*}
        \Ext(A,G) = \triv,
      \end{equation*}
      denn:
      \begin{equation*}
        \begin{tikzcd}
          S\colon 0 \arrow{r}{}  & 0 \arrow{r}{0}  & A \arrow{r}{\id}  & A \arrow{r}{} & 0
        \end{tikzcd}
      \end{equation*}
      ist freie Auflösung und $\coker 0^* = \triv$.
    \item
      Für zwei abelsche Gruppen $A_1$ und $A_2$ (und $G$ beliebig) ist
      \begin{equation*}
        \Ext(A_1 \oplus A_2,G) \cong \Ext(A_1,G) \oplus \Ext(A_2,G),
      \end{equation*}
      denn: Sind
      \begin{equation*}
        \begin{tikzcd}[row sep=tiny]
          S_1\colon 0 \arrow{r}{}  & R_1 \arrow{r}{j_1}  & F_1 \arrow{r}{}  & A_1 \arrow{r}{} & 0 \\
          S_2\colon 0 \arrow{r}{}  & R_2 \arrow{r}{j_2}  & F_2 \arrow{r}{}  & A_2 \arrow{r}{} & 0
        \end{tikzcd}
      \end{equation*}
      freie Auflösungen, so ist
      \begin{equation*}
        \begin{tikzcd}[row sep=tiny]
          S_1 \oplus S_2\colon 0 \arrow{r}{}  & R_1 \oplus R_2 \arrow{r}{j_1 \oplus j_2}  & F_1 \oplus F_2 \arrow{r}{}  & A_1 \oplus A_2 \arrow{r}{} & 0
        \end{tikzcd}
      \end{equation*}
      eine freie Auflösung von $A_1 \oplus A_2$ und daher ist
      \begin{align*}
        \Ext(A_1 \oplus A_2,G) \cong \coker {(j_1 \oplus j_2)}^*
          & \cong \coker (j_1^* \oplus j_2^*) \\
          & \cong \coker j_1^* \oplus \coker j_2^* \\
          & \cong \Ext(A_1,G) \oplus \Ext(A_2,G).
      \end{align*}
    \item
      Für jede abelsche Gruppe $G$ und $n \in \N_0$ sei
      \begin{equation*}
        n G := \{ \underset{\mathclap{g+\dotsb+g \ (\text{$n$-mal})}}{\underbrace{ng}} \in G : g \in G \} \leq G
      \end{equation*}
      Denn gilt für $\Z_n := \Z / {Z_n}$:
      \begin{equation*}
        \Ext(\Z_n, G) = G/{nG},
      \end{equation*}
      denn
      \begin{equation*}
        \begin{tikzcd}
          0 \arrow{r}{} & \Z \arrow{r}{\cdot n}[swap]{=:j}  & \Z \arrow{r}{\pi} & \Z_n \arrow{r}{}  & 0
        \end{tikzcd}
      \end{equation*}
      ist eine freie Auflösung von $\Z_n$.
      Identifiziert man nun noch $\Hom(\Z,G)$ mit $G$ (vermöge $\varphi \mapsto \varphi(1)$), so erhält man, dass $j^*\colon G \to G, j^*(g) = n \cdot g$ ist.
      Daher ist
      \begin{equation*}
        \Ext(\Z_n,G) \cong \coker j^* \cong G/{nG}.
      \end{equation*}
      \emph{Bemerkung}: Beachte, dass also
      \begin{equation*}
        \Ext(\Z_n,\Z) \cong \Z/{n\Z} = \Z_n \neq \triv = \Ext(\Z,\Z_n)
      \end{equation*}
      für $n \ge 2$.
      Also ist $\Ext$ (im Unterschied zu $\operatorname{Tor}(-,-)$, gebildet ganz ähnlich wie $\Ext$ aus $\Hom$, aus $-\oplus-$) im Allgemeinen nicht symmetrisch in seinen beiden Argumenten, genauso wie $\Hom(-,-)$ auch, denn z.B.\ ist
      \begin{equation*}
        \Hom(\Z,\Z_n) \cong \Z_n \neq \triv = \hom(\Z_n,\Z) \quad \text{für $n\geq 2$}.
      \end{equation*}
    \item
      Ist insbesondere $G$ ein Körper der Charakteristik 0 (z.B.\ $G = \Q,\R,\CC$), so ist also für alle $n \in \N$
      \begin{equation*}
        n = 1 + \dotsb + 1 \ \text{($n$-mal)}
      \end{equation*}
      verschieden von $0$, also existiert $n^{-1} \in G$ und damit ist $nG = G$, also ist $g/{nG} = \triv$.
      In diesem Fall ist also
      \begin{equation*}
        \Ext(\Z_n,G) = \triv.
      \end{equation*}
    \item
      Für endlich erzeugte abelsche Gruppe $A$, mit $\mathrm{Tor}(A) \coloneqq \left\{ a \in A : \exists m \in \N : m \cdot a = 0 \right\}$, wo also
      \begin{equation*}
        A \cong \Z^b \oplus \mathrm{Tor}(A) \cong \Z^b \oplus \Z_{n_1} \oplus \dotsb \oplus \Z_{n_r}
      \end{equation*}
      (mit $b \in \N_0$ und $n_i \in \N^{\ge 2}$) ist, erhält man also
    \begin{equation*}
      \Ext(A,G) = \triv
    \end{equation*}
    für jeden Körper $G$ der Charakteristik 0.
  \end{enumerate}
\end{beispiel}
\begin{defn}
  Sei $G$ fest und $f\colon A \to B$ Homomorphismus abelscher Gruppen.
  Wir nennen dann
  \begin{equation*}
    f^* = \Phi(f;S(A),S(B))\colon \Ext(B,G) \to \Ext(A,G)
  \end{equation*}
  den \emph{von $f$ induzierten Homomorphismus}.
\end{defn}
\begin{kommentar}
  Es wird dann
  \begin{equation*}
    T := \Ext(-,G)\colon \Ab \to \Ab
    \end{equation*}
    ein kontravarianter Funktor,
    \begin{align*}
      T(A) & = \Ext(A,G) \\
      T(f) & = \Ext(f,G) = f^*,
    \end{align*}
    denn
    \begin{align*}
      T(\id) & = \id^* = \id \\
      T(g \circ f) & = T(f) \circ T(g)
    \end{align*}
\end{kommentar}


\section{Cohomologie}

\begin{defn}
  Ein \emph{Cokettenkomplex} $(C,\delta)$ besteht aus einer Familie ${(C^k)}_{m \in \Z}$ abelscher Gruppen und Homomorphismen (\emph{Corandoperatoren}) $\delta^k\colon C^k \to C^{k+1}$ mit $\delta^{k+1} \circ \delta^k = 0$ für alle $k \in \Z$.
\end{defn}
\begin{kommentar}
  \begin{enumerate}
    \item
      Setzt man für einen Cokettenkomplex $( (C^k), (\delta^k))$
      \begin{equation*}
        C_{-k} := C^k, \quad \del_k\colon C_k \to C_{k-1}, \del_k := \delta^{-k},
      \end{equation*}
      so erhält man einen Kettenkomplex $( (C_k), (\del_k) )$ und umgekehrt.
      Auf diese übertragen sich alle Konzepte der Kettenkomplexe auf solche von Cokettenkomplexen, z.B.:
      \begin{enumerate}[(i)]
        \item
          Es heißen die Elemente von $C^k$ \emph{Coketten}, die Elemente von
          \begin{equation*}
            Z^k := \ker \delta^k \subseteq C^k
          \end{equation*}
          \emph{Cozyklen} und die Elemente von
          \begin{equation*}
            B^k := \im \delta^{k-1} \subseteq C^k
          \end{equation*}
          \emph{Coränder} und wegen $\delta^{\circ 2} = 0$ ist $B^k \subseteq  Z^k$.
        \item
          Man nennt $H^k(C) := Z^k/{B^k}$ die $k$-te \emph{Cohomologiegruppe} von $(C,\delta)$ und ihre Elemente \emph{Cohomologieklassen}.
        \item
          Für zwei Cokettenkomplexe $(C,\delta)$ und $(C',\delta')$ heißt die Familie von Homomorphismen $f = {(f^k\colon C^k \to {C'}^k)}_{k \in \Z}$ eine \emph{Cokettenabbildung}, falls
          \begin{equation*}
            {\delta'}^k \circ f^k = f^{k+1} \circ \delta^k \quad \text{für alle $k \in \Z$}
          \end{equation*}
          ist.
          Eine solche induziert dann einen Homomorphismus
          \begin{align*}
            f_{*,k}\colon H^k(C) & \to H^k(C') \\
            [\alpha] & \mapsto [f^k(\alpha)] \qquad \text{für alle $\alpha \in Z^k$}
          \end{align*}
        \item
          Eine Familie von Homomorphismen $D = {(D^k \colon C^k \to {(C')}^{k-1})}_{k \in \Z}$ heißt eine \emph{Coketten-Homotopie von $f$ nach $g$} für zwei Cokettenabbildungen $f,g \colon C \to C'$, wenn für alle $k \in \Z$ gilt:
          \begin{equation*}
            \delta^{k-1} \circ D^k + D^{k+1} \circ \delta^k = g^k - f^k.
          \end{equation*}
      \end{enumerate}
    \item
      Weiter übertragen sich Resultate über Kettenkomplexe, z.B.\
      \begin{enumerate}
        \item
          Ist
          \begin{equation*}
            \begin{tikzcd}
              0 \arrow{r}{} & C' \arrow{r}{f} & C \arrow{r}{g}  & C'' \arrow{r}{} & 0
            \end{tikzcd}
          \end{equation*}
          kurze exakte Sequenz von Cokettenkomplexen, so existiert ein natürlicher Homomorphismus (sogar eine natürliche Transformation)
          \begin{equation*}
            \delta^k_*\colon H^k(C'') \to H^{k+1}(C'),
          \end{equation*}
          sodass folgende \emph{lange Cohomologiesequenz} exakt wird:
          \begin{equation*}
            \begin{tikzcd}
              \dotsb \arrow{r}{}  & H^k(C') \arrow{r}{f^*}  & H^k(C) \arrow{r}{g^*} & H^k(C'') \arrow{r}{\delta^k_*}  & H^{k+1}(C') \arrow{r}{} & \dotsb
            \end{tikzcd}
          \end{equation*}
        \item
          Induziert für zwei Teilcokomplexe $C,C'' \subseteq C$ die Inklusion $i\colon C' + C'' \to C$ einen Isomorphismus in der Cohomologie
          \begin{equation*}
            i_*\colon H(C' + C'') \overset{\cong}{\longrightarrow} H(C),
            %i_*\colon H(C' + C'') \xrightarrow{\cong} H(C),
          \end{equation*}
          dann heißt $(C',C'')$ ein \emph{Ausschneidungspaar} von $C$ und man erhält eine exakte Sequenz (die Maier-Vietoris-Sequenz)
          \begin{equation*}
            \begin{tikzcd}[column sep=small]
              \dotsb \arrow{r}{}  & H^k(C' \cap C'') \arrow{r}{\mu}  & H^k(C') \oplus H^k(C'') \arrow{r}{\nu} & H^k(C) \arrow{r}{\Delta}  & H^{k+1}(C' \cap C'') \arrow{r}{} & \dotsb
            \end{tikzcd}
          \end{equation*}
          wobei $\mu$, $\nu$ und $\Delta$ entsprechend definiert werden.
      \end{enumerate}
  \end{enumerate}
\end{kommentar}
\begin{beispiel}[wichtig]
  Sei nun ${( C_k, \del_k )}_{k \in \Z}$ eine Kettenkomplex
  \begin{equation*}
    \begin{tikzcd}
      \dotsb \arrow{r}{}  & C_{k+1} \arrow[bend right]{rr}{0}
                            \arrow{r}{\del_{k+1}}     & C_{k} \arrow{r}{\del_k} & C_{k-1} \arrow{r}{} & \dotsb
    \end{tikzcd}
  \end{equation*}
  und $G$ eine feste abelsche Gruppe.
  Durch Anwenden von $\Hom(-,G)$ erhält man dann den \emph{zugehörigen Cokettenkomplex ${(C^k,\delta^k)}_{k \in \Z}$ mit Koeffizienten in $G$} durch
  \begin{align*}
    C^k & := \Hom(C_k,G) \\
    \delta^k \colon & C^k \to C^{k+1}, \ \delta^k := \del_{k+1}^*
  \end{align*}
  Weil $\Hom(_,G)$ funktoriell ist, ist $(C^k,\delta^k)$ tatsächlich ein Cokettenkomplex,
  \begin{equation*}
    \delta^{k+1} \circ \delta^k = \del_{k+2}^* \circ \del_{k+1}^* = {(\del_{k+1} \circ \del_{k+2})}^* = 0^* = 0.
  \end{equation*}
\end{beispiel}


\begin{kommentar}
  Ist $f\colon C \to C'$ eine Kettenabbildung, so ist
  \begin{equation*}
    f^*\colon \Hom(C',G) \to \Hom(C,G)
  \end{equation*}
  eine Cokettenabbildung und induziert damit einen Homomorphismus in der Cohomologie
  \begin{equation*}
    f^* := f_*^* \colon H^k(C',G) \to H^k(C,G).
  \end{equation*}
  Die \emph{Cohomologie mit Koeffizienten in $G$} wird damit ein kontravarianter Funktor von $\KK \xrightarrow{H} \GAb$.
  Er ist die Hintereinanderausführung des kontravarianten $\Hom(-,G)\colon \KK \to \CoKK$ mit dem Cohomologie-Funktor $H\colon \CoKK \to \GAb$.
\end{kommentar}
\emph{Frage:} Wie passen denn die (ganzzahlige) Homologie ${(H_k(C))}_{k\in\Z}$, die Cohomologie ${(H^k(C,G))}_{k\in\Z}$ und $G$ zusammen?
Ist vielleicht $H^k(C,G) \underset{\text{kan.}}{\cong} \Hom(H_k(C),G)$?
\begin{defn}
  Sei $G$ abelsche Gruppe, $(C_k,\del_k)$ ein Kettenkomplex, $(C^k,\delta^k)$ der induzierte Cokettenkomplex, $C^k = \Hom(C_k,G)$.
  Wir notierten mit $\langle -,- \rangle \colon C^k \times C_k \to G$ die folgende natürliche Paarung:
  \begin{equation*}
    \langle \varphi,c \rangle := \varphi(c) \qquad \text{für alle } \varphi\in C^k, c \in C_k.
  \end{equation*}
\end{defn}
\begin{kommentar}
  \begin{enumerate}
    \item 
      $\langle -,- \rangle$ ist bilinear, also
      \begin{align*}
        \langle \varphi+\varphi', c \rangle & = \langle \varphi, c \rangle + \langle \varphi',c \rangle \\
        \langle \varphi, c+c' \rangle & = \langle \varphi, c \rangle + \langle \varphi,c' \rangle
      \end{align*}
      für alle $\varphi,\varphi' \in C^k$, $c,c' \in C_k$.

      Beachte aber, dass der Homomorphismus
      \begin{align*}
        C_k & \to \Hom(C^k,G)\\
        c   & \mapsto [ \varphi \mapsto \langle \varphi, c \rangle ]
      \end{align*}
      im Allgemeinen weder injektiv noch surjektiv ist.
    \item
      Es gilt dann die \emph{Corand-Rand-Formel}
      \begin{equation*}
        \langle \delta\varphi,c \rangle = \langle \varphi, \del c \rangle \qquad \text{für alle $\varphi \in C^k$, $c \in C_{k+1}$}
      \end{equation*}
      denn
      \begin{equation*}
        \langle \delta\varphi, c \rangle = \delta \varphi (c) = \varphi (\del c) = \langle \varphi, \del c \rangle.
      \end{equation*}
  \end{enumerate}
\end{kommentar}
\begin{bemerkung}
  \begin{enumerate}
    \item 
      Ist $\alpha \in C^k$ ein Cozyklus und $z \in C_k$ ein Rand, so ist $\langle \alpha, z \rangle = 0$.
    \item
      Ist $\alpha \in C^k$ ein Corand und $z \in C_k$ ein Zyklus, so ist $\langle \alpha, z \rangle = 0$.
  \end{enumerate}
\end{bemerkung}
\begin{proof}
  \begin{enumerate}
    \item 
      Ist $z = \del c \in B_k$, $\alpha \in Z^k$, so gilt
      \begin{equation*}
        \langle \alpha,z \rangle = \langle \alpha, \del c \rangle = \langle \smallunderbrace{\delta\alpha}_{=0}, c \rangle = 0.
      \end{equation*}
    \item
      Ist $\alpha = \delta \varphi \in B^k$, $z \in Z_k$, so gilt
      \begin{equation*}
        \langle \alpha,z \rangle = \langle \delta\varphi, z \rangle = \langle \varphi, \smallunderbrace{\del z}_{=0} \rangle = 0.
      \end{equation*}
  \end{enumerate}
\end{proof}
\begin{kommentar}
  nach dem Homomorphiesatz drückt sich daher die Einschränkung der natürlichen Paarung zwischen Cohomologie und Homologie herunter,
  \begin{equation*}
    \begin{tikzcd}
      Z^k \times Z_k  \arrow[swap]{d}{\pi^k \times \pi_k}
                      \arrow{r}{\langle -,- \rangle} & G \\
      H^k \times H_k \arrow[swap]{ur}{\langle \,\boldsymbol\cdot\,,\,\boldsymbol\cdot\, \rangle}
    \end{tikzcd}
  \end{equation*}
  wobei $\pi^k \colon Z^k \to H^k$ und $\pi_k \colon Z_k \to H_k$ die natürlichen Projektionen sind.
\end{kommentar}
Es ist also wohldefiniert:
\begin{defn}
  Sei $C = (C_k,\del_k)$ ein Kettenkomplex, $G$ eine abelsche Gruppe und $(C^k,\delta^k)$ der induzierte Cokettenkomplex.
  Dann definiert man die natürliche Paarung
  \begin{equation*}
    \langle - , - \rangle\colon H^k(C,G) \times H_k(C) \to G
  \end{equation*}
  durch
  \begin{equation*}
    \langle [\alpha], [z] \rangle := \langle \alpha, z \rangle.
  \end{equation*}
\end{defn}
\begin{kommentar}
  \begin{enumerate}
    \item 
      Da $\langle - ,- \rangle \colon C^k \times C_k \to G$ bilinear ist, erhält man also einen (natürlichen) Homomorphismus
      \begin{align*}
        \kappa \colon H^k(C,G) & \to \Hom(H_k(C),G) \\
        \kappa ([\alpha])([z]) & := \langle [\alpha],[z] \rangle
      \end{align*}
    \item
      \emph{Frage}: ist das ein Isomorphismus?
  \end{enumerate}
\end{kommentar}
\stepcounter{prop}
\textbf{(\theprop) Ab sofort.\ }
Sei unser Kettenkomplex $C$ \emph{frei}, d.h.\ alle abelschen Gruppen $C_k$ seien frei.
Wir benutzen weiter, dass jede Untergruppe einer frei-abelschen Gruppe selbst auch wieder frei ist.
\begin{vorbereitung}
  \begin{enumerate}
    \item 
      Bezeichne zunächst mit $i_k$ und $j_k$ die Inklusionen $i_k \colon B_k \hookrightarrow Z_k$ und $j_k \colon Z_k \hookrightarrow C_k$ und mit $\del'\colon C_k \to B_{k-1}, \del'_k(c) := \del_k(c)$, sodass also
      \begin{equation*}
        j_{k-1} \circ i_{k-1} \circ \del' = \del.
      \end{equation*}
      Betrachte nun die exakte Sequenz
      \begin{equation*}
        \begin{tikzcd}
          0 \arrow{r}{} & Z_k \arrow{r}{j_k}  & C_k \arrow[bend left]{l}{l_k}
                                                    \arrow{r}{\del'_k}        & B_{k-1} \arrow{r}{} & 0
        \end{tikzcd}
        \tag{$*$}\label{eqn:inklusionssequenz}
      \end{equation*}
      Mit $C_{k-1}$ ist auch $B_{k-1} \subseteq C_{k-1}$ frei, deshalb spaltet~\eqref{eqn:inklusionssequenz}.
      Es gibt also ein Linksinverses $l_k\colon C_k \to Z_k$ zu $j_k$, $l_k \circ j_k = \id_{Z_k}$.
      Es ist deshalb auch exakt (und spaltet):
      \begin{equation*}
        \begin{tikzcd}
          0 \arrow{r}{} & \Hom(B_{k-1},G) \arrow{r}{{\del'_k}^*}  & \Hom(C_k,G) \arrow{r}{j_k^*}  & \Hom(Z_k,G) \arrow[bend left]{l}{l_k^*}
                                                                                                              \arrow{r}{} & 0.
        \end{tikzcd}
      \end{equation*}
    \item
      Betrachte außerdem die kurze exakte Sequenz
      \begin{equation*}
        \begin{tikzcd}
          0 \arrow{r}{} & B_{k-1} \arrow{r}{i_{k-1}}  & Z_{k-1} \arrow{r}{p_{k-1}}  & H_{k-1} \arrow{r}{} & 0.
        \end{tikzcd}
        \tag{$**$}\label{eqn:inklusion_projektion_sequenz}
      \end{equation*}
      Es ist dann auch exakt:
      \begin{equation*}
        \begin{tikzcd}
          0 \arrow{r}{} & \Hom(H_{k-1},G) \arrow{r}{p_{k-1}^*}  & \Hom(Z_{k-1},G) \arrow{r}{i_{k-1}^*}  & \Hom(B_{k-1},G).
        \end{tikzcd}
      \end{equation*}
  \end{enumerate}
\end{vorbereitung}
\begin{lemma}
\label{thm:induzierter_hom_lemma}
  Es ist
  \begin{align*}
    \im {\del'_k}^* & \subseteq Z^k(C,G), & {\del'_k}^* (\im i_{k-1}^*) & \subseteq B^k(C,G)
  \end{align*}
  Deshalb induziert ${\del'_k}^*$ einen Homomorphismus
  \begin{equation*}
    h\colon \Hom(B_{k-1},G) / {\im i_{k-1}^*} \to Z^k(C,G) / {B^k(C,G)}
  \end{equation*}
  mit $h([\varphi]) = [{\del'_k}^*(\varphi)]$.
  \begin{equation*}
    \begin{tikzcd}
      \Hom(B_{k-1},G)   \arrow{d}{\pi_{k-1}}
                        \arrow{r}{{\del'_k}^*}  & Z^k(C,G) \arrow{d}{\pi^k}\\
      \coker i_{k-1}^*  \arrow{r}{h}            & H^k(C,G)
    \end{tikzcd}
  \end{equation*}
\end{lemma}
\begin{vorbereitung}
  Weil schließlich~\eqref{eqn:inklusion_projektion_sequenz} eine freie Auflösung von $H_{k-1}(C)$ ist, gibt es einen (natürlichen) Isomorphismus
  \begin{equation*}
    \Phi \colon \Ext(H_{k-1}(C),G) \xrightarrow{\cong} \coker i_{k-1}^*.
  \end{equation*}
  Fassen wir $\Phi$ und $h$ zusammen, so erhält man einen (natürlichen) Homomorphismus $\rho := h \circ \Phi$,
  \begin{equation*}
    \rho \colon \Ext(H_{k-1}(C),G) \to H^k(C,G).
  \end{equation*}
\end{vorbereitung}
Es gilt nun
\begin{satz}[Universelles Koeffiziententheorem]
  Sei $(C,\del)$ ein freier Kettenkomplex, $G$ abelsche Gruppe und für jedes $k \in \Z$ seien $\rho$ und $\kappa$ die Homomorphismen von oben.
  Dann ist die folgende Sequenz (natürlich) exakt und spaltet:
  \begin{equation*}
    \begin{tikzcd}
      0 \arrow{r}{} & \Ext(H_{k-1}(C),G)  \arrow{r}{\rho} & H^k(C,G)  \arrow{r}{\kappa} & \Hom(H_k(C),G)  \arrow{r}{} & 0.
    \end{tikzcd}
  \end{equation*}
\end{satz}
\begin{proof}[Beweis zu Lemma~\ref{thm:induzierter_hom_lemma}]
  \begin{enumerate}
    \item 
      $\im {\del'_k}^* \subseteq Z^k(C,G)$:
      Für $\varphi \in \Hom(B_{k-1},G)$ ist
      \begin{equation*}
        \delta^k({\del'_k}^* \varphi) = \delta^k(\varphi \circ \del'_k) = \varphi \circ \underbrace{\del'_k \circ \del_{k+1}}_{=0} = 0.
      \end{equation*}
    \item
      ${\del'_k}^* (\im i_{k-1}^*) \subseteq B^k(C,G)$:
        Sei $\varphi \in \im i_{k-1}^* \subseteq \Hom(B_{k-1},G)$, also $\varphi = i_{k-1}^* \varphi'$ für ein $\varphi' \in \Hom(Z_{k-1},G)$ (d.h.\ $\varphi'\colon Z_{k-1} \to G$ ist Fortsetzung von $\varphi$).
        Weil aber~\eqref{eqn:inklusionssequenz} spaltet (mit $k-1$ statt $k$), existiert sogar Fortsetzung $\psi$ von $\varphi'$ auf ganz $C_{k-1}$,
        \begin{equation*}
          \psi := \varphi' \circ l_{k-1},
        \end{equation*}
        wobei $l_{k-1} \circ j_{k-1} = \id_{Z_{k-1}}$ ist,
        \begin{equation*}
          \psi \circ j_{k-1} \circ i_{k-1} = \varphi' \circ \underbrace{l_{k-1} \circ j_{k-1}}_{=\id} {}\circ i_{k-1} = \varphi
        \end{equation*}
        Es folgt:
        \begin{align*}
          \delta^{k-1} (\psi)
            & = \psi \circ \del_k\\
            & = \psi \circ j_{k-1} \circ i_{k-1} \circ \del'_k\\
            & = \varphi \circ \del'_k\\
            & = {\del'_k}^* (\varphi) \in B^k(C,G)
        \end{align*}
  \end{enumerate}
\end{proof}


%%%%%%%%%%%%%%%%%%%%%%%%%%%%%%%%%%%%
%%%%%%%% Beginn WS 17/18 %%%%%%%%%%%
%%%%%%%%%%%%%%%%%%%%%%%%%%%%%%%%%%%%


\begin{vorbereitung}
  \begin{enumerate}
    \item
      Seien $i_k \colon B_k \hookrightarrow Z_k$, $j_k \colon Z_k \hookrightarrow C_k$ die Inklusionen und sei $\del_k'\colon C_k \to B_{k-1}$ gegeben durch
      \begin{equation*}
        \del_k'(c) \coloneqq \del_k(c).
      \end{equation*}
      Dann gilt $\del_k = j_k \circ i_k \circ \del_k'$.
      Betrachte nun folgende exakte Sequenz:
      \begin{equation}
        \begin{tikzcd}
          \tag{$\star$}
          \label{seq:kes_auf_kette}
          0 \arrow{r}{} & Z_k \arrow{r}{j_k}
          & C_k \arrow[bend left]{l}{l_k}
          \arrow{r}{\del_k'}
          & B_{k-1} \arrow{r}{}
          & 0
        \end{tikzcd}
      \end{equation}
      Weil mit $C_{k-1}$ auch $B_{k-1}$ frei ist, spaltet~\eqref{seq:kes_auf_kette}.
      Man erhält also ein Linksinverses $l_k \colon C_k \to Z_k$ mit $l_k \circ j_k = \id_{Z_k}$.

      Es ist damit auch exakt:
      \begin{equation*}
        \begin{tikzcd}
          0 \arrow{r}{}
          & \Hom(B_{k-1},G) \arrow{r}{\del_k'^*}
          & \Hom(C_k,G) \arrow{r}{j_k^*}
          & \Hom(Z_k,G) \arrow{r}{}
          & 0
        \end{tikzcd}
      \end{equation*}
    \item
      Betrachte außderdem die kurze exakte Sequenz
      \begin{equation}
        \tag{$\star\star$}
        \label{seq:kes_mit_homologie}
        \begin{tikzcd}
          0 \arrow{r}{}
          & B_{k-1} \arrow{r}{i_{k-1}}
          & Z_{k-1} \arrow{r}{p_{k-1}}
          & H_{k-1}(C) \arrow{r}{}
          & 0
        \end{tikzcd}
      \end{equation}
      Es ist dann auch exakt:
      \begin{equation*}
        \begin{tikzcd}
          0 \arrow{r}{}
          & \Hom(H_{k-1}(C),G) \arrow{r}{p_{k-1}^*}
          & \Hom(Z_{k-1},G) \arrow{r}{i_{k-1}^*}
          & \Hom(B_{k-1},G).
        \end{tikzcd}
      \end{equation*}
  \end{enumerate}
\end{vorbereitung}

\begin{lemma}
  Es ist
  \begin{itemize}
    \item
      $\im(\del_k'^*) \subseteq Z^k(C,G)$,
    \item
      $\del_{k}'^*(\im(i_{k-1}^*)) \subseteq B^k(C,G)$.
  \end{itemize}
  Daher induziert $\del_k'^*\colon \Hom(B_{k-1},G) \to \Hom(C_k,G)$ nach dem Homomorphiesatz (genau) einen Homomorphismus
  \begin{equation*}
    h \colon \Hom(B_{k-1},G) / \im(i_{k-1}^*) = \coker(i_{k-1}^*) \to Z^k(C,G) / B^k(C,G) = H^k(C,G)
  \end{equation*}
  mit $h([\varphi]) = [\del_k'^*(\varphi)]$.
  %\label{}
  \begin{equation*}
    \begin{tikzcd}
      \Hom(B_{k-1},G) \arrow{r}{\del_k'^*} \arrow{d}{\pi_{k-1}}
      & Z^k(C,G) \subseteq \Hom(C_k,G) = C^k \arrow{d}{\pi_k} \\
      \Hom(B_{k-1},G) / \im(i_{k-1}^*) = \coker(i_{k-1}^*) \arrow[dashed]{r}{h}
      & H^k(C,G)
    \end{tikzcd}
  \end{equation*}
\end{lemma}
\begin{proof}
  \begin{enumerate}
    \item
      Für $\varphi \in \Hom(B_{k-1},G)$ ist $\delta^k \circ \del_{k}'^*(\varphi) = \delta^k(\varphi \circ \del_k') = \varphi \circ \underbrace{\del_k' \circ \del_{k+1}}_{= 0} = 0$.
    \item
      Sei $\varphi \in \im(i_{k-1}^*) \subseteq  \Hom(B_{k-1},G)$, also: $\varphi = i_{k-1}^*(\varphi')$ mit $\psi \in \Hom(Z_{k-1},G)$, d.h. $\varphi = \varphi' \circ i_{k-1}$.
      Weil aber~\eqref{seq:kes_auf_kette} spaltet, (mit $k-1$ statt $k$), existiert sogar eine Fortsetzung von $\varphi'$ auf ganz $C_{k-1}$:
      \begin{equation*}
        \psi \coloneqq \varphi' \circ l_{k-1} \implies \psi \circ j_{k-1} = \varphi' \circ {} \underbrace{l_{k-1} \circ j_{k-1}}_{= \id} = \varphi'.
      \end{equation*}
      Es ist dann $\delta^k(\psi) = \del_k^* (\psi) = \underbrace{\psi \circ j_{k-1} \circ i_{k-1}}_{\varphi} {}\circ \del_k' = \varphi \circ \del_{k}' = \del_{k}'^*(\varphi) \implies \del_{k}'^* \in B^k(C,G)$.
  \end{enumerate}
\end{proof}
\begin{vorbereitung}
  Weil schließlich~\eqref{seq:kes_mit_homologie} eine freie Auflösung von $H_{k-1}(C)$ ist, haben wir einen (natürlichen) Isomorphismus
  \begin{equation*}
    \Phi\colon \Ext(H_{k-1}(C),G) \to \coker(i_{k-1}^*).
  \end{equation*}
  Schalten wir diesen vor $h \colon \coker(i_{k-1}^*) \to H^K(C,G)$, so erhalten wir einen (natürlichen) Homomorphismus
  \begin{equation*}
    \rho \coloneqq h \circ \Phi \colon \Ext(H_{k-1}(C),G) \to H^k(C,G).
  \end{equation*}
\end{vorbereitung}
Es gilt nun
\begin{satz}[Universelles Koeffiziententheorem]
  Sei $(C,\del)$ ein freier Kettenkomplex, $G$ eine abelsche Gruppe und $\rho$ und $\kappa$ wie beschrieben.
  Dann ist die folgende Sequenz exakt und spaltet:
  \begin{equation*}
    \label{seq:koeffiziententheorem}
    \tag{$\ast$}
    \begin{tikzcd}
      0 \arrow{r}{}
      & \Ext(H_{k-1}(C),G) \arrow{r}{\rho}
      & H^K(C,G) \arrow{r}{\kappa}
      & \Hom(H_k(C), G) \arrow{r}{} \arrow[bend left]{l}{r}
      & 0
    \end{tikzcd}
  \end{equation*}
\end{satz}
\begin{proof}
  \begin{enumerate}[(i)]
    \item
      $\rho$ ist injektiv:
      Da $\Phi$ Isomorphismus ist, muss man zeigen, dass $h$ injektiv ist.
      Sei dazu $\varphi \in \Hom(B_{k-1},G)$ mit $h([\varphi]) = 0$, d.h.:
      \begin{equation*}
        0 \equiv \del_{k}'^*(\varphi) \equiv \varphi \circ \del_k' \mod B^k
      \end{equation*}
      Also gibt es ein $\psi \colon C_{k-1} \to G$ mit
      \begin{equation*}
        \varphi \circ \del_k' = \delta^{k-1}(\psi) = \psi \circ \del_k = \psi \circ j_{k-1} \circ i_{k-1} \circ \del_k'.
      \end{equation*}
      Da $\del_k'$ surjektiv ist, folgt:
      \begin{equation*}
        \varphi = \psi \circ j_{k-1} \circ i_{k-1} = i_{k-1}^*(\psi \circ j_{k-1}) \in \im(i_{k-1}^*)
      \end{equation*}
      (also fortsetzbar auf $Z_{k-1}$), d.h.: $[\varphi] = 0$ in $\coker(i_{k-1}^*)$, somit ist $h$ injektiv.
    \item
      $\im \rho = \ker \kappa$:
      Sei dafür $\varphi \in \Hom(B_{k-1},G)$ und $\psi \coloneqq \del_{k}'^* (\varphi) = \varphi \circ \del_k'$, also:
      \begin{equation*}
        [\psi] = h([\varphi]).
      \end{equation*}
      \emph{Zeige}: $\kappa([\psi]) = 0$ ($\implies \kappa \circ h = 0)$.
      Sei dazu $z \in Z_k$ beliebig.
      Dann ist:
      \begin{equation*}
        \kappa \circ h ([\varphi])([z]) = \kappa([\psi])([z]) = \langle \psi, z \rangle = \langle \varphi \circ \del_k', z \rangle = \varphi(\underbrace{\del_k(z)}_{= 0}) = 0
      \end{equation*}
      Daraus folgt $\im \rho \subseteq \ker \kappa$.
    \item
      $\ker \kappa \subseteq \im \rho$:
      Sei dazu $\psi \in Z^k$, sodass $\langle \psi, z \rangle = 0$ für alle $z \in Z_k$, also $[\psi] \in \ker \kappa$.
      \emph{Behauptung}: Es existiert Homomorphismus $\varphi \colon B_{k-1} \to G$  mit $\psi = \varphi \circ \del_k'$, denn dann ist $[\psi] = h([\varphi])$ (und damit $[\psi] \in \im h = \im \rho$).
      Wegen $\langle \psi, z \rangle = 0$ für alle $z \in Z_k$ ist
      \begin{equation*}
        j_k^*(\psi) = \psi \circ j_k = 0.
      \end{equation*}
      Weil die duale Sequenz von~\eqref{seq:kes_auf_kette} bei $\Hom(C_{k-1},G)$ exakt ist, existiert ein $\varphi \in \Hom(B_{k-1},G)$ mit $\del_{k}'^*(\varphi) = \psi$.
      Daher ist
      \begin{equation*}
        [\psi] = [\varphi \circ \del_k'] = h([\varphi]) \in \im h.
      \end{equation*}
    \item
      $\kappa$ ist surjektiv und hat ein Rechtsinverses:
      Sei $\lambda \colon H_k(C) \to G$ beliebig und (wie früher) $l_k \colon C_k \to Z_k$ linksinvers zu $j_k$, $l_k \circ j_k = \id$.
      Setze dann $\varphi \colon C_k \to G, \varphi \coloneqq \lambda \circ p_k \circ l_k$.
      Dann ist wegen $\im (l_k \circ \del_{k+1}) = \im \del_{k-1} = B_k$:
      \begin{equation*}
        \delta^k(\varphi) = \varphi \circ \del_{k+1} = \lambda \circ \underbrace{p_k \circ \underbrace{l_k \circ \del_{k+1}}_{\to B_k}}_{= 0},
      \end{equation*}
      also $\varphi \in Z^k(C,G)$.
      Weiter gilt für alle $z \in Z_k$:
      \begin{equation*}
        \kappa([\varphi])([z)] = \langle \varphi, j_k(z) \rangle = \varphi \circ j_k(z) = \lambda \circ p_k \circ \underbrace{l_k \circ j_k}_{= 0}(z)
        = \lambda([z])
      \end{equation*}
      und somit $\kappa([\varphi]) = \lambda$, also $\kappa$ surjektiv.
      Außerdem ist die Zuordnung
      \begin{align*}
        r \colon \Hom(H_k(C),G) & \to H^k(C,G) \\
        \lambda & \mapsto [\lambda \circ p_k \circ l_k]
      \end{align*}
      homomorph und damit $r$ rechtsinvers zu $\kappa$.
  \end{enumerate}
\end{proof}



\begin{kommentar}
  \begin{enumerate}
    \item
      Weil Sequenz~\eqref{seq:kes_auf_kette} spaltet, erhält man insbesondere einen Isomorphismus
      \begin{equation*}
        \label{eqn:iso_cohom-ext_plus_hom}
        \tag{$\ast\ast$}
        H^k(C,G) \cong \Ext(H_{k-1}(C),G) \oplus \Hom(H_k(C),G)
      \end{equation*}
    \item
      Die Sequenz~\eqref{seq:koeffiziententheorem} is in dem Sinne natürlich wie die Homomorphismen $\rho$ und $r$ natürliche Transformationen von $F_1 = \Ext(H_{k-1}(-),G)$, $F_2 = H^k(-,G)$und $F_3 = \Hom(H_k(-),G)$, d.h.:
      Ist $f \colon C \to C'$ Kettenabbildung zwischen freien Kettenkomplexen, so kommutiert:
      \begin{equation*}
        \begin{tikzcd}
          0 \arrow{r}{}
          & \Ext(H_{k-1}(C),G) \arrow{r}{\rho_C}
          & H^k(C,G) \arrow{r}{\kappa_C}
          & \Hom(H_k(C),G) \arrow{r}{}
          & 0 \\
          0 \arrow{r}{}
          & \Ext(H_{k-1}(C'),G) \arrow{r}{\rho_{C'}}
            \arrow{u}{f^*}
          & H^k(C',G) \arrow{r}{\kappa_{C'}}
            \arrow{u}{f^*}
          & \Hom(H_k(C'),G) \arrow{r}{}
            \arrow{u}{f^*}
          & 0
        \end{tikzcd}
      \end{equation*}
      wo $f^* = F(f)$ ist mit $F = F_1,F_2$ beziehungsweise $F_3$.
      Allerdingskann man den Isomorphismus in~\eqref{eqn:iso_cohom-ext_plus_hom} für jedes freie $C$ nicht so wählen, dass er natürlich ist.
      [Stö/Zie: S. 264/265]
      \todo[]{Referenz hinzufügen}
    \item
      Is die Homologie von $(C,\del)$ endlich erzeugt, so kann man die Cohomologie von $C$ mit Koeffizienten in $G$ berechnen, falls $G = \Z,\Q,\R$ oder $\mathbb C$.
      Ist nämlich
      \begin{equation*}
        H_k(C) \cong \Z^{b_k} \oplus \Tor(H_k(C)),
      \end{equation*}
      so ist wegen
      \begin{align*}
        \Hom(\Tor(H_k(C)),G) & = \triv,\\
      \Hom(\Z^b,G) & = G^k
      \end{align*}
      Andererseits ist
      \begin{equation*}
        \Ext(H_{k-1}(C),G) =
        \begin{cases}
          0 & \text{für $G = \Q,\R$ oder $\mathbb C$}\\
          \Tor(H_{k-1}(C))  & \text{für $G = \Z$}
        \end{cases}
      \end{equation*}
      dann $\Ext(\Z_n,\Z) = \Z_n$ und $\Tor(H_{k-1}(C)) = \Z^{n_1} \oplus \dotsb \oplus \Z^{n_r}$ mit $n_1, \dotsc n_r \in \N$ mit $n_1 | n_2 | \dotsb | n_r$.
      Für $G = \Z$ hat die Cohomologie
      \begin{equation*}
        H^*(C) \coloneqq H^*(C,G) \coloneqq \bigoplus_{k \in \Z} H^k(C,\Z)
      \end{equation*}
      die gleiche Information wie die Homologie $H_*(C) = \bigoplus_{k\in\Z}H_k(C)$, denn ist
      \begin{equation*}
        H^k(C) = \Z^{b_k} \oplus \underbrace{\Tor(H^k(C))}_{= \Z_{n_1^k} \oplus \dotsb \oplus \Z_{n_{r_k}^k}},
      \end{equation*}
      so ist die "`Cobettizahl"' $b_k$ also gleich der Bettizahl $\rg(H^k(C)) = \rg(H_k(C))$ und die "`Cotorsionskoeffizienten"' $n_1^k, \dotsc, n_{r_k}^k$ im Grad $k$ sind die Torsionskoeffizienten $m_{1}^{k-1}, \dotsc, m_{s_{k-1}}^{k-1}$ im Grad $k-1$, $\Tor(H^k(C)) \cong \Tor(H_{k-1}(C))$.
  \end{enumerate}
\end{kommentar}


\section{Homologie mit Koeffizienten}

\begin{motivation}
  \begin{enumerate}
    \item
      Homologie mit Koeffizienten (aus einer beliebigen abelschen  Gruppe $G$) wird aus Kettenkomplexen gebildet, in denen die Kettengruppen direkte Summen von Kopien von $G$ (nicht von $\Z$) sind, ihre Elemente also von der Form
      \begin{equation*}
        c = g_{1}\sigma_1 + \dotsb + g_r\sigma_r
      \end{equation*}
      mit $r \in \N_0$, $g_j \in G$ ($j = 1,\dotsc, r)$ und (in der singulären Theorie) singulären $k$-Simplexen $\sigma_j$ ($j = 1, \dotsc, r)$.
    \item
      So bekommt man beispielsweise für die Koeffizientengruppe $G = \Z_2$, dass für den Randoperator $\del$ des zellulären Kettenkomplexes von $\mathbb{P}^n(\R)$ mit Koeffizienten in $\Z_2$ gilt: $\del = 0$.
      Das führt dann zu
      \begin{equation*}
        H_k(\mathbb{P}^n(\R),\Z_2) =
        \begin{cases}
          \Z_2 & \text{für $0 \le k \le n$}\\
          0   & \text{für $k > n$}
        \end{cases}
      \end{equation*}
  \end{enumerate}
\end{motivation}

\begin{defn}
  Seien $A$ und $B$ abelsche Gruppen.
  Wir nennen ein Paar $(T,t)$ bestehend aus einer abelschen Gruppe $T$ und einer bilinearen Abbildung $t \colon A \times B \to T$ ein \emph{Tensorprodukt von $A$ und $B$}, wenn folgende universelle Eigenschaft gilt:
  Ist $(C,s)$ ein Konkurrenzpaar ($C$ abelsche Gruppe, $s$ bilinear), so gibt es genau einen Homomorphismus $\Phi \colon T \to C$ mit $\Phi \circ t = s$.
  \begin{equation*}
    \begin{tikzcd}
      A\times B \arrow{r}{s}
          \arrow{d}{t}
      & C\\
      T \arrow[dashed,swap]{ur}{\Phi}
    \end{tikzcd}
  \end{equation*}
\end{defn}

\begin{kommentar}
  \begin{enumerate}
    \item
      Ein Tensorprodukt $(T,t)$ ist -- wenn es existiert -- im folgenden Sinne eindeutig bestimmt:
      Sind $(T_1,t_1)$ und $(T_2,t_2)$ zwei Tensorprodukte, so existiert ein (sogar eindeutiger) Isomorphismus $\Phi \colon T_1 \to T_2$ mit $\Phi \circ t_1 = t_2$ (Übung).
    \item
      Die Existenz kann man so sehen:
      Bilde zunächst die frei-abelsche Gruppe $(\F(A\times B),i)$ und betrachte dann die Untergruppe $R \subseteq \F(A \times B)$, die von allen Elementen
      \begin{gather*}
        (a_1 + a_2, b) - (a_1,b) - (a_2,b)\\
        (a,b_1 + b_2) - (a,b_1) - (a, b_2)
      \end{gather*}
    mit $a,a_1,a_2 \in A$ sowie $b,b_1,b_2 \in B$ erzeugt wird.
    Ist $i \colon A \times B \hookrightarrow \F(A \times B)$ die natürliche Inklusion und $\pi \colon \F(A \times B) \to \F(A \times B) / R$ die kanonische Projektion, so setze
    \begin{align*}
      A \otimes B & \coloneqq \F(A \times B) / R, \\
      \otimes & \coloneqq \pi \circ i \colon A \times B \to A \otimes B.
    \end{align*}
    Es ist dann $A \otimes B$ eine abelsche Gruppe und $\otimes$ ist bilinear, denn mit $\otimes(a,b) \eqqcolon a \otimes b$:
    \begin{equation*}
      (a_1 + a_2) \otimes b - a_1 \otimes b - a_2 \otimes b = \pi(\underbrace{(a_1 + a_2, b) - (a_1,b) - (a_2,b)}_{\in R}) = 0
    \end{equation*}
    und genau so:
    \begin{equation*}
      a \otimes (b_1+b_2) = a \otimes b_1 + a \otimes b_2
    \end{equation*}
    für alle $a,a_1,a_2 \in A$ und $b,b_1,b_2 \in B$.
    Es ist auch
    \begin{equation*}
      n (a \otimes b) = (na) \otimes b = a \otimes (nb)
    \end{equation*}
    für alle $n \in \N$ (Übung).

    Zur universellen Eigenschaft:
    Ist $s \colon A \times B \to C$ Konkurrent (zu $\otimes \colon A \times B \to A \otimes B$), so existiert zunächst nach der universellen Eigenschaft von $(\F(A \times B), i)$ ein eindeutig bestimmter Homomorphismus $\tilde \Phi \colon \F(A \times B) \to C$ mit $\tilde \Phi \circ i = s$.
    \begin{equation*}
      \begin{tikzcd}
        A \times B \arrow{r}{s}
          \arrow{d}{i}
        & C\\
        \F(A \times B) \arrow[dashed,swap]{ur}{\tilde \Phi}
      \end{tikzcd}
    \end{equation*}
    Da $\tilde \Phi | _R = 0$ ist, weil $s$ bilinear ist, existiert nach der universellen Eigenschaft des Quotienten $(A \otimes B, \pi)$ genau ein Homomorphismus $\Phi \colon A \otimes B \to C$ mit $\Phi \circ \pi = \tilde \Phi$.
    \begin{equation*}
      \begin{tikzcd}
        \F(A \times B) \arrow{r}{\tilde \Phi}
          \arrow{d}{\pi}
        & C\\
        A \otimes B \arrow[dashed,swap]{ur}{\Phi}
      \end{tikzcd}
    \end{equation*}
    Es kommutiert also auch das große Dreieck im Diagramm, $\Phi \circ \otimes = s$
    \begin{equation*}
      \begin{tikzcd}
        A \times B \arrow{r}{s}
          \arrow{d}{i}
          \arrow[bend right,swap,out=305,in=235,looseness=1.4]{dd}{\oplus}
        & C\\
        \F(A \times B) \arrow[swap]{ur}{\tilde \Phi}
          \arrow{d}{\pi}
        \\
        A \otimes B \arrow[bend right,swap]{uur}{\Phi}
      \end{tikzcd}
    \end{equation*}
    $\Phi$ is auch eindeutig, denn kommutiert das große Dreieck, so auch das untere kleine.
  \end{enumerate}
\end{kommentar}


\end{document}
