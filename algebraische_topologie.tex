
\documentclass[parskip=half,a4paper]{scrartcl}

\usepackage[utf8x]{inputenc}
\usepackage[T1]{fontenc}
\usepackage{lmodern}
\usepackage[ngerman]{babel}

\usepackage{amsmath,amssymb,amsthm}
\usepackage{mathtools}

\usepackage{microtype}

\usepackage{enumerate}

\usepackage[]{hyperref}

\renewcommand{\labelenumi}{(\alph{enumi})}
\renewcommand{\labelenumii}{(\roman{enumii})}

\usepackage{tikz}
%\usetikzlibrary{cd}
\usepackage{tikz-cd}

\tikzset{%
  >=stealth
}
%\tikzcdset{%
  %arrow style=tikz,
  %diagrams={>={Straight Barb}}
  %%diagrams={>=stealth}
%}

%\listfiles
\usepackage{todonotes}

\newtheoremstyle{default}
{3pt}% pace above
{3pt}% Space below
{}% Body font
{}% Indent amount
{\bfseries}% Theorem head font
{.}% Punctuation after theorem head
{.5em}% Space after theorem head
{\thmnumber{(#2)} \thmname{#1}\thmnote{ (#3)}}% Theorem head spec (can be left empty, meaning ‘normal’)
\theoremstyle{default}
\newtheorem{prop}{Proposition}[section]
\newtheorem{lemma}[prop]{Lemma}
\newtheorem{erinnerung}[prop]{Erinnerung}
\newtheorem{kommentar}[prop]{Kommentar}
\newtheorem{korollar}[prop]{Korollar}
\newtheorem{beispiel}[prop]{Beispiel}
\newtheorem{zusatz}[prop]{Zusatz}
\newtheorem{defn}[prop]{Definition}
\newtheorem{bemerkung}[prop]{Bemerkung}
\newtheorem{proposition}[prop]{Proposition}
\newtheorem{vorbereitung}[prop]{Vorbereitung}
\newtheorem{motivation}[prop]{Motivation}
\newtheorem{satz}[prop]{Satz}

\title{Algebraische Topologie II ff}
\author{Ingo Skupin}
%\renewcommand*{\sectionformat}{\textrm\S \thesection\enskip}


\newcommand{\Hom}{\mathrm{Hom}}
\newcommand{\Ext}{\mathrm{Ext}}
\newcommand{\im}{\operatorname{im}}
%\newcommand{\ker}{\operatorname{ker}}
\newcommand{\coker}{\operatorname{coker}}
\newcommand{\id}{\operatorname{id}}
\newcommand{\triv}{(0)}
\newcommand{\del}{\partial}
\newcommand{\Z}{\ensuremath{\mathbb{Z}}}
\newcommand{\N}{\ensuremath{\mathbb{N}}}
\newcommand{\F}{\ensuremath{\mathbb{F}}}
\newcommand{\R}{\ensuremath{\mathbb{R}}}
\newcommand{\Q}{\ensuremath{\mathbb{Q}}}
\newcommand{\CC}{\ensuremath{\mathbb{C}}}
\newcommand{\C}{\ensuremath{\mathcal{C}}}
\newcommand{\Ab}{\typesetConcreteCat{Ab}}
\newcommand{\GAb}{\typesetConcreteCat{GAb}}
\newcommand{\KK}{\typesetConcreteCat{KK}}
\newcommand{\CoKK}{\typesetConcreteCat{Co\kern1.5pt\text{\textbf{-}}KK}}
\newcommand{\typesetConcreteCat}[1]{\ensuremath{\mathbf{#1}}}
% https://tex.stackexchange.com/questions/224805/i-want-a-really-small-underbrace
\makeatletter
\def\smallunderbrace#1{\mathop{\vtop{\m@th\ialign{##\crcr
   $\hfil\displaystyle{#1}\hfil$\crcr
   \noalign{\kern3\p@\nointerlineskip}%
   \scriptsize\upbracefill\crcr\noalign{\kern3\p@}}}}\limits}
\makeatother


\begin{document}
\thispagestyle{empty}
\maketitle

\clearpage

%\renewcommand*{\thesection}{§\arabic{section}}

\setcounter{section}{8}
\section{Homologische Algebra}

\setcounter{prop}{3}
\begin{prop}
  Sei $G$ eine abelsche Gruppe und
  \begin{equation*}
    \begin{tikzcd}
      A \arrow{r}{f}  & B
        \arrow{r}{g}  & C
        \arrow{r}{}   & 0
    \end{tikzcd}
  \end{equation*}
  eine exakte Sequenz abelscher Gruppen. Dann is auch die induzierte Sequenz
  \begin{equation*}
    \begin{tikzcd}
      \Hom(A,G) & \Hom(B,G) \arrow{l}{f^*}
                & \Hom(C,G) \arrow{l}{g^*}
                & 0         \arrow{l}{}
    \end{tikzcd}
  \end{equation*}
  exakt. $\Hom(-,G)$ ist \emph{links-exakt}.
\end{prop}
\begin{proof}
  \begin{enumerate}[(i)]
    \item 
      Exaktheit bei $\Hom(C,G)$.
      Zeige $g^*$ ist injektiv.
      Sei $\varphi \in \Hom(C,G)$ mit $g^*(\varphi) = \varphi \circ g  = 0$.
      \begin{align*}
        \begin{tikzcd}[ampersand replacement=\&]
          B \arrow{rd}{0} \arrow[twoheadrightarrow]{r}{g}  \& C \arrow{d}{\varphi} \\
                                        \& G
        \end{tikzcd}
        & \qquad \implies \varphi = 0
      \end{align*}
      %\begin{equation*}
        %\begin{tikzcd}
          %B \arrow{rd}{0} \arrow[twoheadrightarrow]{r}{g}  & C \arrow{d}{\varphi} \\
                                        %& G
        %\end{tikzcd}
      %\end{equation*}
      %Dann folgt $\varphi = 0$.
    \item
      Exaktheit bei $\Hom(B,G)$:
      \begin{enumerate}[(a)]
        \item 
          $\im g^* \subseteq \ker f^*$, also $f^* \circ g^* = 0$.
          Aber $f^* \circ g^* = {(g \circ f)}^* = 0^* = 0$.
        \item
          $\ker f^* \subseteq \im g^*$:
          Sei $\varphi \colon B \to G \in \ker f^*$, $0 = f^*(\varphi) = \varphi \circ f$.
          \begin{equation*}
            \begin{tikzcd}
              B \arrow[twoheadrightarrow]{r}{g}
                \arrow[twoheadrightarrow]{rd}{\pi}
                \arrow{d}{\phi}
                & C \\
              G & B/{\ker g}  \arrow{u}{\overline g}
                              \arrow{l}{\overline \varphi}
            \end{tikzcd}
          \end{equation*}
          Dann ist $\ker g = \im f \subseteq \ker \varphi$ und daraus folgt die eindeutige Existenz eines $\overline \varphi \colon B/{\ker g} \to G$ mit $\overline \varphi \circ \pi = \varphi$.

          Ebenso induziert $g$ einen Morphismus $\overline g\colon B/{\ker g} \to C$ mit $\overline g \circ \pi = g$.
          Außerdem ist $\overline g$ injektiv und surjektiv, also ein Isomorphismus und somit
          \begin{equation*}
            \varphi = \overline \varphi \circ \pi = \overline \varphi \circ {\overline g}^{-1} \circ g = g^* (\overline \varphi \circ {\overline g}^{-1}).
            \end{equation*}
      \end{enumerate}
  \end{enumerate}
\end{proof}
\begin{kommentar}
  Man sagt, dass der kontravariante Funktor $\Hom(-,G) =: F$ links-exakt ist.
  Beachte, dass $F$ allerdings i.A.\ nicht exakte Sequenzen
  \begin{equation*}
    \begin{tikzcd}
      0 \arrow{r}{}   & A
        \arrow{r}{f}  & B
        \arrow{r}{g}  & C
        \arrow{r}{}   & 0
    \end{tikzcd}
  \end{equation*}
  in exakte Sequenzen überführt.
  \begin{equation*}
    \begin{tikzcd}
      0         & \Hom(A,G) \arrow{l}{}
                & \Hom(B,G) \arrow{l}{f^*}
                & \Hom(C,G) \arrow{l}{g^*}
                & 0         \arrow{l}{}
    \end{tikzcd}
  \end{equation*}
  \begin{tikz}[overlay]
    \draw[dashed] (3.1,.65) rectangle +(2.2,1);
  \end{tikz}
  ist also i.A.\ nicht exakt.
\end{kommentar}
\begin{erinnerung}
  eine exakte Sequenz abelscher Gruppen
  \begin{equation*}
    \begin{tikzcd}
      0 \arrow{r}{}
        & A \arrow{r}{f}
        & B \arrow[bend left]{r}{\theta}
        & C \arrow[bend left]{l}{r}
            \arrow[loop above]{r}{id_C}
            \arrow{r}{}
        & 0
    \end{tikzcd}
  \end{equation*}
  mit $g \circ r = id_C$ \emph{spaltet}.
  Äquivalent:
  \begin{equation*}
    \begin{tikzcd}
      0 \arrow{r}{}
        & A \arrow[bend left]{r}{f}
            \arrow[loop above]{r}{id_A}
        & B \arrow[bend left]{l}{l}
            \arrow{r}{g}
        & C \arrow{r}{}
        & 0
    \end{tikzcd}
  \end{equation*}
  mit $l \circ f = id_A$.

  In diesem Fall gilt $B \cong A \oplus C$.
\end{erinnerung}
\begin{zusatz}
  Ist
  \begin{equation*}
    \begin{tikzcd}
      0 \arrow{r}{}   & A
        \arrow{r}{f}  & B
        \arrow{r}{g}  & C
        \arrow{r}{}   & 0
    \end{tikzcd}
  \end{equation*}
  exakt und spaltet, so ist auch
  \begin{equation*}
    \begin{tikzcd}
      0         & \Hom(A,G) \arrow{l}{}
                & \Hom(B,G) \arrow{l}{f^*}
                & \Hom(C,G) \arrow{l}{g^*}
                & 0         \arrow{l}{}
    \end{tikzcd}
    \tag{$*$}\label{eqn:hom_spaltung}
  \end{equation*}
  exakt und spaltet.
\end{zusatz}
\begin{proof}
  ist $l \colon B \to A$ linksinvers zu $f$, $l \circ f = id_A$, so ist $id_{\Hom(A,G)} = id_{A}^* = {(l \circ f)}^* = f^* \circ l^*$, also ist $f^*$ surjektiv.
  Außerdem ist nun $l^*$ offenbar rechtsinvers zu $f^*$, also eine Spaltung von~\eqref{eqn:hom_spaltung}.
\end{proof}
\begin{defn}
  Sei $A$ eine abelsche Gruppe.
  Dann heißt eine kure exakte Sequenz
  \begin{equation*}
    \begin{tikzcd}
      0 \arrow{r}{}
        & R \arrow{r}{\alpha}
        & F \arrow{r}{\beta}
        & F \arrow{r}{}
        & 0
    \end{tikzcd}
  \end{equation*}
  eine \emph{freie Auflösung}, wenn $F$ eine frei abelsche Gruppe ist.
\end{defn}
\begin{kommentar}
  Als Untergruppe (vie $\alpha$) von $F$ ist $R$ selbst eine frei abelsche Gruppe.
  Ist ${(e_i)}_{i \in I}$ eine Basis von $F$, so ist $\varepsilon = {(\beta(e_i))}_{i \in I}$ ein Erzeugendensystem von $A$.
  Und ist ${(r_j)}_{j \in J}$ eine Basis von $R$, so erzeugt ${(\alpha(r_j))}_{j \in J}$ die Relationen von $\varepsilon$ (\emph{Relationen auf $\varepsilon$}: $f \in F$ mit $\beta(f) = 0$).
\end{kommentar}
\begin{beispiel}
  \begin{enumerate}
    \item 
      ist $A$ selbst frei, s okann man $F = A$ und $\beta = id_A$ wählen (dann $R = \triv$).
    \item
      Ist $A = \Z_4$, so ist
      \begin{equation*}
        \begin{tikzcd}
          0 \arrow{r}{}
            & \Z \arrow{r}{\cdot 2}
            & \Z \arrow{r}{\pi}
            & \Z_2 \arrow {r}{}
            & 0
        \end{tikzcd}
      \end{equation*}
      eine freie Auflösung.
    \item
      Ist $A$ beliebig, so betrachte $A$ als menge und setze $F = \F(A)$ und $\pi\colon F \to A$ der Homorphismus, der auf der Basis ${(i(a))}_{a \in A}$ durch $\pi(i(a)) = a$ gegeben ist.
      Natürlich ist dann $\pi(2 \cdot a) = \pi(1\cdot (2a)) = 2a$ und $\pi(0_A) = \pi(0_{\F(A)}) = 0_A$.
      Ist $R = \ker \pi$ und $j\colon R \hookrightarrow F$ die Inklusion, so ist
      \begin{equation*}
        \begin{tikzcd}
          0 \arrow{r}{}
            & R \arrow{r}{j} 
            & F \arrow{r}{\pi}
            & A \arrow{r}{}
            & 0
        \end{tikzcd}
      \end{equation*}
      offenbar exakt (weil $\pi$ surjektiv ist).
      Das ist die \emph{Standardauflösung} $S(A)$ von $A$:
      \begin{equation*}
        \begin{tikzcd}
          S(A):
          0 \arrow{r}{}
            & R \arrow{r}{j} 
            & F \arrow{r}{\pi}
            & A \arrow{r}{}
            & 0
        \end{tikzcd}
      \end{equation*}
  \end{enumerate}
\end{beispiel}


Ist
\begin{equation*}
  \begin{tikzcd}
    0 \arrow{r}{}
      & A \arrow{r}{f}
      & B \arrow{r}{g}
      & C \arrow{r}{}
      & 0
  \end{tikzcd}
\end{equation*}
exakt, so ist
\begin{equation*}
  \begin{tikzcd}[arrows=leftarrow]
    ? \arrow{r}{}
      & \Hom(A,G) \arrow{r}{f^*}
      & \Hom(B,G) \arrow{r}{g^*}
      & \Hom(C,G) \arrow{r}{}
      & 0
  \end{tikzcd}
  \tag{$*$}\label{eqn:kes_unvollstaendig}
\end{equation*}
ist exakt, aber $f^*$ i.A.\ nicht surjektiv.
Naheliegend könnte man~\eqref{eqn:kes_unvollstaendig} mit
\begin{equation*}
  \coker f^* := \Hom(A,G) / {\im f^*}
\end{equation*}
und
\begin{equation*}
  \nu\colon \Hom/A,G) \to \coker f^*
\end{equation*}
fortsetzen, was aber so aussieht, dass es von zu vielen Wahlen abhängt.
\begin{defn}
  Seien $A$ und $G$ abelsche Gruppen und $S(A)$ die Standardauflösung von $A$.
  Dann nennt man
  \begin{align*}
    \Ext(A,G) &:= \coker i^* = \Hom(R,g) / {\im i^*}, \\
    i^* & \colon \Hom(F,G) \to \Hom/R,G)
  \end{align*}
  das \emph{Extensionsprodukt} (kurz: Ext-Produkt) \emph{von $A$ und $G$}.
\end{defn}
\begin{kommentar}
  Für die Standardauflösung $S(A)\colon \begin{tikzcd} 0 \arrow{r}{} &R\arrow{r}{i} &F\arrow{r}{\pi} &A\arrow{r}{}&0\end{tikzcd}$ von $A$ wird dann also mit der kanonischen Projektion $\nu\colon \Hom(R,G) \to \Ext(A,G)$ die folgende Sequenz exakt:
  \begin{equation*}
    \begin{tikzcd}
      0 \arrow{r}{}
        & \Hom(C,G) \arrow{r}{g^*}
        & \Hom(B,G) \arrow{r}{f^*}
        & \Hom(A,G) \arrow{r}{\nu}
        & \Ext(A,G) \arrow{r}{}
        & 0
    \end{tikzcd}
  \end{equation*}
\end{kommentar}
\emph{Frage}: Wie ist das mit anderen freien Auflösungen?
\begin{lemma}
  Seien $A$ und $A'$ abelsche Gruppen und
  \begin{equation*}
    \begin{tikzcd}[row sep=tiny]
      S\colon 0  \arrow{r}{}  & R \arrow{r}{f}  & F \arrow{r}{g}  & A \arrow{r}{} & 0 \\
      S'\colon 0 \arrow{r}{}  & R \arrow{r}{f'} & F \arrow{r}{g'} & A \arrow{r}{} & 0
    \end{tikzcd}
  \end{equation*}
  freie Auflösungen sowie $h\colon A \to A'$ ein Homomorphismus.
  \begin{equation*}
    \begin{tikzcd}
      0 \arrow{r}{} & R   \arrow{d}{\alpha}
                          \arrow{r}{f}  & F   \arrow{d}{\beta}
                                              \arrow{r}{g}  & A   \arrow{d}{h}
                                                                  \arrow{r}{} & 0 \\
      0 \arrow{r}{} & R'  \arrow{r}{f'} & F'  \arrow{r}{g'} & A'  \arrow{r}{} & 0 
    \end{tikzcd}
    \tag{$*$}\label{eqn:frei_aufl_hom}
  \end{equation*}
  \begin{enumerate}[(a)]
    \item 
      Dann existieren Homomorphismen $\alpha \colon R \to R'$ und $\beta \colon F \to F'$, die das Diagramm~\eqref{eqn:frei_aufl_hom} kommutieren lassen.
      \begin{equation*}
        \begin{tikzcd}
      0 \arrow{r}{} & R   \arrow[swap]{d}{\tilde\alpha - \alpha}
                          \arrow{r}{f}  & F   \arrow{d}{\tilde\beta - \beta}
                                              \arrow{dl}{\varphi}
                                              \arrow{r}{g}  & A   \arrow{d}{h-h = 0}
                                                                  \arrow{r}{} & 0 \\
      0 \arrow{r}{} & R'  \arrow[swap]{r}{f'} & F'  \arrow{r}{g'} & A'  \arrow{r}{} & 0 
        \end{tikzcd}
        \tag{$**$}\label{eqn:frei_aufl_homotopie}
      \end{equation*}
    \item
      Sind $\tilde\beta \colon F \to F'$ und $\tilde\alpha \colon R \to R'$ weitere Homomorphismen, die~\eqref{eqn:frei_aufl_hom} kommutieren lassen, so existiert ein $\varphi \colon F \to R'$ mit
      \begin{equation*}
        f' \circ \varphi = \tilde\beta - \beta, \qquad \varphi \circ f = \tilde\alpha - \alpha.
      \end{equation*}
  \end{enumerate}
\end{lemma}
\begin{proof}
  \begin{enumerate}[(a)]
    \item 
      Sei ${(e_i)}_{i \in I}$ eine Basis von $F$.
      Betrachte dann $h \circ g (e_i) \in A'$.
      Da $f'$ surjektiv ist, existiert $x_i \in F'$ mit $f'(x_i) = h \circ g(e_i)$.
      Definiere dann $\beta \colon F \to F'$ durch $\beta(e_i) := x_i$.
      Dann gilt für alle $i \in I$:
      \begin{equation*}
        g' \circ \beta(e_i) = g'(x_i) = h \circ g(e_i)
      \end{equation*}
      und somit $g' \circ \beta = h \circ g$.

      Für alle $x \in R$ ist
      \begin{align*}
        g' \circ (\beta \circ f) (x)
          & = (g' \circ \beta) \circ f (x) \\
          & = h \circ \underset{=0}{\underbrace{g \circ f(x)}}
      \end{align*}
      und somit $\im (\beta \circ f) \subseteq  \ker g' = \im f'$.
      Bezeichne mit $f'\colon R' \to \im f'$ auch den Homomorphismus $f'$, wenn man der Wertebereich auf $\im f'$ einschränkt.
      Dann ist (dieses) $f'$ ein Isomorphismus, da $f'$ injektiv und (nun auch) surjektiv ist.
      Setze dann
      \[
        \alpha \colon R \to R',\quad \alpha := {(f')}^{-1} \circ \beta \circ f.
      \]
      Dann ist offenbar
      \begin{equation*}
        f' \circ \alpha = \beta \circ f.
      \end{equation*}
    \item
      Für  alle $x \in F$ ist
      \begin{equation*}
        g' \circ (\tilde\beta - \beta) (x) = (g' \circ \tilde \beta - g' \circ \beta) (x) = h\circ g(x) - h \circ g(x) = 0
      \end{equation*}
      und somit $\im (\tilde\beta - \beta) \subseteq \ker g' = \im f'$.
      Setze daher $\varphi \colon F \to R'$ mit $\varphi := {(f')}^{-1} \circ (\tilde \beta - \beta)$, dann gilt $f' \circ \varphi = \tilde \beta - \beta$.
      Es ist aber auch:
      \begin{align*}
        f' \circ (\tilde\alpha - \alpha)
          & = f' \circ \tilde \alpha - f' \circ \alpha \\
          & = \tilde \beta \circ f - \beta \circ f \\
          & = (\tilde \beta - \beta) \circ f \\
          & = f' \circ \varphi \circ f.
      \end{align*}
      Da $f'$ injektiv ist, folgt $\tilde \alpha - \alpha = \varphi \circ f$.
  \end{enumerate}
\end{proof}
\begin{prop}
  \label{thm:hom_freier_aufloesungen}
  Seien $A$, $A'$ abelsche Gruppen, $S$ und $S'$ freie Auflösungen von $A$ und $A'$,
  \begin{equation*}
    \begin{tikzcd}
      S\colon\,0 \arrow{r}{} & R   \arrow{d}{f}
                              \arrow{r}{j}  & F   \arrow{d}{}
                                                  \arrow{r}{}   & A   \arrow{d}{h}
                                                                      \arrow{r}{} & 0, \\
      S'\colon 0 \arrow{r}{} & R'  \arrow{r}{j'} & F'  \arrow{r}{}   & A'  \arrow{r}{} & 0,
    \end{tikzcd}
  \end{equation*}
  und sei $h\colon A \to A'$ Homomorphismus.
  Dann gibt es genau einen Homomorphismus
  \begin{equation*}
    \Phi(h;S,S') \colon \coker {(j')}^* \to \coker j^*,
  \end{equation*}
  sodass gilt: Sind $g \colon F \to F'$ und $f \colon R \to R'$ Homomorphismen derart, dass $(f,g,h) \colon S \to SÄ$ Homomorphismus zwischen freien Auflösungen ist, d.h.~\eqref{eqn:frei_aufl_hom} kommutiert, so kommutiert auch (für jede abelsche Gruppe $G$):
  \begin{equation*}
    \begin{tikzcd}
      0 \arrow{r}{}   & \Hom(A,G)   \arrow{r}{}   & \Hom(F,G) \arrow{r}{j^*}        & \Hom(R,G) \arrow{r}{\nu}  & \coker j^* \arrow{r}{}      & 0 \\
      0 \arrow{r}{}   & \Hom(A',G)  \arrow{u}{h^*}
                                    \arrow{r}{}   & \Hom(F',G)\arrow{u}{g^*}
                                                              \arrow{r}{{(j')}^*}  & \Hom(R',G) \arrow{u}{f^*}
                                                                                                \arrow[swap]{ul}{\alpha^*}
                                                                                                \arrow{r}{\nu} & \coker {(j')}^* \arrow{u}{\Phi}
                                                                                                                              \arrow{r}{} & 0
    \end{tikzcd}
  \end{equation*}
\end{prop}
\begin{kommentar}
  Das Besondere ist nicht so sehr die Existenz und Eindeutigkeit von $\Phi$, denn es gibt nur einen Kandidaten
  \begin{equation*}
    \Phi([\varphi]) = [f^*(\varphi)]  \qquad \text{für } \varphi \in \Hom(R',G), 
  \end{equation*}
  sondern mehr die Unabhängigkeit von $f$ (und $g$).
\end{kommentar}
\begin{proof}[Beweis von Proposition~\ref{thm:hom_freier_aufloesungen}]
  Wegen $f^* \circ {(j')}^* = j^* \circ g^*$ ist
  \begin{equation*}
    f^* (\im {(j')} ^*) \subseteq \im j^*
  \end{equation*}
  und somit $\nu \circ f^* (\im {(j')}^* = 0$, also existiert eindeutiges $\Phi \colon \coker {(j')}^* \to \coker j^*$ mit $\Phi \circ \nu' = \nu \circ f^*$, also
  \begin{equation*}
    \Phi([\varphi]) = [f^*(\varphi)] \quad \text{für alle $\varphi \in \Hom(R',G)$}.
  \end{equation*}
  Ist $(\tilde f, \tilde g)$ eine weitere Wahl, die~\eqref{eqn:frei_aufl_hom} kommutieren lässt, und $\alpha \colon F \to R'$ mit $\alpha \circ j = \tilde f - f$ (und $j' \circ \alpha = \tilde g - g$), so ist $j^* \circ \alpha^* = {\tilde f}^* - f^*$ und somit ${\tilde f}^* (\varphi) - f^*(\varphi) \in \im j^*$.
  Macht daher $\tilde\Phi$~\eqref{eqn:frei_aufl_hom} kommutativ (mit $\tilde f$ statt $f$ und $\tilde g$ statt $g$), so ist
  \begin{equation*}
    \tilde\Phi ([\varphi]) = [f^*(\varphi)] = [{\tilde f}^*(\varphi)] = \Phi ([\varphi]) \quad \text{für alle $\varphi \in \Hom(R',G)$}.
  \end{equation*}
\end{proof}
\begin{korollar}
  Definiert man eine Kategorie \C\ dadurch, dass man als Objekte freie Auflösungen $S$ von abelschen Gruppen $A$ betrachtet und als Morphismen solche von $A$ nach $A'$,
  \begin{equation*}
    \begin{tikzcd}
      S\colon 0 \arrow{r}{}  & R \arrow{r}{j}  & F \arrow{r}{} & A \arrow{d}{h}
                                                              \arrow{r}{}   & 0 \\
      \phantom{S\colon}
      0 \arrow{r}{}  & R'\arrow{r}{j}  & F'\arrow{r}{} & A'\arrow{r}{}   & 0,
    \end{tikzcd}
  \end{equation*}
  so ist die Zuordnung (für gegebene $G$) $\Phi \colon \C \to \Ab$ auf Objekten $\Phi(S) := \coker j^*$ und auf Morphismen $h \mapsto \Phi(h;S,S')$, wie in der Proposition, funktoriell:
  \begin{enumerate}[(a)]
    \item 
      $\Phi(\id;S,S) = \id$
    \item
      $\Phi(h' \circ h;S,S'') = \Phi(h;S,S') \circ \Phi(h';S',S'')$ für alle $h,h',S,S',S''$ etc.
    \end{enumerate}
\end{korollar}
\begin{proof}
  \begin{enumerate}[(a)]
    \item 
      Ist $h = \id\colon S \to S$, so kann $f = \id$ und $g = \id$ gewählt werden und man erhält: $\Phi = \id$.
    \item
      Sind $(f,g)$ bzw.\ $(f',g')$ für $h \colon S \to S'$ bzw.\ $h'\colon S' \to S''$ gewählt, so kann man für $(f'',g'')$ die Wahlen $f'' = f' \circ f$ und $g'' = g' \circ g$ treffen, weil das Diagramm~\eqref{eqn:neu} dann als Zusammensetzung von~\eqref{eqn:} und~\eqref{eqqn:} kommutiert.
      \todo{Fill the gaps}
      Somit gilt
      \begin{align*}
        \Phi(h' \circ h; S,S'')
          & = [{(f' \circ f)}^* (\varphi)]\\
          & = [f^* \circ {(f')}^* (\varphi)]\\
          & = \Phi(h;S,S') \circ \Phi(h';S',S'').
      \end{align*}
  \end{enumerate}
\end{proof}



\begin{kommentar}
  \begin{enumerate}[(a)]
    \item
      Ist $h\colon A \to A'$ Isomorphismus, so ist
      \begin{equation*}
        \Phi(h;S,S')\colon \coker {(j')}^* \to \coker j^*
      \end{equation*}
      ebenfalls Isomorphismus, denn $\Phi(h^{-1};S',S) = {\Phi(h;S,S')}^{-1}$.
    \item
      Ist insbesondere $S(A)$ die Standardauflösung von $A$, $S$ eine beliebige freie Auflösung von $A$ und $h = \id_A$, so ist also das induzierte
      \begin{equation*}
        \Phi(\id;S,S(A))\colon \coker i^* = \Ext(A,G) \to \coker j^*
      \end{equation*}
      ein (kanonischer) Isomorphismus.
      Daher hängt der Isomorphietyp von $\Ext(A,G)$ nicht von der Wahl der freien Auflösung von $A$ ab.
  \end{enumerate}
\end{kommentar}
\begin{beispiel}
  \begin{enumerate}[(a)]
    \item
      ist $A$ frei-abelsch und $G$ beliebig, so ist
      \begin{equation*}
        \Ext(A,G) = \triv,
      \end{equation*}
      denn:
      \begin{equation*}
        \begin{tikzcd}
          S\colon 0 \arrow{r}{}  & 0 \arrow{r}{0}  & A \arrow{r}{\id}  & A \arrow{r}{} & 0
        \end{tikzcd}
      \end{equation*}
      ist freie Auflösung und $\coker 0^* = \triv$.
    \item
      Für zwei abelsche Gruppen $A_1$ und $A_2$ (und $G$ beliebig) ist
      \begin{equation*}
        \Ext(A_1 \oplus A_2,G) \cong \Ext(A_1,G) \oplus \Ext(A_2,G),
      \end{equation*}
      denn: Sind
      \begin{equation*}
        \begin{tikzcd}[row sep=tiny]
          S_1\colon 0 \arrow{r}{}  & R_1 \arrow{r}{j_1}  & F_1 \arrow{r}{}  & A_1 \arrow{r}{} & 0 \\
          S_2\colon 0 \arrow{r}{}  & R_2 \arrow{r}{j_2}  & F_2 \arrow{r}{}  & A_2 \arrow{r}{} & 0
        \end{tikzcd}
      \end{equation*}
      freie Auflösungen, so ist
      \begin{equation*}
        \begin{tikzcd}[row sep=tiny]
          S_1 \oplus S_2\colon 0 \arrow{r}{}  & R_1 \oplus R_2 \arrow{r}{j_1 \oplus j_2}  & F_1 \oplus F_2 \arrow{r}{}  & A_1 \oplus A_2 \arrow{r}{} & 0
        \end{tikzcd}
      \end{equation*}
      eine freie Auflösung von $A_1 \oplus A_2$ und daher ist
      \begin{align*}
        \Ext(A_1 \oplus A_2,G) \cong \coker {(j_1 \oplus j_2)}^*
          & \cong \coker (j_1^* \oplus j_2^*) \\
          & \cong \coker j_1^* \oplus \coker j_2^* \\
          & \cong \Ext(A_1,G) \oplus \Ext(A_2,G).
      \end{align*}
    \item
      Für jede abelsche Gruppe $G$ und $n \in \N_0$ sei
      \begin{equation*}
        n G := \{ \underset{\mathclap{g+\dotsb+g \ (\text{$n$-mal})}}{\underbrace{ng}} \in G : g \in G \} \leq G
      \end{equation*}
      Denn gilt für $\Z_n := \Z / {Z_n}$:
      \begin{equation*}
        \Ext(\Z_n, G) = G/{nG},
      \end{equation*}
      denn
      \begin{equation*}
        \begin{tikzcd}
          0 \arrow{r}{} & \Z \arrow{r}{\cdot n}[swap]{=:j}  & \Z \arrow{r}{\pi} & \Z_n \arrow{r}{}  & 0
        \end{tikzcd}
      \end{equation*}
      ist eine freie Auflösung von $\Z_n$.
      Identifiziert man nun noch $\Hom(\Z,G)$ mit $G$ (vermöge $\varphi \mapsto \varphi(1)$), so erhält man, dass $j^*\colon G \to G, j^*(g) = n \cdot g$ ist.
      Daher ist
      \begin{equation*}
        \Ext(\Z_n,G) \cong \coker j^* \cong G/{nG}.
      \end{equation*}
      \emph{Bemerkung}: Beachte, dass also
      \begin{equation*}
        \Ext(\Z_n,\Z) \cong \Z/{n\Z} = \Z_n \neq \triv = \Ext(\Z,\Z_n)
      \end{equation*}
      für $n \ge 2$.
      Also ist $\Ext$ (im Unterschied zu $\operatorname{Tor}(-,-)$, gebildet ganz ähnlich wie $\Ext$ aus $\Hom$, aus $-\oplus-$) im Allgemeinen nicht symmetrisch in seinen beiden Argumenten, genauso wie $\Hom(-,-)$ auch, denn z.B.\ ist
      \begin{equation*}
        \Hom(\Z,\Z_n) \cong \Z_n \neq \triv = \hom(\Z_n,\Z) \quad \text{für $n\geq 2$}.
      \end{equation*}
    \item
      Ist insbesondere $G$ ein Körper der Charakteristik 0 (z.B.\ $G = \Q,\R,\CC$), so ist also für alle $n \in \N$
      \begin{equation*}
        n = 1 + \dotsb + 1 \ \text{($n$-mal)}
      \end{equation*}
      verschieden von $0$, also existiert $n^{-1} \in G$ und damit ist $nG = G$, also ist $g/{nG} = \triv$.
      In diesem Fall ist also
      \begin{equation*}
        \Ext(\Z_n,G) = \triv.
      \end{equation*}
    \item
      Für endlich erzeugte abelsche Gruppe $A$, mit $\mathrm{Tor}(A) \coloneqq \left\{ a \in A : \exists m \in \N : m \cdot a = 0 \right\}$, wo also
      \begin{equation*}
        A \cong \Z^b \oplus \mathrm{Tor}(A) \cong \Z^b \oplus \Z_{n_1} \oplus \dotsb \oplus \Z_{n_r}
      \end{equation*}
      (mit $b \in \N_0$ und $n_i \in \N^{\ge 2}$) ist, erhält man also
    \begin{equation*}
      \Ext(A,G) = \triv
    \end{equation*}
    für jeden Körper $G$ der Charakteristik 0.
  \end{enumerate}
\end{beispiel}
\begin{defn}
  Sei $G$ fest und $f\colon A \to B$ Homomorphismus abelscher Gruppen.
  Wir nennen dann
  \begin{equation*}
    f^* = \Phi(f;S(A),S(B))\colon \Ext(B,G) \to \Ext(A,G)
  \end{equation*}
  den \emph{von $f$ induzierten Homomorphismus}.
\end{defn}
\begin{kommentar}
  Es wird dann
  \begin{equation*}
    T := \Ext(-,G)\colon \Ab \to \Ab
    \end{equation*}
    ein kontravarianter Funktor,
    \begin{align*}
      T(A) & = \Ext(A,G) \\
      T(f) & = \Ext(f,G) = f^*,
    \end{align*}
    denn
    \begin{align*}
      T(\id) & = \id^* = \id \\
      T(g \circ f) & = T(f) \circ T(g)
    \end{align*}
\end{kommentar}


\section{Cohomologie}

\begin{defn}
  Ein \emph{Cokettenkomplex} $(C,\delta)$ besteht aus einer Familie ${(C^k)}_{m \in \Z}$ abelscher Gruppen und Homomorphismen (\emph{Corandoperatoren}) $\delta^k\colon C^k \to C^{k+1}$ mit $\delta^{k+1} \circ \delta^k = 0$ für alle $k \in \Z$.
\end{defn}
\begin{kommentar}
  \begin{enumerate}
    \item
      Setzt man für einen Cokettenkomplex $( (C^k), (\delta^k))$
      \begin{equation*}
        C_{-k} := C^k, \quad \del_k\colon C_k \to C_{k-1}, \del_k := \delta^{-k},
      \end{equation*}
      so erhält man einen Kettenkomplex $( (C_k), (\del_k) )$ und umgekehrt.
      Auf diese übertragen sich alle Konzepte der Kettenkomplexe auf solche von Cokettenkomplexen, z.B.:
      \begin{enumerate}[(i)]
        \item
          Es heißen die Elemente von $C^k$ \emph{Coketten}, die Elemente von
          \begin{equation*}
            Z^k := \ker \delta^k \subseteq C^k
          \end{equation*}
          \emph{Cozyklen} und die Elemente von
          \begin{equation*}
            B^k := \im \delta^{k-1} \subseteq C^k
          \end{equation*}
          \emph{Coränder} und wegen $\delta^{\circ 2} = 0$ ist $B^k \subseteq  Z^k$.
        \item
          Man nennt $H^k(C) := Z^k/{B^k}$ die $k$-te \emph{Cohomologiegruppe} von $(C,\delta)$ und ihre Elemente \emph{Cohomologieklassen}.
        \item
          Für zwei Cokettenkomplexe $(C,\delta)$ und $(C',\delta')$ heißt die Familie von Homomorphismen $f = {(f^k\colon C^k \to {C'}^k)}_{k \in \Z}$ eine \emph{Cokettenabbildung}, falls
          \begin{equation*}
            {\delta'}^k \circ f^k = f^{k+1} \circ \delta^k \quad \text{für alle $k \in \Z$}
          \end{equation*}
          ist.
          Eine solche induziert dann einen Homomorphismus
          \begin{align*}
            f_{*,k}\colon H^k(C) & \to H^k(C') \\
            [\alpha] & \mapsto [f^k(\alpha)] \qquad \text{für alle $\alpha \in Z^k$}
          \end{align*}
        \item
          Eine Familie von Homomorphismen $D = {(D^k \colon C^k \to {(C')}^{k-1})}_{k \in \Z}$ heißt eine \emph{Coketten-Homotopie von $f$ nach $g$} für zwei Cokettenabbildungen $f,g \colon C \to C'$, wenn für alle $k \in \Z$ gilt:
          \begin{equation*}
            \delta^{k-1} \circ D^k + D^{k+1} \circ \delta^k = g^k - f^k.
          \end{equation*}
      \end{enumerate}
    \item
      Weiter übertragen sich Resultate über Kettenkomplexe, z.B.\
      \begin{enumerate}
        \item
          Ist
          \begin{equation*}
            \begin{tikzcd}
              0 \arrow{r}{} & C' \arrow{r}{f} & C \arrow{r}{g}  & C'' \arrow{r}{} & 0
            \end{tikzcd}
          \end{equation*}
          kurze exakte Sequenz von Cokettenkomplexen, so existiert ein natürlicher Homomorphismus (sogar eine natürliche Transformation)
          \begin{equation*}
            \delta^k_*\colon H^k(C'') \to H^{k+1}(C'),
          \end{equation*}
          sodass folgende \emph{lange Cohomologiesequenz} exakt wird:
          \begin{equation*}
            \begin{tikzcd}
              \dotsb \arrow{r}{}  & H^k(C') \arrow{r}{f^*}  & H^k(C) \arrow{r}{g^*} & H^k(C'') \arrow{r}{\delta^k_*}  & H^{k+1}(C') \arrow{r}{} & \dotsb
            \end{tikzcd}
          \end{equation*}
        \item
          Induziert für zwei Teilcokomplexe $C,C'' \subseteq C$ die Inklusion $i\colon C' + C'' \to C$ einen Isomorphismus in der Cohomologie
          \begin{equation*}
            i_*\colon H(C' + C'') \overset{\cong}{\longrightarrow} H(C),
            %i_*\colon H(C' + C'') \xrightarrow{\cong} H(C),
          \end{equation*}
          dann heißt $(C',C'')$ ein \emph{Ausschneidungspaar} von $C$ und man erhält eine exakte Sequenz (die Maier-Vietoris-Sequenz)
          \begin{equation*}
            \begin{tikzcd}[column sep=small]
              \dotsb \arrow{r}{}  & H^k(C' \cap C'') \arrow{r}{\mu}  & H^k(C') \oplus H^k(C'') \arrow{r}{\nu} & H^k(C) \arrow{r}{\Delta}  & H^{k+1}(C' \cap C'') \arrow{r}{} & \dotsb
            \end{tikzcd}
          \end{equation*}
          wobei $\mu$, $\nu$ und $\Delta$ entsprechend definiert werden.
      \end{enumerate}
  \end{enumerate}
\end{kommentar}
\begin{beispiel}[wichtig]
  Sei nun ${( C_k, \del_k )}_{k \in \Z}$ eine Kettenkomplex
  \begin{equation*}
    \begin{tikzcd}
      \dotsb \arrow{r}{}  & C_{k+1} \arrow[bend right]{rr}{0}
                            \arrow{r}{\del_{k+1}}     & C_{k} \arrow{r}{\del_k} & C_{k-1} \arrow{r}{} & \dotsb
    \end{tikzcd}
  \end{equation*}
  und $G$ eine feste abelsche Gruppe.
  Durch Anwenden von $\Hom(-,G)$ erhält man dann den \emph{zugehörigen Cokettenkomplex ${(C^k,\delta^k)}_{k \in \Z}$ mit Koeffizienten in $G$} durch
  \begin{align*}
    C^k & := \Hom(C_k,G) \\
    \delta^k \colon & C^k \to C^{k+1}, \ \delta^k := \del_{k+1}^*
  \end{align*}
  Weil $\Hom(_,G)$ funktoriell ist, ist $(C^k,\delta^k)$ tatsächlich ein Cokettenkomplex,
  \begin{equation*}
    \delta^{k+1} \circ \delta^k = \del_{k+2}^* \circ \del_{k+1}^* = {(\del_{k+1} \circ \del_{k+2})}^* = 0^* = 0.
  \end{equation*}
\end{beispiel}


\begin{kommentar}
  Ist $f\colon C \to C'$ eine Kettenabbildung, so ist
  \begin{equation*}
    f^*\colon \Hom(C',G) \to \Hom(C,G)
  \end{equation*}
  eine Cokettenabbildung und induziert damit einen Homomorphismus in der Cohomologie
  \begin{equation*}
    f^* := f_*^* \colon H^k(C',G) \to H^k(C,G).
  \end{equation*}
  Die \emph{Cohomologie mit Koeffizienten in $G$} wird damit ein kontravarianter Funktor von $\KK \xrightarrow{H} \GAb$.
  Er ist die Hintereinanderausführung des kontravarianten $\Hom(-,G)\colon \KK \to \CoKK$ mit dem Cohomologie-Funktor $H\colon \CoKK \to \GAb$.
\end{kommentar}
\emph{Frage:} Wie passen denn die (ganzzahlige) Homologie ${(H_k(C))}_{k\in\Z}$, die Cohomologie ${(H^k(C,G))}_{k\in\Z}$ und $G$ zusammen?
Ist vielleicht $H^k(C,G) \underset{\text{kan.}}{\cong} \Hom(H_k(C),G)$?
\begin{defn}
  Sei $G$ abelsche Gruppe, $(C_k,\del_k)$ ein Kettenkomplex, $(C^k,\delta^k)$ der induzierte Cokettenkomplex, $C^k = \Hom(C_k,G)$.
  Wir notierten mit $\langle -,- \rangle \colon C^k \times C_k \to G$ die folgende natürliche Paarung:
  \begin{equation*}
    \langle \varphi,c \rangle := \varphi(c) \qquad \text{für alle } \varphi\in C^k, c \in C_k.
  \end{equation*}
\end{defn}
\begin{kommentar}
  \begin{enumerate}
    \item 
      $\langle -,- \rangle$ ist bilinear, also
      \begin{align*}
        \langle \varphi+\varphi', c \rangle & = \langle \varphi, c \rangle + \langle \varphi',c \rangle \\
        \langle \varphi, c+c' \rangle & = \langle \varphi, c \rangle + \langle \varphi,c' \rangle
      \end{align*}
      für alle $\varphi,\varphi' \in C^k$, $c,c' \in C_k$.

      Beachte aber, dass der Homomorphismus
      \begin{align*}
        C_k & \to \Hom(C^k,G)\\
        c   & \mapsto [ \varphi \mapsto \langle \varphi, c \rangle ]
      \end{align*}
      im Allgemeinen weder injektiv noch surjektiv ist.
    \item
      Es gilt dann die \emph{Corand-Rand-Formel}
      \begin{equation*}
        \langle \delta\varphi,c \rangle = \langle \varphi, \del c \rangle \qquad \text{für alle $\varphi \in C^k$, $c \in C_{k+1}$}
      \end{equation*}
      denn
      \begin{equation*}
        \langle \delta\varphi, c \rangle = \delta \varphi (c) = \varphi (\del c) = \langle \varphi, \del c \rangle.
      \end{equation*}
  \end{enumerate}
\end{kommentar}
\begin{bemerkung}
  \begin{enumerate}
    \item 
      Ist $\alpha \in C^k$ ein Cozyklus und $z \in C_k$ ein Rand, so ist $\langle \alpha, z \rangle = 0$.
    \item
      Ist $\alpha \in C^k$ ein Corand und $z \in C_k$ ein Zyklus, so ist $\langle \alpha, z \rangle = 0$.
  \end{enumerate}
\end{bemerkung}
\begin{proof}
  \begin{enumerate}
    \item 
      Ist $z = \del c \in B_k$, $\alpha \in Z^k$, so gilt
      \begin{equation*}
        \langle \alpha,z \rangle = \langle \alpha, \del c \rangle = \langle \smallunderbrace{\delta\alpha}_{=0}, c \rangle = 0.
      \end{equation*}
    \item
      Ist $\alpha = \delta \varphi \in B^k$, $z \in Z_k$, so gilt
      \begin{equation*}
        \langle \alpha,z \rangle = \langle \delta\varphi, z \rangle = \langle \varphi, \smallunderbrace{\del z}_{=0} \rangle = 0.
      \end{equation*}
  \end{enumerate}
\end{proof}
\begin{kommentar}
  nach dem Homomorphiesatz drückt sich daher die Einschränkung der natürlichen Paarung zwischen Cohomologie und Homologie herunter,
  \begin{equation*}
    \begin{tikzcd}
      Z^k \times Z_k  \arrow[swap]{d}{\pi^k \times \pi_k}
                      \arrow{r}{\langle -,- \rangle} & G \\
      H^k \times H_k \arrow[swap]{ur}{\langle \,\boldsymbol\cdot\,,\,\boldsymbol\cdot\, \rangle}
    \end{tikzcd}
  \end{equation*}
  wobei $\pi^k \colon Z^k \to H^k$ und $\pi_k \colon Z_k \to H_k$ die natürlichen Projektionen sind.
\end{kommentar}
Es ist also wohldefiniert:
\begin{defn}
  Sei $C = (C_k,\del_k)$ ein Kettenkomplex, $G$ eine abelsche Gruppe und $(C^k,\delta^k)$ der induzierte Cokettenkomplex.
  Dann definiert man die natürliche Paarung
  \begin{equation*}
    \langle - , - \rangle\colon H^k(C,G) \times H_k(C) \to G
  \end{equation*}
  durch
  \begin{equation*}
    \langle [\alpha], [z] \rangle := \langle \alpha, z \rangle.
  \end{equation*}
\end{defn}
\begin{kommentar}
  \begin{enumerate}
    \item 
      Da $\langle - ,- \rangle \colon C^k \times C_k \to G$ bilinear ist, erhält man also einen (natürlichen) Homomorphismus
      \begin{align*}
        \kappa \colon H^k(C,G) & \to \Hom(H_k(C),G) \\
        \kappa ([\alpha])([z]) & := \langle [\alpha],[z] \rangle
      \end{align*}
    \item
      \emph{Frage}: ist das ein Isomorphismus?
  \end{enumerate}
\end{kommentar}
\stepcounter{prop}
\textbf{(\theprop) Ab sofort.\ }
Sei unser Kettenkomplex $C$ \emph{frei}, d.h.\ alle abelschen Gruppen $C_k$ seien frei.
Wir benutzen weiter, dass jede Untergruppe einer frei-abelschen Gruppe selbst auch wieder frei ist.
\begin{vorbereitung}
  \begin{enumerate}
    \item 
      Bezeichne zunächst mit $i_k$ und $j_k$ die Inklusionen $i_k \colon B_k \hookrightarrow Z_k$ und $j_k \colon Z_k \hookrightarrow C_k$ und mit $\del'\colon C_k \to B_{k-1}, \del'_k(c) := \del_k(c)$, sodass also
      \begin{equation*}
        j_{k-1} \circ i_{k-1} \circ \del' = \del.
      \end{equation*}
      Betrachte nun die exakte Sequenz
      \begin{equation*}
        \begin{tikzcd}
          0 \arrow{r}{} & Z_k \arrow{r}{j_k}  & C_k \arrow[bend left]{l}{l_k}
                                                    \arrow{r}{\del'_k}        & B_{k-1} \arrow{r}{} & 0
        \end{tikzcd}
        \tag{$*$}\label{eqn:inklusionssequenz}
      \end{equation*}
      Mit $C_{k-1}$ ist auch $B_{k-1} \subseteq C_{k-1}$ frei, deshalb spaltet~\eqref{eqn:inklusionssequenz}.
      Es gibt also ein Linksinverses $l_k\colon C_k \to Z_k$ zu $j_k$, $l_k \circ j_k = \id_{Z_k}$.
      Es ist deshalb auch exakt (und spaltet):
      \begin{equation*}
        \begin{tikzcd}
          0 \arrow{r}{} & \Hom(B_{k-1},G) \arrow{r}{{\del'_k}^*}  & \Hom(C_k,G) \arrow{r}{j_k^*}  & \Hom(Z_k,G) \arrow[bend left]{l}{l_k^*}
                                                                                                              \arrow{r}{} & 0.
        \end{tikzcd}
      \end{equation*}
    \item
      Betrachte außerdem die kurze exakte Sequenz
      \begin{equation*}
        \begin{tikzcd}
          0 \arrow{r}{} & B_{k-1} \arrow{r}{i_{k-1}}  & Z_{k-1} \arrow{r}{p_{k-1}}  & H_{k-1} \arrow{r}{} & 0.
        \end{tikzcd}
        \tag{$**$}\label{eqn:inklusion_projektion_sequenz}
      \end{equation*}
      Es ist dann auch exakt:
      \begin{equation*}
        \begin{tikzcd}
          0 \arrow{r}{} & \Hom(H_{k-1},G) \arrow{r}{p_{k-1}^*}  & \Hom(Z_{k-1},G) \arrow{r}{i_{k-1}^*}  & \Hom(B_{k-1},G).
        \end{tikzcd}
      \end{equation*}
  \end{enumerate}
\end{vorbereitung}
\begin{lemma}
\label{thm:induzierter_hom_lemma}
  Es ist
  \begin{align*}
    \im {\del'_k}^* & \subseteq Z^k(C,G), & {\del'_k}^* (\im i_{k-1}^*) & \subseteq B^k(C,G)
  \end{align*}
  Deshalb induziert ${\del'_k}^*$ einen Homomorphismus
  \begin{equation*}
    h\colon \Hom(B_{k-1},G) / {\im i_{k-1}^*} \to Z^k(C,G) / {B^k(C,G)}
  \end{equation*}
  mit $h([\varphi]) = [{\del'_k}^*(\varphi)]$.
  \begin{equation*}
    \begin{tikzcd}
      \Hom(B_{k-1},G)   \arrow{d}{\pi_{k-1}}
                        \arrow{r}{{\del'_k}^*}  & Z^k(C,G) \arrow{d}{\pi^k}\\
      \coker i_{k-1}^*  \arrow{r}{h}            & H^k(C,G)
    \end{tikzcd}
  \end{equation*}
\end{lemma}
\begin{vorbereitung}
  Weil schließlich~\eqref{eqn:inklusion_projektion_sequenz} eine freie Auflösung von $H_{k-1}(C)$ ist, gibt es einen (natürlichen) Isomorphismus
  \begin{equation*}
    \Phi \colon \Ext(H_{k-1}(C),G) \xrightarrow{\cong} \coker i_{k-1}^*.
  \end{equation*}
  Fassen wir $\Phi$ und $h$ zusammen, so erhält man einen (natürlichen) Homomorphismus $\rho := h \circ \Phi$,
  \begin{equation*}
    \rho \colon \Ext(H_{k-1}(C),G) \to H^k(C,G).
  \end{equation*}
\end{vorbereitung}
Es gilt nun
\begin{satz}[Universelles Koeffiziententheorem]
  Sei $(C,\del)$ ein freier Kettenkomplex, $G$ abelsche Gruppe und für jedes $k \in \Z$ seien $\rho$ und $\kappa$ die Homomorphismen von oben.
  Dann ist die folgende Sequenz (natürlich) exakt und spaltet:
  \begin{equation*}
    \begin{tikzcd}
      0 \arrow{r}{} & \Ext(H_{k-1}(C),G)  \arrow{r}{\rho} & H^k(C,G)  \arrow{r}{\kappa} & \Hom(H_k(C),G)  \arrow{r}{} & 0.
    \end{tikzcd}
  \end{equation*}
\end{satz}
\begin{proof}[Beweis zu Lemma~\ref{thm:induzierter_hom_lemma}]
  \begin{enumerate}
    \item 
      $\im {\del'_k}^* \subseteq Z^k(C,G)$:
      Für $\varphi \in \Hom(B_{k-1},G)$ ist
      \begin{equation*}
        \delta^k({\del'_k}^* \varphi) = \delta^k(\varphi \circ \del'_k) = \varphi \circ \underbrace{\del'_k \circ \del_{k+1}}_{=0} = 0.
      \end{equation*}
    \item
      ${\del'_k}^* (\im i_{k-1}^*) \subseteq B^k(C,G)$:
        Sei $\varphi \in \im i_{k-1}^* \subseteq \Hom(B_{k-1},G)$, also $\varphi = i_{k-1}^* \varphi'$ für ein $\varphi' \in \Hom(Z_{k-1},G)$ (d.h.\ $\varphi'\colon Z_{k-1} \to G$ ist Fortsetzung von $\varphi$).
        Weil aber~\eqref{eqn:inklusionssequenz} spaltet (mit $k-1$ statt $k$), existiert sogar Fortsetzung $\psi$ von $\varphi'$ auf ganz $C_{k-1}$,
        \begin{equation*}
          \psi := \varphi' \circ l_{k-1},
        \end{equation*}
        wobei $l_{k-1} \circ j_{k-1} = \id_{Z_{k-1}}$ ist,
        \begin{equation*}
          \psi \circ j_{k-1} \circ i_{k-1} = \varphi' \circ \underbrace{l_{k-1} \circ j_{k-1}}_{=\id} {}\circ i_{k-1} = \varphi
        \end{equation*}
        Es folgt:
        \begin{align*}
          \delta^{k-1} (\psi)
            & = \psi \circ \del_k\\
            & = \psi \circ j_{k-1} \circ i_{k-1} \circ \del'_k\\
            & = \varphi \circ \del'_k\\
            & = {\del'_k}^* (\varphi) \in B^k(C,G)
        \end{align*}
  \end{enumerate}
\end{proof}


%%%%%%%%%%%%%%%%%%%%%%%%%%%%%%%%%%%%
%%%%%%%% Beginn WS 17/18 %%%%%%%%%%%
%%%%%%%%%%%%%%%%%%%%%%%%%%%%%%%%%%%%


\begin{vorbereitung}
  \begin{enumerate}
    \item
      Seien $i_k \colon B_k \hookrightarrow Z_k$, $j_k \colon Z_k \hookrightarrow C_k$ die Inklusionen und sei $\del_k'\colon C_k \to B_{k-1}$ gegeben durch
      \begin{equation*}
        \del_k'(c) \coloneqq \del_k(c).
      \end{equation*}
      Dann gilt $\del_k = j_k \circ i_k \circ \del_k'$.
      Betrachte nun folgende exakte Sequenz:
      \begin{equation}
        \begin{tikzcd}
          \tag{$\star$}
          \label{seq:kes_auf_kette}
          0 \arrow{r}{} & Z_k \arrow{r}{j_k}
          & C_k \arrow[bend left]{l}{l_k}
          \arrow{r}{\del_k'}
          & B_{k-1} \arrow{r}{}
          & 0
        \end{tikzcd}
      \end{equation}
      Weil mit $C_{k-1}$ auch $B_{k-1}$ frei ist, spaltet~\eqref{seq:kes_auf_kette}.
      Man erhält also ein Linksinverses $l_k \colon C_k \to Z_k$ mit $l_k \circ j_k = \id_{Z_k}$.

      Es ist damit auch exakt:
      \begin{equation*}
        \begin{tikzcd}
          0 \arrow{r}{}
          & \Hom(B_{k-1},G) \arrow{r}{\del_k'^*}
          & \Hom(C_k,G) \arrow{r}{j_k^*}
          & \Hom(Z_k,G) \arrow{r}{}
          & 0
        \end{tikzcd}
      \end{equation*}
    \item
      Betrachte außderdem die kurze exakte Sequenz
      \begin{equation}
        \tag{$\star\star$}
        \label{seq:kes_mit_homologie}
        \begin{tikzcd}
          0 \arrow{r}{}
          & B_{k-1} \arrow{r}{i_{k-1}}
          & Z_{k-1} \arrow{r}{p_{k-1}}
          & H_{k-1}(C) \arrow{r}{}
          & 0
        \end{tikzcd}
      \end{equation}
      Es ist dann auch exakt:
      \begin{equation*}
        \begin{tikzcd}
          0 \arrow{r}{}
          & \Hom(H_{k-1}(C),G) \arrow{r}{p_{k-1}^*}
          & \Hom(Z_{k-1},G) \arrow{r}{i_{k-1}^*}
          & \Hom(B_{k-1},G).
        \end{tikzcd}
      \end{equation*}
  \end{enumerate}
\end{vorbereitung}

\begin{lemma}
  Es ist
  \begin{itemize}
    \item
      $\im(\del_k'^*) \subseteq Z^k(C,G)$,
    \item
      $\del_{k}'^*(\im(i_{k-1}^*)) \subseteq B^k(C,G)$.
  \end{itemize}
  Daher induziert $\del_k'^*\colon \Hom(B_{k-1},G) \to \Hom(C_k,G)$ nach dem Homomorphiesatz (genau) einen Homomorphismus
  \begin{equation*}
    h \colon \Hom(B_{k-1},G) / \im(i_{k-1}^*) = \coker(i_{k-1}^*) \to Z^k(C,G) / B^k(C,G) = H^k(C,G)
  \end{equation*}
  mit $h([\varphi]) = [\del_k'^*(\varphi)]$.
  %\label{}
  \begin{equation*}
    \begin{tikzcd}
      \Hom(B_{k-1},G) \arrow{r}{\del_k'^*} \arrow{d}{\pi_{k-1}}
      & Z^k(C,G) \subseteq \Hom(C_k,G) = C^k \arrow{d}{\pi_k} \\
      \Hom(B_{k-1},G) / \im(i_{k-1}^*) = \coker(i_{k-1}^*) \arrow[dashed]{r}{h}
      & H^k(C,G)
    \end{tikzcd}
  \end{equation*}
\end{lemma}
\begin{proof}
  \begin{enumerate}
    \item
      Für $\varphi \in \Hom(B_{k-1},G)$ ist $\delta^k \circ \del_{k}'^*(\varphi) = \delta^k(\varphi \circ \del_k') = \varphi \circ \underbrace{\del_k' \circ \del_{k+1}}_{= 0} = 0$.
    \item
      Sei $\varphi \in \im(i_{k-1}^*) \subseteq  \Hom(B_{k-1},G)$, also: $\varphi = i_{k-1}^*(\varphi')$ mit $\psi \in \Hom(Z_{k-1},G)$, d.h. $\varphi = \varphi' \circ i_{k-1}$.
      Weil aber~\eqref{seq:kes_auf_kette} spaltet, (mit $k-1$ statt $k$), existiert sogar eine Fortsetzung von $\varphi'$ auf ganz $C_{k-1}$:
      \begin{equation*}
        \psi \coloneqq \varphi' \circ l_{k-1} \implies \psi \circ j_{k-1} = \varphi' \circ {} \underbrace{l_{k-1} \circ j_{k-1}}_{= \id} = \varphi'.
      \end{equation*}
      Es ist dann $\delta^k(\psi) = \del_k^* (\psi) = \underbrace{\psi \circ j_{k-1} \circ i_{k-1}}_{\varphi} {}\circ \del_k' = \varphi \circ \del_{k}' = \del_{k}'^*(\varphi) \implies \del_{k}'^* \in B^k(C,G)$.
  \end{enumerate}
\end{proof}
\begin{vorbereitung}
  Weil schließlich~\eqref{seq:kes_mit_homologie} eine freie Auflösung von $H_{k-1}(C)$ ist, haben wir einen (natürlichen) Isomorphismus
  \begin{equation*}
    \Phi\colon \Ext(H_{k-1}(C),G) \to \coker(i_{k-1}^*).
  \end{equation*}
  Schalten wir diesen vor $h \colon \coker(i_{k-1}^*) \to H^K(C,G)$, so erhalten wir einen (natürlichen) Homomorphismus
  \begin{equation*}
    \rho \coloneqq h \circ \Phi \colon \Ext(H_{k-1}(C),G) \to H^k(C,G).
  \end{equation*}
\end{vorbereitung}
Es gilt nun
\begin{satz}[Universelles Koeffiziententheorem]
  Sei $(C,\del)$ ein freier Kettenkomplex, $G$ eine abelsche Gruppe und $\rho$ und $\kappa$ wie beschrieben.
  Dann ist die folgende Sequenz exakt und spaltet:
  \begin{equation*}
    \label{seq:koeffiziententheorem}
    \tag{$\ast$}
    \begin{tikzcd}
      0 \arrow{r}{}
      & \Ext(H_{k-1}(C),G) \arrow{r}{\rho}
      & H^K(C,G) \arrow{r}{\kappa}
      & \Hom(H_k(C), G) \arrow{r}{} \arrow[bend left]{l}{r}
      & 0
    \end{tikzcd}
  \end{equation*}
\end{satz}
\begin{proof}
  \begin{enumerate}[(i)]
    \item
      $\rho$ ist injektiv:
      Da $\Phi$ Isomorphismus ist, muss man zeigen, dass $h$ injektiv ist.
      Sei dazu $\varphi \in \Hom(B_{k-1},G)$ mit $h([\varphi]) = 0$, d.h.:
      \begin{equation*}
        0 \equiv \del_{k}'^*(\varphi) \equiv \varphi \circ \del_k' \mod B^k
      \end{equation*}
      Also gibt es ein $\psi \colon C_{k-1} \to G$ mit
      \begin{equation*}
        \varphi \circ \del_k' = \delta^{k-1}(\psi) = \psi \circ \del_k = \psi \circ j_{k-1} \circ i_{k-1} \circ \del_k'.
      \end{equation*}
      Da $\del_k'$ surjektiv ist, folgt:
      \begin{equation*}
        \varphi = \psi \circ j_{k-1} \circ i_{k-1} = i_{k-1}^*(\psi \circ j_{k-1}) \in \im(i_{k-1}^*)
      \end{equation*}
      (also fortsetzbar auf $Z_{k-1}$), d.h.: $[\varphi] = 0$ in $\coker(i_{k-1}^*)$, somit ist $h$ injektiv.
    \item
      $\im \rho = \ker \kappa$:
      Sei dafür $\varphi \in \Hom(B_{k-1},G)$ und $\psi \coloneqq \del_{k}'^* (\varphi) = \varphi \circ \del_k'$, also:
      \begin{equation*}
        [\psi] = h([\varphi]).
      \end{equation*}
      \emph{Zeige}: $\kappa([\psi]) = 0$ ($\implies \kappa \circ h = 0)$.
      Sei dazu $z \in Z_k$ beliebig.
      Dann ist:
      \begin{equation*}
        \kappa \circ h ([\varphi])([z]) = \kappa([\psi])([z]) = \langle \psi, z \rangle = \langle \varphi \circ \del_k', z \rangle = \varphi(\underbrace{\del_k(z)}_{= 0}) = 0
      \end{equation*}
      Daraus folgt $\im \rho \subseteq \ker \kappa$.
    \item
      $\ker \kappa \subseteq \im \rho$:
      Sei dazu $\psi \in Z^k$, sodass $\langle \psi, z \rangle = 0$ für alle $z \in Z_k$, also $[\psi] \in \ker \kappa$.
      \emph{Behauptung}: Es existiert Homomorphismus $\varphi \colon B_{k-1} \to G$  mit $\psi = \varphi \circ \del_k'$, denn dann ist $[\psi] = h([\varphi])$ (und damit $[\psi] \in \im h = \im \rho$).
      Wegen $\langle \psi, z \rangle = 0$ für alle $z \in Z_k$ ist
      \begin{equation*}
        j_k^*(\psi) = \psi \circ j_k = 0.
      \end{equation*}
      Weil die duale Sequenz von~\eqref{seq:kes_auf_kette} bei $\Hom(C_{k-1},G)$ exakt ist, existiert ein $\varphi \in \Hom(B_{k-1},G)$ mit $\del_{k}'^*(\varphi) = \psi$.
      Daher ist
      \begin{equation*}
        [\psi] = [\varphi \circ \del_k'] = h([\varphi]) \in \im h.
      \end{equation*}
    \item
      $\kappa$ ist surjektiv und hat ein Rechtsinverses:
      Sei $\lambda \colon H_k(C) \to G$ beliebig und (wie früher) $l_k \colon C_k \to Z_k$ linksinvers zu $j_k$, $l_k \circ j_k = \id$.
      Setze dann $\varphi \colon C_k \to G, \varphi \coloneqq \lambda \circ p_k \circ l_k$.
      Dann ist wegen $\im (l_k \circ \del_{k+1}) = \im \del_{k-1} = B_k$:
      \begin{equation*}
        \delta^k(\varphi) = \varphi \circ \del_{k+1} = \lambda \circ \underbrace{p_k \circ \underbrace{l_k \circ \del_{k+1}}_{\to B_k}}_{= 0},
      \end{equation*}
      also $\varphi \in Z^k(C,G)$.
      Weiter gilt für alle $z \in Z_k$:
      \begin{equation*}
        \kappa([\varphi])([z)] = \langle \varphi, j_k(z) \rangle = \varphi \circ j_k(z) = \lambda \circ p_k \circ \underbrace{l_k \circ j_k}_{= 0}(z)
        = \lambda([z])
      \end{equation*}
      und somit $\kappa([\varphi]) = \lambda$, also $\kappa$ surjektiv.
      Außerdem ist die Zuordnung
      \begin{align*}
        r \colon \Hom(H_k(C),G) & \to H^k(C,G) \\
        \lambda & \mapsto [\lambda \circ p_k \circ l_k]
      \end{align*}
      homomorph und damit $r$ rechtsinvers zu $\kappa$.
  \end{enumerate}
\end{proof}



\begin{kommentar}
  \begin{enumerate}
    \item
      Weil Sequenz~\eqref{seq:kes_auf_kette} spaltet, erhält man insbesondere einen Isomorphismus
      \begin{equation*}
        \label{eqn:iso_cohom-ext_plus_hom}
        \tag{$\ast\ast$}
        H^k(C,G) \cong \Ext(H_{k-1}(C),G) \oplus \Hom(H_k(C),G)
      \end{equation*}
    \item
      Die Sequenz~\eqref{seq:koeffiziententheorem} is in dem Sinne natürlich wie die Homomorphismen $\rho$ und $r$ natürliche Transformationen von $F_1 = \Ext(H_{k-1}(-),G)$, $F_2 = H^k(-,G)$und $F_3 = \Hom(H_k(-),G)$, d.h.:
      Ist $f \colon C \to C'$ Kettenabbildung zwischen freien Kettenkomplexen, so kommutiert:
      \begin{equation*}
        \begin{tikzcd}
          0 \arrow{r}{}
          & \Ext(H_{k-1}(C),G) \arrow{r}{\rho_C}
          & H^k(C,G) \arrow{r}{\kappa_C}
          & \Hom(H_k(C),G) \arrow{r}{}
          & 0 \\
          0 \arrow{r}{}
          & \Ext(H_{k-1}(C'),G) \arrow{r}{\rho_{C'}}
            \arrow{u}{f^*}
          & H^k(C',G) \arrow{r}{\kappa_{C'}}
            \arrow{u}{f^*}
          & \Hom(H_k(C'),G) \arrow{r}{}
            \arrow{u}{f^*}
          & 0
        \end{tikzcd}
      \end{equation*}
      wo $f^* = F(f)$ ist mit $F = F_1,F_2$ beziehungsweise $F_3$.
      Allerdingskann man den Isomorphismus in~\eqref{eqn:iso_cohom-ext_plus_hom} für jedes freie $C$ nicht so wählen, dass er natürlich ist.
      [Stö/Zie: S. 264/265]
      \todo[]{Referenz hinzufügen}
    \item
      Is die Homologie von $(C,\del)$ endlich erzeugt, so kann man die Cohomologie von $C$ mit Koeffizienten in $G$ berechnen, falls $G = \Z,\Q,\R$ oder $\mathbb C$.
      Ist nämlich
      \begin{equation*}
        H_k(C) \cong \Z^{b_k} \oplus \Tor(H_k(C)),
      \end{equation*}
      so ist wegen
      \begin{align*}
        \Hom(\Tor(H_k(C)),G) & = \triv,\\
      \Hom(\Z^b,G) & = G^k
      \end{align*}
      Andererseits ist
      \begin{equation*}
        \Ext(H_{k-1}(C),G) =
        \begin{cases}
          0 & \text{für $G = \Q,\R$ oder $\mathbb C$}\\
          \Tor(H_{k-1}(C))  & \text{für $G = \Z$}
        \end{cases}
      \end{equation*}
      dann $\Ext(\Z_n,\Z) = \Z_n$ und $\Tor(H_{k-1}(C)) = \Z^{n_1} \oplus \dotsb \oplus \Z^{n_r}$ mit $n_1, \dotsc n_r \in \N$ mit $n_1 | n_2 | \dotsb | n_r$.
      Für $G = \Z$ hat die Cohomologie
      \begin{equation*}
        H^*(C) \coloneqq H^*(C,G) \coloneqq \bigoplus_{k \in \Z} H^k(C,\Z)
      \end{equation*}
      die gleiche Information wie die Homologie $H_*(C) = \bigoplus_{k\in\Z}H_k(C)$, denn ist
      \begin{equation*}
        H^k(C) = \Z^{b_k} \oplus \underbrace{\Tor(H^k(C))}_{= \Z_{n_1^k} \oplus \dotsb \oplus \Z_{n_{r_k}^k}},
      \end{equation*}
      so ist die "`Cobettizahl"' $b_k$ also gleich der Bettizahl $\rg(H^k(C)) = \rg(H_k(C))$ und die "`Cotorsionskoeffizienten"' $n_1^k, \dotsc, n_{r_k}^k$ im Grad $k$ sind die Torsionskoeffizienten $m_{1}^{k-1}, \dotsc, m_{s_{k-1}}^{k-1}$ im Grad $k-1$, $\Tor(H^k(C)) \cong \Tor(H_{k-1}(C))$.
  \end{enumerate}
\end{kommentar}


\section{Homologie mit Koeffizienten}

\begin{motivation}
  \begin{enumerate}
    \item
      Homologie mit Koeffizienten (aus einer beliebigen abelschen  Gruppe $G$) wird aus Kettenkomplexen gebildet, in denen die Kettengruppen direkte Summen von Kopien von $G$ (nicht von $\Z$) sind, ihre Elemente also von der Form
      \begin{equation*}
        c = g_{1}\sigma_1 + \dotsb + g_r\sigma_r
      \end{equation*}
      mit $r \in \N_0$, $g_j \in G$ ($j = 1,\dotsc, r)$ und (in der singulären Theorie) singulären $k$-Simplexen $\sigma_j$ ($j = 1, \dotsc, r)$.
    \item
      So bekommt man beispielsweise für die Koeffizientengruppe $G = \Z_2$, dass für den Randoperator $\del$ des zellulären Kettenkomplexes von $\mathbb{P}^n(\R)$ mit Koeffizienten in $\Z_2$ gilt: $\del = 0$.
      Das führt dann zu
      \begin{equation*}
        H_k(\mathbb{P}^n(\R),\Z_2) =
        \begin{cases}
          \Z_2 & \text{für $0 \le k \le n$}\\
          0   & \text{für $k > n$}
        \end{cases}
      \end{equation*}
  \end{enumerate}
\end{motivation}

\begin{defn}
  Seien $A$ und $B$ abelsche Gruppen.
  Wir nennen ein Paar $(T,t)$ bestehend aus einer abelschen Gruppe $T$ und einer bilinearen Abbildung $t \colon A \times B \to T$ ein \emph{Tensorprodukt von $A$ und $B$}, wenn folgende universelle Eigenschaft gilt:
  Ist $(C,s)$ ein Konkurrenzpaar ($C$ abelsche Gruppe, $s$ bilinear), so gibt es genau einen Homomorphismus $\Phi \colon T \to C$ mit $\Phi \circ t = s$.
  \begin{equation*}
    \begin{tikzcd}
      A\times B \arrow{r}{s}
          \arrow{d}{t}
      & C\\
      T \arrow[dashed,swap]{ur}{\Phi}
    \end{tikzcd}
  \end{equation*}
\end{defn}

\begin{kommentar}
  \begin{enumerate}
    \item
      Ein Tensorprodukt $(T,t)$ ist -- wenn es existiert -- im folgenden Sinne eindeutig bestimmt:
      Sind $(T_1,t_1)$ und $(T_2,t_2)$ zwei Tensorprodukte, so existiert ein (sogar eindeutiger) Isomorphismus $\Phi \colon T_1 \to T_2$ mit $\Phi \circ t_1 = t_2$ (Übung).
    \item
      Die Existenz kann man so sehen:
      Bilde zunächst die frei-abelsche Gruppe $(\F(A\times B),i)$ und betrachte dann die Untergruppe $R \subseteq \F(A \times B)$, die von allen Elementen
      \begin{gather*}
        (a_1 + a_2, b) - (a_1,b) - (a_2,b)\\
        (a,b_1 + b_2) - (a,b_1) - (a, b_2)
      \end{gather*}
    mit $a,a_1,a_2 \in A$ sowie $b,b_1,b_2 \in B$ erzeugt wird.
    Ist $i \colon A \times B \hookrightarrow \F(A \times B)$ die natürliche Inklusion und $\pi \colon \F(A \times B) \to \F(A \times B) / R$ die kanonische Projektion, so setze
    \begin{align*}
      A \otimes B & \coloneqq \F(A \times B) / R, \\
      \otimes & \coloneqq \pi \circ i \colon A \times B \to A \otimes B.
    \end{align*}
    Es ist dann $A \otimes B$ eine abelsche Gruppe und $\otimes$ ist bilinear, denn mit $\otimes(a,b) \eqqcolon a \otimes b$:
    \begin{equation*}
      (a_1 + a_2) \otimes b - a_1 \otimes b - a_2 \otimes b = \pi(\underbrace{(a_1 + a_2, b) - (a_1,b) - (a_2,b)}_{\in R}) = 0
    \end{equation*}
    und genau so:
    \begin{equation*}
      a \otimes (b_1+b_2) = a \otimes b_1 + a \otimes b_2
    \end{equation*}
    für alle $a,a_1,a_2 \in A$ und $b,b_1,b_2 \in B$.
    Es ist auch
    \begin{equation*}
      n (a \otimes b) = (na) \otimes b = a \otimes (nb)
    \end{equation*}
    für alle $n \in \N$ (Übung).

    Zur universellen Eigenschaft:
    Ist $s \colon A \times B \to C$ Konkurrent (zu $\otimes \colon A \times B \to A \otimes B$), so existiert zunächst nach der universellen Eigenschaft von $(\F(A \times B), i)$ ein eindeutig bestimmter Homomorphismus $\tilde \Phi \colon \F(A \times B) \to C$ mit $\tilde \Phi \circ i = s$.
    \begin{equation*}
      \begin{tikzcd}
        A \times B \arrow{r}{s}
          \arrow{d}{i}
        & C\\
        \F(A \times B) \arrow[dashed,swap]{ur}{\tilde \Phi}
      \end{tikzcd}
    \end{equation*}
    Da $\tilde \Phi | _R = 0$ ist, weil $s$ bilinear ist, existiert nach der universellen Eigenschaft des Quotienten $(A \otimes B, \pi)$ genau ein Homomorphismus $\Phi \colon A \otimes B \to C$ mit $\Phi \circ \pi = \tilde \Phi$.
    \begin{equation*}
      \begin{tikzcd}
        \F(A \times B) \arrow{r}{\tilde \Phi}
          \arrow{d}{\pi}
        & C\\
        A \otimes B \arrow[dashed,swap]{ur}{\Phi}
      \end{tikzcd}
    \end{equation*}
    Es kommutiert also auch das große Dreieck im Diagramm, $\Phi \circ \otimes = s$
    \begin{equation*}
      \begin{tikzcd}
        A \times B \arrow{r}{s}
          \arrow{d}{i}
          \arrow[bend right,swap,out=305,in=235,looseness=1.4]{dd}{\oplus}
        & C\\
        \F(A \times B) \arrow[swap]{ur}{\tilde \Phi}
          \arrow{d}{\pi}
        \\
        A \otimes B \arrow[bend right,swap]{uur}{\Phi}
      \end{tikzcd}
    \end{equation*}
    $\Phi$ is auch eindeutig, denn kommutiert das große Dreieck, so auch das untere kleine.
  \end{enumerate}
\end{kommentar}




\begin{kommentar}
    Jedes Element $t$ in $A \tensor B$ ist damit von der Form
    \begin{equation*}
      t = a_1 \tensor b_1 + \dotsb a_r \tensor b_r
    \end{equation*}
    mit $r \in \N_0$, $a_1, \dotsc a_r, \in A$ und $b_1, \dotsc, b_r \in B$.
    Beachte aber, dass diese Darstellung im Allgemeinen nicht eindeutig ist.
\end{kommentar}

\begin{defn}
  Seien $f \colon A \to A'$ und $g \colon B \to B'$ Homomorphismen.
  Dann wird durch
  \begin{equation*}
    \label{eqn:tensor_hom}
    \tag{$\ast$}
    h(a \tensor b) \coloneqq f(a) \tensor g(b)
  \end{equation*}
  ein Homomorphismus $h \colon A \tensor B \to A' \tensor B'$ definiert, der mit $h \eqqcolon f \tensor g$ bezeichnet wird.
\end{defn}

\begin{kommentar}
  \begin{enumerate}
    \item
      Durch~\eqref{eqn:tensor_hom} wird tatsächlich genau ein Homomorphismus $h \colon A \tensor B \to A' \tensor B'$ gegeben, denn:
      Betrachtet man
      \begin{align*}
        f \times g \colon A \times B & \to A' \times B', \\
        (f \times g) (a,b) & \coloneqq (f(a),g(b)),
      \end{align*}
      so ist $H \coloneqq \tensor \circ (f \times g) \colon A \times B \to A' \tensor B'$ bilinear und daher existiert eindeutig bestimmtes $h \colon A \tensor B \to A' \tensor B'$ mit $h \circ \tensor = H$, das heißt $h(a \tensor b) = f(a) \tensor g(b)$.
      \begin{equation*}
        \begin{tikzcd}
          A \times B
            \arrow{r}{f \times g}
            \arrow{d}{\tensor}
            \arrow{dr}{H}
          & A' \times B'
            \arrow{d}{\tensor}
          \\
          A \tensor B
            \arrow{r}{h}
          & A' \tensor B'
        \end{tikzcd}
      \end{equation*}
    \item
      Sind $f' \colon A' \to A''$ und $g' \colon B' \to B''$ weitere Homomorphismen, so gilt (offenbar)
      \begin{equation*}
        (f' \circ f) \tensor (g' \circ g) = (f' \tensor g') \circ (f \tensor g),
      \end{equation*}
      insbesondere
      \begin{equation*}
        (f \tensor g) = (f \tensor \id) \circ (\id \tensor g).
      \end{equation*}
      Hält man daher einen Faktor $G \in \Ob(\Ab)$ fest, so erhält man durch
      \begin{align*}
        F \colon \Ab & \to \Ab\\
        F(A) & \coloneqq A \tensor G \\
        F(f) & \coloneqq f \tensor \id_G
      \end{align*}
      einen (covarianten) Funktor ($F\colon \Ab \times \Ab \to \Ab$ ist ein Bifunktor, covariant in beiden Argumenten).
  \end{enumerate}
\end{kommentar}

\begin{bemerkung}
\label{bem:tensor_summen_iso}
  Seien $A_1$, $A_2$ und $G$ abelsche Gruppen.
  Dann definiert
  \begin{align*}
    \Phi \colon (A_1 \oplus A_2) \tensor G & \to (A_1 \tensor G) \oplus (A_2 \tensor G),\\
    \Phi( (a_1,a_2) \tensor g) & \coloneqq (a_1 \tensor g, a_2 \tensor g)
  \end{align*}
  einen (kanonischen) Isomorphismus von $(A_1 \oplus A_2) \tensor G$ nach $A_1 \tensor G \oplus A_2 \tensor G$, wobei hier vereinbart wird, dass $\tensor$ stärker bindet als $\oplus$.
\end{bemerkung}
\begin{proof}
  Wieder ist
  \begin{align*}
    \tilde \Phi \colon (A_1 \oplus A_2) \times G & \to (A_1 \tensor G) \oplus (A_2 \tensor G), \\
    ( (a_1, a_2), g) & \mapsto (a_1 \tensor g, a_2 \tensor g)
  \end{align*}
  bilinear und daher $\Phi$ (als Homomorphismus) durch~\eqref{eqn:tensor_hom} wohldefiniert.
  Ebenso ist
  \begin{align*}
    \Psi \colon A_1 \tensor G \oplus A_2 \tensor G & \to (A_1 \oplus A_2) \tensor G, \\
    (a_1 \tensor g_1, a_2 \tensor g_2) & \mapsto (a_1,0) \tensor g_1 + (0,a_2) \tensor g_2
  \end{align*}
  wohldefiniert und
  \begin{align*}
    \Psi \circ \Phi & = \id,
    & \Phi \circ \Psi & = \id,
  \end{align*}
  also ist $\Phi$ ein Isomorphismus.
\end{proof}

\begin{beispiel}
  \begin{enumerate}
    \item
      Ähnlich wie bei Bemerkung~\ref{bem:tensor_summen_iso} sieht man, dass
      \begin{equation*}
        A \tensor B \cong B \tensor A
      \end{equation*}
      kanonisch isomorph vermöge $a \tensor b \mapsto b \tensor a$ ist.
    \item
      Für alle abelschen Gruppen ist
      \begin{equation*}
        A \tensor \Z \cong A
      \end{equation*}
      vermöge
      \begin{equation*}
        a \tensor n \mapsto n a,
      \end{equation*}
      wo $n a \coloneqq \underbrace{a + a + \dotsb + a}_{n\text{-mal}}$ für $n \in \N$, $0 a \coloneqq 0$ und $n a \coloneqq (-n) (-a)$ für $n < 0$ bezeichnet.
      Dann ist $A \to A \tensor \Z, a \mapsto a \tensor 1$ offenbar invers dazu.
    \item
      Ist $G$ ein Körper und $A$ abelsche Gruppe, so hat $A \tensor G$ sogar die Struktur eines $G$-Vektorraums vermöge
      \begin{equation*}
        g \cdot (a \tensor g') = (a \tensor gg').
      \end{equation*}
    \item
      Ist $A = \Z^r$, so ist deshalb $\Z^r \tensor G \cong G^{r}$, denn
      \begin{equation*}
        \Z^r \tensor G \cong (\Z \oplus \dotsb \oplus \Z) \tensor G
        \cong \Z \tensor G \oplus \dotsb \oplus \Z \tensor G
        \cong G \oplus \dotsm \oplus G = G^r
      \end{equation*}
    \item
      Ist $A$ eine \emph{Torsionsgruppe}, das heißt $\forall a \in A \exists n \in \N$ mit $n a = 0$, und $G$ ein Körper der Charakteristik $0$, so ist
      \begin{equation*}
        A \tensor G = 0,
      \end{equation*}
      denn für beliebiges $a \in A$ sei $n \in \N$ mit $n a = 0$, dann folgt, dass
      \begin{equation*}
        a \tensor 1 = a \tensor (n \cdot n^{-1}) = (na) \tensor n^{-1} = 0 \tensor n^{-1} = 0
      \end{equation*}
      und somit für alle $g \in G$
      \begin{equation*}
        a \tensor g = g \cdot (a \tensor 1) = g \cdot 0 = 0,
      \end{equation*}
      also $t = 0$ für alle $t \in A \tensor G$.
    \item
      Für endlich erzeugte abelsche Gruppe $A$, also
      \begin{equation*}
        A \cong \Tor(A) \oplus \Z^b
      \end{equation*}
      mit $b = \rg(A)$, annulliert also das Tensorprodukt mit $\Q$ den Torsionsanteil,
      \begin{equation*}
        A \tensor \Q \cong \underbrace{\Tor(A) \tensor \Q}_{= 0} \oplus \underbrace{\Z^b \tensor \Q}_{= \Q^b} \cong \Q^b
      \end{equation*}
      und man erhält: $\rg(A) = \rg(A \tensor \Q)$.
    \item
      Für $m, n \in \N$ ist
      \begin{equation*}
        \Z^m \tensor \Z^n \cong \Z^{mn}
      \end{equation*}
      (Übung, ist $(e_i)$ Basis von $\Z^m$ und $(e_j)$ Basis von $\Z^n$, so ist $(e_i \tensor e_j)$ Basis von $\Z^m \tensor \Z^n$).
    \item
      Für $m,n \in \N$ ist
      \begin{equation*}
        \Z_m \tensor \Z_n \cong \Z_{\mathrm{ggT}(m,n)}
      \end{equation*}
      (Übung).
      Das ermöglicht also die Berechnung von $A \tensor B$ für endlich erzeugte abelsche Gruppen $A$ und $B$.
  \end{enumerate}
\end{beispiel}

\begin{bemerkung}
  Sei $G$ abelsche Gruppe.
  \begin{enumerate}
    \item
      Ist
      \begin{equation*}
        \begin{tikzcd}
          A \arrow{r}{\alpha}
          & B \arrow{r}{\beta}
          & C \arrow{r}{}
          & 0
        \end{tikzcd}
      \end{equation*}
      eine exakte Sequenz abelscher Gruppen, so ist auch
      \begin{equation*}
        \begin{tikzcd}
          A \tensor G \arrow{r}{\alpha \tensor \id}
          & B \tensor G \arrow{r}{\beta \tensor \id}
          & C \tensor G \arrow{r}{}
          & 0
        \end{tikzcd}
      \end{equation*}
      exakt.
    \item
      Ist
      \begin{equation*}
        \begin{tikzcd}
          0 \arrow{r}{}
          & A \arrow{r}{\alpha}
          & B \arrow{r}{\beta}
              \arrow[bend right,swap]{l}{l}
          & C \arrow{r}{}
          & 0
        \end{tikzcd}
      \end{equation*}
      exakt und spaltet, so ist auch
      \begin{equation*}
        \label{eqn:tensor_spaltung}
        \tag{$\ast\ast$}
        \begin{tikzcd}
          0 \arrow{r}{}
          & A \tensor G \arrow{r}{\alpha \tensor \id}
          & B \tensor G \arrow{r}{\beta \tensor \id}
                        \arrow[bend left]{l}{l \tensor \id}
          & C \tensor G \arrow{r}{}
          & 0
        \end{tikzcd}
      \end{equation*}
      exakt und spaltet.
  \end{enumerate}
\end{bemerkung}
\begin{proof}
  \begin{enumerate}
    \item
      \begin{enumerate}
        \item
          Sei $c \in C$ und $g \in G$ beliebig.
          Dann existiert $b \in B$ mit $\beta(b) = c$.
          Somit ist $(\beta \tensor \id) (b \tensor g) = \beta(b) \tensor \id(g) = c \tensor g$.
          Da $C \tensor G$ von $\left\{ c \tensor g : c \in C, g \in G \right\}$ erzeugt ist, folgt: $\beta \tensor \id$ ist surjektiv.
        \item
          \emph{Exaktheit bei $B \tensor G$}:
          \begin{enumerate}[($\alpha$)]
            \item
              Wegen
              \begin{equation*}
                (\beta \tensor \id) \circ (\alpha \tensor \id) = \underbrace{(\beta \circ \alpha)}_{= 0} \tensor \id = 0 \tensor \id = 0
              \end{equation*}
              ist
              \begin{equation*}
                \im (\alpha \tensor \id) \subseteq \ker (\beta \tensor \id)
              \end{equation*}
            \item
              \begin{equation*}
                \begin{tikzcd}
                  A \tensor G
                    \arrow{r}{\alpha \tensor \id}
                  & B \tensor G
                    \arrow{r}{\beta \tensor \id}
                    \arrow{d}{\pi}
                  & C \tensor G
                    \arrow{r}{}
                    \arrow[dashed]{dl}{\varphi}
                  & 0\\
                  & B \tensor G / U
                \end{tikzcd}
              \end{equation*}
              mit $U \coloneqq \im(\alpha \tensor \id)$.

              Betrachte $\tilde \varphi \colon C \times G \to B \tensor G / U$ mit $\tilde \varphi (c,g) \coloneqq [ b \tensor g ]$ und $b \in \beta^{-1}(c)$.
              Wegen der Surjektivität von $\beta$ existiert zunächst ein solches $b$, $\beta^{-1}(c) \neq \emptyset$, und $\tilde \varphi(c,g)$ hängt nicht von der Auswahl ab, denn ist $b' \in \beta^{-1}(c)$ ein weiteres Urbild, $\beta(b) = \beta(b') = c$, so ist $\beta(b' - b) = \beta(b') - \beta(b) = c - c = 0$ und damit existiert ein $a \in A$ mit $\alpha(a) = b' - b$, daher
              \begin{equation*}
                b' \tensor g - b \tensor g = (b' - b) \tensor g = \alpha(a) \tensor g = (\alpha \tensor \id)(a, g) \in U,
              \end{equation*}
              also $[b' \tensor g] = [b \tensor g]$.
              Aus diesem Grund ist $\tilde \varphi$ dann auch ein Homomorphismus.
              \todo[]{Ab hier Fortsetzung am 2017{-}11{-}24}

              $\tilde \varphi$ ist auch bilinear: $c_1,c_2 \in C$, $g \in G$, $b_1 \in \beta^{-1}(c_1)$, $b_2 \in \beta^{-1}(c_2)$.
              Dann ist $b_1 + b_2 \in \beta^{-1}(c_1 + c_2)$.
              Somit also
              \begin{equation*}
                \tilde \varphi(c_1+c_2, g) = [ (b_1 + b_2) \tensor g ] = [ b_1 \tensor g + b_2 \tensor g ]= [b_1 \tensor g] + [b_2 \tensor g] = \tilde \varphi (b_1,g) + \tilde \varphi (b_2, g).
              \end{equation*}
              Außerdem $\tilde \varphi(c, g_1 + g_2) = [ b \tensor (g_1 + g_2) ] = \dotsb = \tilde \varphi(c,g_1) + \tilde \varphi(c,g_2)$.

              Daher existiert Homomorphismus $\varphi \colon C \tensor G \to B \tensor G / U$ mit $\varphi(c \tensor g) = [b \tensor g]$ und $b \in \beta^{-1}(c)$.
              Somit also
              \begin{equation*}
                \varphi \circ (\beta \tensor \id) (b \tensor g) = \varphi (\beta(b) \tensor g) = [b \tensor g] = ¸pi (b \tensor g),
              \end{equation*}
              und damit $\varphi \circ (\beta \tensor \id) = \pi$.

              Sei nun $t \in \ker (\beta \tensor \id)$, dann ist
              \begin{equation*}
                0 = \varphi(0) = \varphi \circ (\beta \tensor \id) (t) = \pi(t),
              \end{equation*}
              also $t \in \ker(\pi) = U = \im(\alpha \tensor \id)$.
              Somit $\ker(\beta \tensor \id) \subseteq \im (\alpha \tensor \id)$.
          \end{enumerate}
      \end{enumerate}
    \item
      Ist $l \colon B \to A$ ein Linksinverses von $\alpha$, also $l \circ \alpha = \id$, so gilt
      \begin{equation*}
        (l \tensor \id) \circ (\alpha \tensor \id) = (l \circ \alpha) \tensor \id = \id \tensor \id = \id,
      \end{equation*}
      somit ist $\alpha \tensor \id$ ist injektiv, also ist~\eqref{eqn:tensor_spaltung} auch exakt und $l \tensor \id$ eine Spaltung dafür.
  \end{enumerate}
\end{proof}


\begin{kommentar}
  \begin{enumerate}
    \item
      Man sagt deshalb, dass $F = - \tensor G$ \emph{rechtsexakt} ist, das heißt exakte Sequenzen der Form $
      \begin{tikzcd}
      \cdot \arrow{r}{}
      & \cdot \arrow{r}{}
      & \cdot \arrow{r}{}
      & 0
      \end{tikzcd}
      $
      in wieder solche überführt.
    \item
      Er ist nicht \emph{exakt}, denn zum Beispiel ist
      \begin{equation*}
        \begin{tikzcd}
          0 \arrow{r}{}
          & \Z \arrow{r}{i}
          & \Q \arrow{r}{\pi}
          & \Q /\Z \arrow{r}{}
          & 0
        \end{tikzcd}
      \end{equation*}
      so ist die mit $\Z_2$ tensorierte Sequenz
      \begin{equation*}
        \begin{tikzcd}
          0 \arrow{r}{}
          & \underbrace{\Z \tensor \Z_2}_{=\Z_2} \arrow{r}{i \tensor \id}
          & \underbrace{\Q \tensor \Z_2}_{=0} \arrow{r}{\pi \tensor \id}
          & \Q /\Z \tensor \Z_2 \arrow{r}{}
          & 0
        \end{tikzcd}
      \end{equation*}
      nicht exakt.
    \item
      Das im Folgenden konstruierte Tensorprodukt ist ein Maß für die Abweichung der Injektivität in dem Sinne, das für eine kurze Sequenz (und abelsche Gruppe $G$)
      \begin{equation*}
        \begin{tikzcd}
          0 \arrow{r}{}
          & A \arrow{r}{\alpha}
          & B \arrow{r}{\beta}
          & C \arrow{r}{}
          & 0
        \end{tikzcd}
      \end{equation*}
      folgende Sequenz induziert wird, die exakt ist:
      \begin{equation*}
        \begin{tikzcd}
          0 \arrow{r}{}
          & \Tor(C,G) \arrow{r}{\Phi}
          & A \tensor G \arrow{r}{\alpha \tensor \id}
          & B \tensor G \arrow{r}{\beta \tensor \id}
          & C \tensor G \arrow{r}{}
          & 0
        \end{tikzcd}
      \end{equation*}
  \end{enumerate}
\end{kommentar}

\begin{defn}
  Seien $A$ und $G$ abelsche Gruppen und
  \begin{equation*}
    S(A):
    \begin{tikzcd}
      0 \arrow{r}{}
      & R \arrow{r}{i}
      & F(A) \arrow{r}{\pi}
      & A \arrow{r}{}
      & 0
    \end{tikzcd}
  \end{equation*}
  die Standard-Auflösung von $A$.
  Dann nennt man
  \begin{equation*}
    \Tor(A,G) \coloneqq \ker(j \tensor \id)
  \end{equation*}
  das \emph{Torsionsprodukt von $A$ mit $G$}.
  Es gilt dann mit sehr ähnlichem Beweis wie bei $\Ext(A,G)$:
\end{defn}

\begin{proposition}
  Seien $A$ und $A'$ abelsche Gruppen und $h \colon A \to A'$ ein Homomorphismus.
  Seien weiter
  \begin{equation*}
    \begin{tikzcd}
      S: 0 \arrow{r}{}
      & R \arrow{r}{}
      & F \arrow{r}{}
      & A \arrow{r}{}
      & 0\\
      S': 0 \arrow{r}{}
      & R' \arrow{r}{}
      & F' \arrow{r}{}
      & A' \arrow{r}{}
      & 0
    \end{tikzcd}
  \end{equation*}
  freie Auflösungen von $A$ und $A'$.
  Dann existiert eindeutig bestimmter Homomorphismus
  \begin{equation*}
    \Phi(h; S,S') \colon \ker(i \tensor \id) \to \ker(i' \tensor \id),
  \end{equation*}
  sodass für alle Fortsetzungen $(f,g)$ von $h$ gilt:
  Diagramm~\eqref{eqn:foo} kommutiert
  \begin{equation*}
    \label{eqn:bar}
    \tag{$\ast$}
    \begin{tikzcd}
      S: 0 \arrow{r}{}
      & R \arrow{r}{i}
          \arrow{d}{f}
      & F \arrow{r}{}
          \arrow{d}{g}
      & A \arrow{r}{}
          \arrow{d}{h}
      & 0\\
      S': 0 \arrow{r}{}
      & R' \arrow{r}{i'}
      & F' \arrow{r}{}
      & A' \arrow{r}{}
      & 0
    \end{tikzcd}
  \end{equation*}
  \begin{equation*}
    \label{eqn:foo}
    \tag{$\ast\ast$}
    \begin{tikzcd}
      0 \arrow{r}{}
      & \ker(i\tensor \id) \arrow{r}{\incl}
          \arrow{d}{\Phi}
      & R \tensor G\arrow{r}{i \tensor \id}
          \arrow{d}{f \tensor \id}
      & F \tensor G \arrow{r}{}
          \arrow{d}{g \tensor \id}
      & A \tensor G \arrow{r}{}
          \arrow{d}{h \tensor \id}
      & 0\\
      0 \arrow{r}{}
      & \ker(i' \tensor \id) \arrow{r}{\incl'}
      & R' \tensor G\arrow{r}{}
      & F' \tensor G \arrow{r}{}
      & A' \tensor G \arrow{r}{}
      & 0\\
    \end{tikzcd}
  \end{equation*}
\end{proposition}

\begin{kommentar}
  Die Existenz von $\Phi$ ist eigentlich nicht erstaunlich, da man natürlich
  \begin{equation*}
    \Phi \coloneqq {(\incl')}^{-1} \circ (f \tensor \id) \circ \incl
  \end{equation*}
  setzt (wenn man gecheckt hat, dass $\im(f \tensor \id \circ \incl) \subseteq  \ker(i' \tensor \id)$ liegt).
  Erstaunlich ist vielmehr, dass $\Phi$ nicht von der Auwahl der Fortsetzung $(f,g)$ von $h$ abhängt (vgl.\ Diskussion bei $\Ext$).
\end{kommentar}

\begin{kommentar}
  \begin{enumerate}[(a)]
      Die Zuordnung $(h;S,S') \mapsto \Phi(h; S, S')$ ist funktoriell in dem Sinne, dass gilt:
      \begin{enumerate}[(i)]
        \item
          $\Phi(\id,S,S) = \id$
        \item
          $\Phi(h' \circ h; S, S'') = \Phi(h'; S, S') \circ \Phi(h; S', S'')$
      \end{enumerate}
    \item
      Ist $h \colon A \to A'$ insbesondere ein Isomorphismus (z.B.\ $h = \id$), so muss für zwei freie Auflösungen $S$ und $S'$
      \begin{equation*}
        \ker(i \tensor \id) \underset{\Phi(h;S,S')}{\cong} \ker(i' \tensor \id),
      \end{equation*}
      denn $\Phi(h^{-1}; S', S) = {\Phi(h; S, S')}^{-1}$.
    \item
      Noch speziell folgt, dass für eine beliebige freie Auflösung
      \begin{equation*}
        \begin{tikzcd}
          0 \arrow{r}{}
          & R \arrow{r}{i}
          & F \arrow{r}{\pi}
          & A \arrow{r}{}
          & 0
        \end{tikzcd}
      \end{equation*}
      $\ker(i \tensor \id)$ (kanonisch) isomorph zu $\Tor(A,G)$ (vermöge $\Phi(\id, S, S(A))$) ist.
  \end{enumerate}
\end{kommentar}

\begin{beispiel}
  \begin{enumerate}
    \item
      Ist $A$ frei abelsch, so ist deshalb
      \begin{equation*}
        \Tor(A,G) = \triv
      \end{equation*}
      für alle abelschen Gruppen $G$, denn
      \begin{equation*}
        \begin{tikzcd}
          0 \arrow{r}{}
          & 0 \arrow{r}{}
          & A \arrow{r}{\id}
          & A \arrow{r}{}
          & 0
        \end{tikzcd}
      \end{equation*}
      ist freie Auflösung und $0 \tensor \id = 0$.
    \item
      Ist $A = A_1 \oplus A_2$ und $G$ beliebig, so gilt
      \begin{equation*}
        \Tor(A_1 \oplus A_2,G) \cong \Tor(A_1, G) \oplus \Tor(A_2, G),
      \end{equation*}
      denn sind
      \begin{equation*}
        \begin{tikzcd}
          S_1\colon 0 \arrow{r}{}
          & R_1 \arrow{r}{}
          & F_1 \arrow{r}{}
          & A_1 \arrow{r}{}
          & 0
        \end{tikzcd}
      \end{equation*}
      und
      \begin{equation*}
        \begin{tikzcd}
          S_2\colon 0 \arrow{r}{}
          & R_2 \arrow{r}{}
          & F_2 \arrow{r}{}
          & A_2 \arrow{r}{}
          & 0
        \end{tikzcd}
      \end{equation*}
      zwei freie Auflösungen, so auch
      \begin{equation*}
        \begin{tikzcd}
          S_1 \oplus S_2 \colon 0 \arrow{r}{}
          & R_1 \oplus R_2 \arrow{r}{j_1 \oplus j_2}
          & F_1 \oplus F_2 \arrow{r}{}
          & A_1 \oplus A_2 \arrow{r}{}
          & 0
        \end{tikzcd}
      \end{equation*}
      und daher:
      \begin{equation*}
        \Tor(A_1 \oplus A_2, G) \cong \ker(j_1 \oplus j_2,\tensor \id) \cong \ker(j_1 \tensor \id) \oplus \ker(j_2 \tensor \id) \cong \Tor(A_1, G) \oplus \Tor(A_2, G)
      \end{equation*}
    \item
      Für jedes $n \in \N$ ist
      \begin{equation*}
        \Tor(\Z_n,G) \cong \left\{ g \in G : n g = 0 \right\},
      \end{equation*}
      denn
      \begin{equation*}
        \begin{tikzcd}
          0 \arrow{r}{}
          & \Z \arrow{r}{\cdot n \eqqcolon \mult[n]}
          & \Z \arrow{r}{\pi}
          & \Z_n \arrow{r}{}
          & 0
        \end{tikzcd}
      \end{equation*}
      ist freie Auflösung und daher ist
      \begin{equation*}
        \Tor(\Z_n, G) \cong \ker(\mult[n] \tensor \id)
      \end{equation*}
      und wegen
      \begin{cd}
        0 \arrow{r}{}
        & \Tor(\Z_n, G) \arrow{r}{}
          \arrow{d}{\cong}
        & \Z \tensor G \arrow{r}{\mult[n] \tensor \id}
          \arrow{d}{\cong,(k,g) \mapsto kg}
        & \Z \tensor G \arrow{r}{}
          \arrow{d}{\cong}
        & \Z_n \tensor G \arrow{r}{}
          \arrow{d}{\cong}
        & 0\\
        0 \arrow{r}{}
        & \ker(\mult[n]) \arrow{r}{\overline {\incl}}
        & G \arrow{r}{\mult[n]}
        & G \arrow[r, "\overline \pi", "(k{,}g) \mapsto kg"']
        & \Z_n \tensor G \arrow{r}{}
        & 0
      \end{cd}
      ist tatsächlich $\Tor(\Z_n,G) = \ker(\mult[n])$.
    \item
      Ist $A$ endlich erzeugt und $G$ Körper der Charakteristik Null (also $A \cong \Tor(A) \oplus \Z^r$, mit $r = \rg(A) \in \N_0$), so ist wegen\ (a)-(c):
      \begin{equation*}
        \Tor(A,G) = \triv
      \end{equation*}
    \item
      Für alle $m,n \in \Z$ ist schließlich (Übung):
      \begin{equation*}
        \Tor(\Z_m,\Z_n) \cong \Z_{\mathrm{ggT}(m,n)}.
      \end{equation*}
  \end{enumerate}
\end{beispiel}

\begin{kommentar}
  Sei $G$ feste abelsche Gruppe.
  Dann wird $F = \Tor(-,G) \colon \Ab \to \Ab$ mit
  \begin{align*}
    F(A) & = \Tor(A,G) \\
    F(h) & = \Phi(h; S(A), S(B)) & \text{für $h \colon A \to B$ Homomorphismus}
  \end{align*}
  zu einem covarianten Funktor von \Ab\ auf sich selbst.
\end{kommentar}

\begin{defn}
  Sei $C = (C_k, \del_k)$ ein Kettenkomplex und $G$ eine abelsche Gruppe.
  Dann nennen wir
  \begin{equation*}
    C \tensor G \coloneqq (C_k \tensor G, \del_k \tensor \id_G)
  \end{equation*}
  den \emph{zugehörigen Kettenkomplex mit Koeffizienten in $G$}.
\end{defn}

\begin{kommentar}
  \begin{enumerate}
    \item
      $C \tensor G$ ist tatsächlich ein Kettenkomplex, denn
      \begin{equation*}
        (\del_k \tensor \id) \circ (\del_{k-1} \tensor \id) = \underbrace{\del_k \circ \del_{k-1}}_{= 0} \tensor \id = 0
      \end{equation*}
  \end{enumerate}
\end{kommentar}



\begin{kommentar}
  \begin{enumerate}
    \item
      Ist beispielsweise $C = S(X)$ der singuläre Kettenkomplex eines topologischen Raumes, $S_k(X) = \mathbb{F} (\Sigma_k(X))$, dann folgt
      \begin{equation*}
        S_k(X) \tensor G = \bigoplus_{\Sigma_k(X)} \Z \tensor G \cong \bigoplus_{\Sigma_k(X)} (\underbrace{\Z \tensor G}_{=G}) = \bigoplus_{\Sigma_k(X)} G,
      \end{equation*}
      das heißt jedes Element $\tau \in S_k(X) \tensor G$ hat eindeutige Darstellung
      \begin{equation*}
        \overline c = g_1 \sigma_1 + \dotsb + g_r \sigma_r,
      \end{equation*}
      wobei $g \sigma \coloneqq g \tensor \sigma$ abkürzt.
    \item
      \emph{Warnung}: Für
      \begin{equation*}
        \overline c = g_1 \sigma_1 + \dotsb + g_r \sigma_r
      \end{equation*}
      ist also
      \begin{equation*}
        \del \overline c = g_1 \del \sigma_1 + \dotsb + g_r \del \sigma_r
      \end{equation*}
      mit
      \begin{equation*}
        \del \sigma = \sum_{i = 0}^k {(-1)}^{i} \sigma \circ \delta_k^i,
      \end{equation*}
      die Elemente $\sigma$ und $\del \sigma$ liegen aber nicht in $S_k(X) \tensor G$, da $G$ im Allgemeinen kein Einselement hat.
    \item
      Die Homologie von $C \tensor G$ notieren wir so:
      \begin{equation*}
        H(C;G) \coloneqq (H_k(C \tensor G))
      \end{equation*}
      und nenne sie die \emph{Homologie von $C$ mit Koeffizienten in $G$}.
  \end{enumerate}
\end{kommentar}

\begin{defn}
  Seien $C$ und $C'$ Kettenkomplexe und $f \colon C \to C'$ eine Kettenabbildung.
  Ist $G$ abelsche Gruppe, so nennt man $f \tensor \id = (f_k \tensor id)$, $f_k \tensor \id \colon C_k \tensor G \to C_k' \tensor G$ die \emph{von $f$ induzierte Kettenabbildung}.
\end{defn}

\begin{kommentar}
  \begin{enumerate}
    \item
      $f$ ist tatsächlich Kettenabbildung, denn:
      \begin{equation*}
        (\del_k' \tensor \id) \circ (f_k \tensor \id) = (\del_k' \circ f_k) \tensor \id = (f_{k1} \circ \del_k) \tensor \id = (f_{k-1} \tensor \id) \circ (\del_k \tensor \id)
      \end{equation*}
    \item
      Das Tensorieren mit $G$ ergibt also einen Funktor
      \begin{equation*}
        F = - \tensor G \colon \KK \to \KK.
      \end{equation*}
  \end{enumerate}
\end{kommentar}

\emph{Frage:} Wie hängen nun $H(C;G)$ und $H(C) = H(C;\Z)$ zusammen?

\begin{vorbereitung}
  Sei $C$ Kettenkomplex, $G$ abelsche Gruppe und $k \in \N_0$.
  Dann induziert das Tensorprodukt $C_k \times G$ ein bilineares
  \begin{align*}
    Z_k \times G &  \xrightarrow{\tensor} Z_k (C \tensor G) \\
    (z,g) & \mapsto z \tensor g,
  \end{align*}
  denn
  \begin{equation*}
    (\del_k \tensor \id) (z \tensor g) = \underbrace{\del_k z}_{=0} \tensor g = 0
  \end{equation*}
  und überführt dabei $B_k \times G$ nach $B_k(C \tensor G)$, denn ist $z = \del w$ ($w \in C_{k+1}$), so ist
  \begin{equation*}
    z \tensor g = \del w \tensor g = \del \tensor \id (w \tensor g)
  \end{equation*}
  Deshalt induziert $\tensor$ ein (eindeutig bestimmtes)
  \begin{align*}
    \Lambda \colon H_k \times G & \to H_k(C;G) \\
    \Lambda([z], g) & = [ z \tensor g]
  \end{align*}
  Da $\Lambda$ offenbar bilinear ist, existiert ein eindeutig bestimmter Homomorphismus
  \begin{align*}
    \lambda \colon H_k \tensor G & \to H_k(C;G) \\
    \intertext{mit}
    \lambda([z] \tensor g) & = [ z \tensor g ]
  \end{align*}
  für alle $z \in Z_k, g \in G$.
  \begin{cd*}
    Z_k \times G \arrow[r, "\tensor"]
      \ar[d, "\pi_k \times \id"]
    & Z_k (C \tensor G)
      \ar[d, "\pi_k"]
    \\
    H_k \times G \ar[r, "\Lambda"]
    & H_k(C;G)
  \end{cd*}
\end{vorbereitung}

\emph{Ab jetzt}: $C$ sei frei.

\begin{lemma}
\label{lem:quux}
  Es gilt mit $\del_k' \colon C_k \to B_{k-1}$ und $i_k \colon B_k \xhookrightarrow{} Z_k$, $j_k \colon Z_k \xhookrightarrow{} C_k$ (also $\del_k = j_{k-1} \circ i_{k-1} \circ \del_k'$):
  \begin{enumerate}
    \item
      \begin{equation*}
        \del_k' \tensor \id C_k \tensor B_{k-1} \tensor \id
      \end{equation*}
      ist surjektiv und
      \begin{equation*}
        \overline{Z_k} \coloneqq Z_k(C \tensor G) = {(\del_k' \tensor \id)}^{-1} (\ker (i_{k-1} \tensor \id))
      \end{equation*}
    \item
      \begin{equation*}
        \del_k' \tensor \id (\overline{B_k}) = 0,
      \end{equation*}
      wobei $\overline{B_k} \coloneqq B_k(C \tensor G)$.
  \end{enumerate}
\end{lemma}

\begin{kommentar}
  \begin{enumerate}
    \item
      Deshalb induziert nun $\del_k' \tensor \id$ einen Homomorphismus
      \begin{align*}
        h_k \colon H_k(C;G) = \overline{Z_k} / \overline{B_k} & \to \ker (i_{k-1} \tensor \id) \\
        [ \overline z ] & \mapsto \del_k' \tensor \id (\overline z).
      \end{align*}
    \item
      Betrachte nun die freie Auflösung
      \begin{cd*}
        S \colon 0 \ar[r]
        & B_{k-1} \ar[r, "i_{k-1}"]
        & Z_{k-1} \ar[r, "\pi_k"]
        & H_{k-1}(C) \ar[r]
        & 0,
      \end{cd*}
      da mit $C_{k-1}$ auch $Z_{k-1}$ und $B_{k-1}$ frei sind.
      Sei dann
      \begin{equation*}
        \Phi = \Phi(\id; S, S(H_{k-1}(C))) \colon \ker(i_{k-1} \tensor \id) \to \Tor(H_{k-1}(C), G)
      \end{equation*}
      der induzierte Isomorphismus.
      Setze schließlich
      \begin{equation*}
        \mu \coloneqq \Phi \circ h \colon H_k(C;G) \to \Tor(H_{k-1}(C), G)
      \end{equation*}
  \end{enumerate}
\end{kommentar}

\begin{proof}[Beweis von Lemma~\ref{lem:quux}]
  \begin{enumerate}
    \item
      \begin{equation*}
        \del_k' \tensor \id \colon C_k \tensor G \to B_{k-1} \tensor G
      \end{equation*}
      surjektiv, $\overline{Z_k}= {(\del_k' \tensor \id)}^{-1} (\ker (i_{k-1} \tensor \id))$.
      Dazu: Da $\del_k'$ surjektiv ist, ist es auch $\del_k' \tensor \id$:
      Ist $z \in B_{k-1}$, $g \in G$, also $z = \del w$ für ein $w \in C_k$, so ist $z \tensor g = \del \tensor \id (w \tensor g)$.
      Da ($z \tensor g : z \in B_{k-1}, g \in G$) $B_{k-1} \tensor G$ erzeugt, folgt, dass $\del_k' \tensor \id$ surjektiv ist.

      Erinnere:
      \begin{equation*}
        \del_k = j_{k-1} \circ i_{k-1} \circ \del_k'.
      \end{equation*}
      Dann gilt
      \begin{equation*}
        \del_k \tensor \id = (j_{k-1} \tensor \id) \circ (i_{k-1} \tensor \id) \circ (\del_{k}' \tensor \id).
      \end{equation*}
      Weil nun die folgende Sequenz spaltet
      \begin{cd*}
        0 \ar[r]
        & Z_{k-1} \ar[r, "j_{k-1}"]
        & C_{k-1} \ar[r, "\del_{k-1}'"]
          \ar[l, bend left, "l_{k-1}"]
        & B_{k-2} \ar[r]
        & 0
      \end{cd*}
      ist auch
      \begin{cd*}
        0 \ar[r]
        & Z_{k-1} \tensor G \ar[r, "j_{k-1} \tensor \id"]
        & C_{k-1} \tensor G \ar[r]
        & B_{k-2} \tensor G \ar[r]
        & 0
      \end{cd*}
      exakt, insbesondere ist $j_{k-1} \tensor \id$ injektiv.
      Somit gilt
      \begin{align*}
        \overline{Z_k}
        & = Z_k(C \tensor G) \\
        & = \ker(\del_k \tensor \id) \\
        & \underset{j_k \tensor \id \text{ injektiv}}{=} \ker( (i_{k-1} \tensor \id) \circ (\del_k' \tensor \id)) \\
        & = {(\del_k' \tensor \id)}^{-1} (\ker (i_{k-1} \tensor \id))
      \end{align*}
    \item
      $\del_k' \tensor \id (\overline{B_k}) = 0$.

      Dazu:
      Sei $\overline z \in \im(\del_{k+1} \tensor \id) = \overline{B_k}$ und ohne Einschränkung $\overline z = (\del_{k+1} \tensor \id) (c \tensor g)$ mit $c \in C_{k+1}, g \in G$ (jedes andere Element ist Summe von solchen).
      Dann ist
      \begin{equation*}
        \del_k' \tensor \id (\overline z) = (\del_k' \tensor \id) \circ (\del_{k+1} \tensor \id) (c \tensor g) = \underbrace{\del_k' \circ \del_{k+1} (c)}_{= 0} {} \tensor g = 0
      \end{equation*}
  \end{enumerate}
\end{proof}

\begin{satz}[Universelles Koeffiziententheorem für Kettenkomplexe]
  Sei $C$ freier Kettenkomplex und $k \in \Z$.
  Dann ist mit $\lambda$ und $\mu$ wie oben beschrieben die folgende Sequenz exakt und spaltet:
  \begin{cd*}
    0 \ar[r]
    & H_k(C) \tensor G \ar[r, "\lambda"]
    & H_k(C;G) \ar[r, "\mu"]
    & \Tor(H_{k-1}(C),G) \ar[r]
    & 0
  \end{cd*}
\end{satz}

\begin{proof}
  \begin{enumerate}
    \item
      EXaktheit bei $\Tor(H_k(C), G)$:
      Nach dem Lemma ist $\del_k' \tensor \id \colon \overline{Z_k} \to \ker (i_{k-1} \tensor \id)$ surjektiv und dann auch $h \colon \overline{H_k} \coloneqq H_k(C;G) \to \ker(i_{k-1} \tensor \id)$ und damit auch $\mu = \Phi \circ h \colon \overline{H_k} \to \Tor(H_{k-1}, G)$, da $\Phi$ surjektiv ist.
    \item
      Bei $H_k(C;G)$:
      \begin{enumerate}
        \item
          $\im(\lambda) \subseteq \ker(\mu)$:
          Für $z \in Z_k, g \in G$ ist:
          \begin{align*}
            \mu \circ \lambda ([z] \tensor g) & = \Phi \circ h ( [z \tensor g]) & \text{da $\lambda([z] \tensor g) = [z \tensor g]$} \\
            & = \Phi(\del_k' \tensor \id (z \tensor g)) & \text{da $h([z \tensor g]) = \del_k' z \tensor g \in \ker(i_{k-1} \tensor \id)$} \\
            & = \Phi(\underbrace{\del_k' z}_{=0} {} \tensor g) = 0
          \end{align*}
          Also ist $\mu \circ \lambda = 0$.
        \item
          $\ker \mu \subseteq \im \lambda$:
          Sei $[\overline z] \in \overline{H_k}$ (mit $\overline{z_k} \in \overline{Z_k}$) mit $\mu([\overline z]) = 0$, also
          \begin{equation*}
            0 = \del_k' \tensor \id (\overline z)
          \end{equation*}
          (da $\Phi$ injektiv ist).
          Da
          \begin{cd*}
            0 \ar[r]
            & Z_k \tensor G \ar[r, "j_k \tensor \id"]
            & C_k \tensor G \ar[r, "\del_k' \tensor \id"]
            & B_k \tensor G \ar[r]
            & 0
          \end{cd*}
          bei $C_k \tensor G$ exakt ist, existiert ein $\overline c \in Z_k \tensor G$ mit:
          \begin{equation*}
            j_k \tensor \id (\overline c) = \overline z.
          \end{equation*}
          Es gibt also $r \in \N_0$, $g_1, \dotsc, g_r \in G$ und $z_1, \dotsc, z_r \in Z_k$ mit
          \begin{equation*}
            \overline z = g_1 z_1 + \dotsb + g_r z_r.
          \end{equation*}
          Dann folgt
          \begin{equation*}
            [\overline z] = [ z_1 \tensor g_1 ] + \dotsm + [ z_r \tensor g_r ] = \lambda( [z_1] \tensor g_1) + \dotsm + \lambda([z_r] \tensor g_r) = \lambda([z_1] \tensor g_1 + \dotsm + [z_r] \tensor g_r) \in \im \lambda.
          \end{equation*}
      \end{enumerate}
    \item
      Bei $H_k(C) \tensor G$:
      Da $B_{k-1}$ frei ist, spaltet die Sequenz
      \begin{cd*}
        0 \ar[r]
        & Z_k \ar[r, "j_k"]
        & C_k \ar[r, "\del_k'"]
          \ar[l, bend left, "l_k"]
        & B_{k-1} \ar[r]
        & 0
      \end{cd*}
      Sei $l_k \colon C_k \to Z_k$ eine Spaltung,
      \begin{equation*}
        l_k \circ j_k = \id.
      \end{equation*}
      Betrachte dann die Komposition
      \begin{cd*}
        \label{seq:komposition}
        \tag{$\ast$}
        \overline{Z_k} \subseteq C_k \tensor G \ar[r, "l_k \tensor \id"]
        & Z_k \tensor G \ar[r, "\pi_k"]
        & H_k \tensor G.
      \end{cd*}
      Da jedes Element in $\overline{B_k}$ Summe von Elementen der Form
      \begin{equation*}
        \del c \tensor g = (\del_{k+1} \tensor \id) (c \tensor g)
      \end{equation*}
      ist, ist
      \begin{equation*}
        (\pi_k \tensor \id) \circ (l_k \tensor \id) (\del_k c \tensor g)
        = \pi_k \circ l_k \circ \underbrace{\del_k (c)}_{= j_k (\del_k'(c))} {} \tensor g
        \underset{l_k \circ j_k = \id}{=} \underbrace{\pi \circ \del_k'}_{= 0} (c) \tensor g = 0.
      \end{equation*}
      Deshalb induziert
      \begin{equation*}
        (\pi_k \tensor \id) \circ (l_k \tensor \id) |_{\overline{Z_k}}
      \end{equation*}
      ein Homomorphismus
      \begin{align*}
        \lambda' \colon \overline{H_k} & \to H_k \tensor G \\
        \intertext{mit}
        [\overline z] & \mapsto (\pi_k \tensor \id) \circ (l_k \tensor \id) (\overline z)
      \end{align*}
      also:
      \begin{equation*}
        \lambda' \circ \lambda ([z] \tensor g) = \lambda' ([z \tensor g]) = (\pi_k \tensor \id) \circ (l_k \tensor \id) (z \tensor g) = \pi_k \circ \underbrace{l_k (z)}_{= z} {} \tensor g = [z] \tensor g
      \end{equation*}
      \todo[]{ersetze $l_k$ durch $l_k \circ j_k$}
      und damit $\lambda' \circ \lambda = \id$.
      Somit ist $\lambda$ injektiv und~\eqref{seq:komposition} spaltet.
  \end{enumerate}
\end{proof}



\begin{kommentar}
  \begin{enumerate}
    \item
      $\lambda_C$ und $\mu_C$ sind natürliche Transformationen, allerdings spaltet die Sequenz nicht natürlich.
    \item
      Es folgt dann also insbesondere
      \begin{equation*}
        H_k(C;G) \cong H_k(C) \tensor G \oplus \Tor(H_{k-1}(C), G).
      \end{equation*}
    \item
      Hat man insbesondere ein Raumpaar $(X,A)$, so ist der zugehörige singuläre Kettenkomplex $S(X,A)$ frei, denn $S_k(X,A)$ wird von allen singulären $k$-Simplexen $\sigma \colon \Sigma_k \to X$ frei erzeugt, die $\sigma (\Sigma_k) \not\subseteq A$ erfüllen.
      Es gilt also hier das universelle Koeffiziententheorem:
      \begin{cd*}
        0 \ar[r]
        & H_k(X,A) \tensor G \ar[r, "\lambda"]
        & H_k(X,A;G) \ar[r, "\mu"]
        & \Tor(H_{k-1}(X,A), G) \ar[r]
        & 0
      \end{cd*}
  \end{enumerate}
\end{kommentar}

\begin{satz}
  Sei $G$ eine abelsche Gruppe und $H = {(H_k)}_{k \in \N}$ die obigen Funktoren $H_k = H_k (- - ; G) \colon \Top_2 \to \Ab$.
  Dann gibt es eine Folge $\del = {(\del_k)}_{k \in \N}$ von natürlichen Transformationen von $F_1 = H_k(-, - ; G)$ nach $F_2$ mit $F_2 (X,A) = H_{k-1}(A, \emptyset; G)$, sodass gilt:
  \begin{enumerate}
    \item
      Ist $f \homot g \colon (X,A) \to (Y,B)$, so ist
      \begin{equation*}
        f_* = g_* \colon H_k(X,A) \to H_k(Y,B)
      \end{equation*}
      \emph{(Homotopie-Axiom)}
    \item
      Sind für ein Raumpaar $(X,A)$ $i \colon A \xhookrightarrow{} X$ und $j \colon X = (X,\emptyset) \to (X,A)$ die Inklusionen, so ist die folgende Sequenz exakt \emph{(Exaktheitsaxiom)}:
      \begin{cd*}
        \tag{$\ast$}
        \label{seq:lange_sequenz}
        \dotsb \ar[r]
        & H_k(X,A;G)  \ar[r, "\del_k(X{,}A)"]
        & H_{k-1}(A;G) \ar[r, "i_*"]
        & H_{k-1}(X;G) \ar[r, "j_*"]
        & H_{k-1}(X,A;G)  \ar[r, "\del_{k-1} (X{,}A)"]
        & \dotsb
      \end{cd*}
    \item
      Ist $(X,A)$ ein Raumpaar und $U \subseteq X$ derart, dass $\overline{U} \subseteq \mathring A$, so induziert die Inklusion $i \colon (X \setminus U, A \setminus U) \xhookrightarrow{} (X,A)$ einen Isomorphismus in der Homologie,
      \begin{equation*}
        i_{*} \colon H_k(X \setminus U, A \setminus U) \xrightarrow{\cong} H_k(X,A)
      \end{equation*}
      \emph{(Ausschneidungsaxiom)}.
    \item
      Für den einpunktigen Raum $pt$ gilt:
      \begin{equation*}
        H_k(pt;G) =
        \begin{cases}
          G & \text{für $k = 0$} \\
          0 & \text{sonst}
        \end{cases}
      \end{equation*}
      \emph{(Normierungsaxiom)}.
  \end{enumerate}
\end{satz}
\begin{proof}
  \begin{enumerate}
    \item
      Sei $f,g \colon (X,A) \to (Y,B)$ homotop, dann sind $S f, S g \colon S(X,A) \to S(Y,B)$ kettenhomotop, sagen wir bezüglich einer Kettenhomotopie $D$.
      Dann gilt
      \begin{equation*}
        S f \tensor \id_G \underset{D \tensor \id}{\cong} S g \tensor \id_G \quad \text{(kettenhomotop)}
      \end{equation*}
      und somit
      \begin{equation*}
        f_* = g_* \colon H_k(X,A;G) \to H_k(Y,B;G).
      \end{equation*}
    \item
      Für jedes Raumpaar $(X,A)$ spaltet folgende Sequenz, weil $(X,A)$ frei ist:
      \begin{cd*}
        0 \ar[r]
        & S(A) \ar[r, "S i"]
        & S(X) \ar[r]
        & S(X,A) \ar[r]
          \ar[l,bend left]
        & 0
      \end{cd*}
      Also ist auch
      \begin{cd*}
        0 \ar[r]
        & S(A) \tensor G \ar[r, "S i \tensor G"]
        & S(X) \tensor G \ar[r]
        & S(X,A) \tensor G \ar[r]
          \ar[l,bend left]
        & 0
      \end{cd*}
      exakt (und spaltet).
      Sei $\del_k (X,A) \colon H_k(X,A;G) \to H_{k-1}(A;G)$ der verbindende Homomorphismus dieser kurzen exakten Sequenz.
      Dann ist $\del_k$ natürliche Transformation von $H_k(-, -; G)$ nach $H_{k-1}(pr_2(-); G)$ und die lange Sequenz~\eqref{seq:lange_sequenz} ist exakt.
    \item
      Nach dem universellen Koeffiziententheorem sind die Reihen des folgenden Diagramms exakt und das Diagramm kommutiert:
      \begin{equation*}
        \begin{cd}
          0 \ar[r]
          & H_k(X \setminus U, A \setminus U) \tensor G \ar[r, "\lambda_{X \setminus U, A \setminus U}"]
            \ar[d, "i_* \tensor \id_G", "\cong"']
          & H_k(X \setminus U, A \setminus U; G) \ar[r, "\mu_{X \setminus U, A \setminus U}"]
            \ar[d, "i_*"]
          & \Tor(H_{k-1}(X \setminus U, A \setminus U), G) \ar[r]
            \ar[d, "\Tor(i_*{,} G)", "\cong"']
          & 0 \\
          0 \ar[r]
          & H_k(X , A ) \tensor G \ar[r, "\lambda_{X , A }"]
          & H_k(X , A ; G) \ar[r, "\mu_{X , A }"]
          & \Tor(H_{k-1}(X , A ), G) \ar[r]
          & 0 \\
        \end{cd}
      \end{equation*}
      Da $i_* \colon H_k(X \setminus U, A \setminus U) \to H_k(X,A)$ nach dem (ganzzahligen) Ausschneidungssatz Isomorphismus ist, sind es auch $i_* \tensor \id$ und $\Tor(i_*, G)$.
      Nach dem Fünferlemma ist es dann auch
      \begin{equation*}
        i_*^G \colon H_k(X \setminus U, A \setminus U; G) \to H_k(X,A;G)
      \end{equation*}
    \item
      Nach dem universellen Koeffiziententheorem ist:
      \begin{equation*}
        H_k(pt; G) \cong H_k(pt) \tensor G \oplus \underbrace{\Tor(\underbrace{H_{k-1}(pt)}_{\text{frei}}, G)}_{= \triv} =
        \begin{cases}
          \overbrace{\Z \tensor G}^{\cong G} & \text{für $k = 0$} \\
          0 & \text{sonst}
        \end{cases}
      \end{equation*}
  \end{enumerate}
\end{proof}


\section{Homologie von Produkten}

\begin{motivation}
  \begin{enumerate}
    \item
      Seien $X$ und $Y$ topologische Räume.
      Wie sieht dann eigentlich die Homologie vom Produkt $X \times Y$ in Termen der Homologie von $X$ und von $Y$ aus?
    \item
      Innerhalb von Kettenkomplexen:
      Kann man für gegebene Kettenkomplexe $C$ und $C'$ eigentlich das Tensorprodukt $C \tensor C' = {({(C \tensor C')}_{k})}_{k \in \Z}$ mit
      \begin{equation*}
        {(C \tensor C')}_k \coloneqq \bigoplus_{p+q = k} C_p \tensor C'_q
      \end{equation*}
      und geeigneten Randabbildungen
      \begin{equation*}
        \overline{\del}_k \colon \overline{C}_k \to \overline{C}_{k-1}
      \end{equation*}
      mit $\overline{C_k} \coloneqq {(C \tensor C')}_k$ selbst zu einem Kettenkomplex machen und wie sieht die Homologie von $C \tensor C'$ in Termen der Homologie von $C$ und $C'$ aus?
    \item
      Wie hängen denn $S(X) \tensor S(Y)$ und $S(X \times Y)$ genau zusammen?
  \end{enumerate}
\end{motivation}

\begin{defn}
  Seien $C$ und $C'$ Kettenkomplexe.
  Man setzt dann
  \begin{equation*}
    \tag{$\ast$}
    \label{eqn:def_del_bar}
    \overline{\del}_k \colon \overline{C}_k \to \overline{C}_{k-1}, \qquad \overline{C}_k = \bigoplus C_p \tensor C'_q
  \end{equation*}
  folgendermaßen fest:
  \begin{equation*}
    \overline \del_k (c \tensor c') \coloneqq \del_p c \tensor c' + {(-1)}^p c \tensor \del_q' c'
  \end{equation*}
  mit $c \in C_p$, $c' \in C_q'$ und $p + q = k$.
\end{defn}

\begin{kommentar}
  \begin{enumerate}
    \item
      Universelle Eigenschaft der direkten Summe zeigt, dass es reicht, $\overline \del_k$ auf jedem Summanden $C_p \tensor C'_q$ anzugeben
    \item
      Universelle Eigenschaft des Tensorproduktes zeight dann, dass $\overline \del_k | _{C_p \tensor C'_q}$ wohldefiniert und eindeuti bestimmt ist durch~\eqref{eqn:def_del_bar}.
    \item
      $\overline C \coloneqq ({(C \tensor C')}_k, \overline \del_k)$ ist dann tatsächlich ein Kettenkomplex, denn
      \begin{align*}
        \overline \del_{k-1} \circ \overline \del_k (c \tensor c')
          & = \overline \del_{k-1} (\del c \tensor c' + {(-1)}^p c \tensor \del' c') \\
          & = \underbrace{\del^2 c}_{= 0} + {(-1)}^{p-1} \del c \tensor \del' c'
          + {(-1)}^{p} \del c \tensor \del' c' + {(-1)}^{2p} c \tensor \underbrace{{\del'}^2 c'}_{= 0} \\
          & = 0
      \end{align*}
      für $c \in C_p$, $c' \in C'_q$ mit $p + q = k$ und das reicht, weil $\overline C_k$ von solchen Elementen erzeugt ist,
      \begin{equation*}
        \overline \del_{k-1} \circ \overline \del_k = 0
      \end{equation*}
  \end{enumerate}
\end{kommentar}

\begin{vorbereitung}
  \begin{enumerate}
    \item
      Seien $C$ und $C'$ Kettenkomplexe, $\overline C = C \tensor C'$, $k \in \Z$ und $p, q \in \Z$ mit $p + q = k$.
      Die bilineare Abbildung
      \begin{align*}
        C_p \times C'_q & \to C_p \tensor C'_q \subseteq \overline C_k\\
        (c, c') & \mapsto c \tensor c'
      \end{align*}
      bildet $Z_p \times Z'_q$ nach $\overline Z_k$ ab, denn
      \begin{equation*}
        \overline \del_k (c \tensor c') = \underbrace{\del c}_{= 0} \tensor c' + {(-1)}^p c \tensor \underbrace{\del' c'}_{= 0} = 0
      \end{equation*}
      Die Komposition der Einschränkung auf $Z_p \times Z'_q$ mit der Projektion $\pi_k \colon \overline Z_k \to \overline H_k \coloneqq H_k(\overline C)$ schickt dann $B_p \times Z'_q$ und $Z_p \times B'_q$ nach Null, denn
      \begin{equation*}
        \overline \del_{k+1} (c \tensor z') = \del c \tensor z' + {(-1)}^{p+1} c \tensor \underbrace{\del' z'}_{= 0} = z \tensor z'
      \end{equation*}
      für $z \in B_p$, $z' \in Z'_q$, also $z = \del c$ für ein $c \in C_{p+1}$.
      Also
      \begin{equation*}
        \pi_k \circ \overline \del_{k+1} (c \tensor z') = \pi_k(z \tensor z') = 0
      \end{equation*}
      (und ähnlich für $Z_p \times B'_q$).
      Deshalb induziert nun $\pi_k \circ \tensor \colon Z_p \times Z'_q \to \overline H_k$ nach dem Homomorphiesatz ein (eindeutig bestimmtes) bilineares
      \begin{align*}
        s \colon H_p \times H'_q & \to \overline H_k \\
        \intertext{mit}
        s([z], [z']) & \coloneqq [ z \tensor z']
      \end{align*}
      und damit ein lineares
      \begin{align*}
        \lambda_{p,q} \colon H_p \tensor H'_q & \to \overline H_k\\
        [z] \tensor [z'] & \mapsto [z \tensor z']
      \end{align*}
  \end{enumerate}
\end{vorbereitung}<++>




\end{document}
