
\begin{kommentar}
    Jedes Element $t$ in $A \tensor B$ ist damit von der Form
    \begin{equation*}
      t = a_1 \tensor b_1 + \dotsb a_r \tensor b_r
    \end{equation*}
    mit $r \in \N_0$, $a_1, \dotsc a_r, \in A$ und $b_1, \dotsc, b_r \in B$.
    Beachte aber, dass diese Darstellung im Allgemeinen nicht eindeutig ist.
\end{kommentar}

\begin{defn}
  Seien $f \colon A \to A'$ und $g \colon B \to B'$ Homomorphismen.
  Dann wird durch
  \begin{equation*}
    \label{eqn:tensor_hom}
    \tag{$\ast$}
    h(a \tensor b) \coloneqq f(a) \tensor g(b)
  \end{equation*}
  ein Homomorphismus $h \colon A \tensor B \to A' \tensor B'$ definiert, der mit $h \eqqcolon f \tensor g$ bezeichnet wird.
\end{defn}

\begin{kommentar}
  \begin{enumerate}
    \item
      Durch~\eqref{eqn:tensor_hom} wird tatsächlich genau ein Homomorphismus $h \colon A \tensor B \to A' \tensor B'$ gegeben, denn:
      Betrachtet man
      \begin{align*}
        f \times g \colon A \times B & \to A' \times B', \\
        (f \times g) (a,b) & \coloneqq (f(a),g(b)),
      \end{align*}
      so ist $H \coloneqq \tensor \circ (f \times g) \colon A \times B \to A' \tensor B'$ bilinear und daher existiert eindeutig bestimmtes $h \colon A \tensor B \to A' \tensor B'$ mit $h \circ \tensor = H$, das heißt $h(a \tensor b) = f(a) \tensor g(b)$.
      \begin{equation*}
        \begin{tikzcd}
          A \times B
            \arrow{r}{f \times g}
            \arrow{d}{\tensor}
            \arrow{dr}{H}
          & A' \times B'
            \arrow{d}{\tensor}
          \\
          A \tensor B
            \arrow{r}{h}
          & A' \tensor B'
        \end{tikzcd}
      \end{equation*}
    \item
      Sind $f' \colon A' \to A''$ und $g' \colon B' \to B''$ weitere Homomorphismen, so gilt (offenbar)
      \begin{equation*}
        (f' \circ f) \tensor (g' \circ g) = (f' \tensor g') \circ (f \tensor g),
      \end{equation*}
      insbesondere
      \begin{equation*}
        (f \tensor g) = (f \tensor \id) \circ (\id \tensor g).
      \end{equation*}
      Hält man daher einen Faktor $G \in \Ob(\Ab)$ fest, so erhält man durch
      \todo[]{fix the $\Ob$ issue}
      \begin{align*}
        F \colon \Ab & \to \Ab\\
        F(A) & \coloneqq A \tensor G \\
        F(f) & \coloneqq f \tensor \id_G
      \end{align*}
      einen (covarianten) Funktor ($F\colon \Ab \times \Ab \to \Ab$ ist ein Bifunktor, covariant in beiden Argumenten).
  \end{enumerate}
\end{kommentar}

\begin{bemerkung}
\label{bem:tensor_summen_iso}
  Seien $A_1$, $A_2$ und $G$ abelsche Gruppen.
  Dann definiert
  \begin{align*}
    \Phi \colon (A_1 \oplus A_2) \tensor G & \to (A_1 \tensor G) \oplus (A_2 \tensor G),\\
    \Phi( (a_1,a_2) \tensor g) & \coloneqq (a_1 \tensor g, a_2 \tensor g)
  \end{align*}
  einen (kanonischen) Isomorphismus von $(A_1 \oplus A_2) \tensor G$ nach $A_1 \tensor G \oplus A_2 \tensor G$, wobei hier vereinbart wird, dass $\tensor$ stärker bindet als $\oplus$.
\end{bemerkung}
\begin{proof}
  Wieder ist
  \begin{align*}
    \tilde \Phi \colon (A_1 \oplus A_2) \times G & \to (A_1 \tensor G) \oplus (A_2 \tensor G), \\
    ( (a_1, a_2), g) & \mapsto (a_1 \tensor g, a_2 \tensor g)
  \end{align*}
  bilinear und daher $\Phi$ (als Homomorphismus) durch~\eqref{eqn:tensor_hom} wohldefiniert.
  Ebenso ist
  \begin{align*}
    \Psi \colon A_1 \tensor G \oplus A_2 \tensor G & \to (A_1 \oplus A_2) \tensor G, \\
    (a_1 \tensor g_1, a_2 \tensor g_2) & \mapsto (a_1,0) \tensor g_1 + (0,a_2) \tensor g_2
  \end{align*}
  wohldefiniert und
  \begin{align*}
    \Psi \circ \Phi & = \id,
    & \Phi \circ \Psi & = \id,
  \end{align*}
  also ist $\Phi$ ein Isomorphismus.
\end{proof}

\begin{beispiel}
  \begin{enumerate}
    \item
      Ähnlich wie bei Bemerkung~\ref{bem:tensor_summen_iso} sieht man, dass
      \begin{equation*}
        A \tensor B \cong B \tensor A
      \end{equation*}
      kanonisch isomorph vermöge $a \tensor b \mapsto b \tensor a$ ist.
    \item
      Für alle abelschen Gruppen ist
      \begin{equation*}
        A \tensor \Z \cong A
      \end{equation*}
      vermöge
      \begin{equation*}
        a \tensor n \mapsto n a,
      \end{equation*}
      wo $n a \coloneqq \underbrace{a + a + \dotsb + a}_{n\text{-mal}}$ für $n \in \N$, $0 a \coloneqq 0$ und $n a \coloneqq (-n) (-a)$ für $n < 0$ bezeichnet.
      Dann ist $A \to A \tensor \Z, a \mapsto a \tensor 1$ offenbar invers dazu.
    \item
      Ist $G$ ein Körper und $A$ abelsche Gruppe, so hat $A \tensor G$ sogar die Struktur eines $G$-Vektorraums vermöge
      \begin{equation*}
        g \cdot (a \tensor g') = (a \tensor gg').
      \end{equation*}
    \item
      Ist $A = \Z^r$, so ist deshalb $\Z^r \tensor G \cong G^{r}$, denn
      \begin{equation*}
        \Z^r \tensor G \cong (\Z \oplus \dotsb \oplus \Z) \tensor G
        \cong \Z \tensor G \oplus \dotsb \oplus \Z \tensor G
        \cong G \oplus \dotsm \oplus G = G^r
      \end{equation*}
    \item
      Ist $A$ eine \emph{Torsionsgruppe}, das heißt $\forall a \in A \exists n \in \N$ mit $n a = 0$, und $G$ ein Körper der Charakteristik $0$, so ist
      \begin{equation*}
        A \tensor G = 0,
      \end{equation*}
      denn für beliebiges $a \in A$ sei $n \in \N$ mit $n a = 0$, dann folgt, dass
      \begin{equation*}
        a \tensor 1 = a \tensor (n \cdot n^{-1}) = (na) \tensor n^{-1} = 0 \tensor n^{-1} = 0
      \end{equation*}
      und somit für alle $g \in G$
      \begin{equation*}
        a \tensor g = g \cdot (a \tensor 1) = g \cdot 0 = 0,
      \end{equation*}
      also $t = 0$ für alle $t \in A \tensor G$.
    \item
      Für endlich erzeugte abelsche Gruppe $A$, also
      \begin{equation*}
        A \cong \Tor(A) \oplus \Z^b
      \end{equation*}
      mit $b = \rg(A)$, annulliert also das Tensorprodukt mit $\Q$ den Torsionsanteil,
      \begin{equation*}
        A \tensor \Q \cong \underbrace{\Tor(A) \tensor \Q}_{= 0} \oplus \underbrace{\Z^b \tensor \Q}_{= \Q^b} \cong \Q^b
      \end{equation*}
      und man erhält: $\rg(A) = \rg(A \tensor \Q)$.
    \item
      Für $m, n \in \N$ ist
      \begin{equation*}
        \Z^m \tensor \Z^n \cong \Z^{mn}
      \end{equation*}
      (Übung, ist $(e_i)$ Basis von $\Z^m$ und $(e_j)$ Basis von $\Z^n$, so ist $(e_i \tensor e_j)$ Basis von $\Z^m \tensor \Z^n$).
    \item
      Für $m,n \in \N$ ist
      \begin{equation*}
        \Z_m \tensor \Z_n \cong \Z_{\mathrm{ggT}(m,n)}
      \end{equation*}
      (Übung).
      Das ermöglicht also die Berechnung von $A \tensor B$ für endlich erzeugte abelsche Gruppen $A$ und $B$.
  \end{enumerate}
\end{beispiel}

\begin{bemerkung}
  Sei $G$ abelsche Gruppe.
  \begin{enumerate}
    \item
      Ist
      \begin{equation*}
        \begin{tikzcd}
          A \arrow{r}{\alpha}
          & B \arrow{r}{\beta}
          & C \arrow{r}{}
          & 0
        \end{tikzcd}
      \end{equation*}
      eine exakte Sequenz abelscher Gruppen, so ist auch
      \begin{equation*}
        \begin{tikzcd}
          A \tensor G \arrow{r}{\alpha \tensor \id}
          & B \tensor G \arrow{r}{\beta \tensor \id}
          & C \tensor G \arrow{r}{}
          & 0
        \end{tikzcd}
      \end{equation*}
      exakt.
    \item
      Ist
      \begin{equation*}
        \begin{tikzcd}
          0 \arrow{r}{}
          & A \arrow{r}{\alpha}
          & B \arrow{r}{\beta}
              \arrow[bend right,swap]{l}{e}
          & C \arrow{r}{}
          & 0
        \end{tikzcd}
      \end{equation*}
      exakt und spaltet, so ist auch
      \begin{equation*}
        \begin{tikzcd}
          0 \arrow{r}{}
          & A \tensor G \arrow{r}{\alpha \tensor \id}
          & B \tensor G \arrow{r}{\beta \tensor \id}
          & C \tensor G \arrow{r}{}
          & 0
        \end{tikzcd}
      \end{equation*}
      exakt und spaltet.
  \end{enumerate}
\end{bemerkung}
\begin{proof}
  \begin{enumerate}
    \item
      \begin{enumerate}
        \item
          Sei $c \in C$ und $g \in G$ beliebig.
          Dann existiert $b \in B$ mit $\beta(b) = c$.
          Somit ist $(\beta \tensor \id) (b \tensor g) = \beta(b) \tensor \id(g) = c \tensor g$.
          Da $C \tensor G$ von $\left\{ c \tensor g : c \in C, g \in G \right\}$ erzeugt ist, folgt: $\beta \tensor \id$ ist surjektiv.
        \item
          \emph{Exaktheit bei $B \tensor G$}:
          \begin{enumerate}[($\alpha$)]
            \item
              Wegen
              \begin{equation*}
                (\beta \tensor \id) \circ (\alpha \tensor \id) = \underbrace{(\beta \circ \alpha)}_{= 0} \tensor \id = 0 \tensor \id = 0
              \end{equation*}
              ist
              \begin{equation*}
                \im (\alpha \tensor \id) \subseteq \ker (\beta \tensor \id)
              \end{equation*}
            \item
              \begin{equation*}
                \begin{tikzcd}
                  A \tensor G
                    \arrow{r}{\alpha \tensor \id}
                  & B \tensor G
                    \arrow{r}{\beta \tensor \id}
                    \arrow{d}{\pi}
                  & C \tensor G
                    \arrow{r}{}
                    \arrow[dashed]{dl}{\varphi}
                  & 0\\
                  & B \tensor G / U
                \end{tikzcd}
              \end{equation*}
              mit $U \coloneqq \im(\alpha \tensor \id)$.

              Betrachte $\tilde \varphi \colon C \times G \to B \tensor G / U$ mit $\tilde \varphi (c,g) \coloneqq [ b \tensor g ]$ und $b \in \beta^{-1}(c)$.
              Wegen der Surjektivität von $\beta$ existiert zunächst ein solches $b$, $\beta^{-1}(c) \neq \emptyset$, und $\tilde \varphi(c,g)$ hängt nicht von der Auswahl ab, denn ist $b' \in \beta^{-1}(c)$ ein weiteres Urbild, $\beta(b) = \beta(b') = c$, so ist $\beta(b' - b) = \beta(b') - \beta(b) = c - c = 0$ und damit existiert ein $a \in A$ mit $\alpha(a) = b' - b$, daher
              \begin{equation*}
                b' \tensor g - b \tensor g = (b' - b) \tensor g = \alpha(a) \tensor g = (\alpha \tensor \id)(a, g) \in U,
              \end{equation*}
              also $[b' \tensor g] = [b \tensor g]$.
              Aus diesem Grund ist $\tilde \varphi$ dann auch ein Homomorphismus.
          \end{enumerate}
      \end{enumerate}
  \end{enumerate}
\end{proof}
