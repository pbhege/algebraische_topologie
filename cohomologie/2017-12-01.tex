
\begin{kommentar}
  \begin{enumerate}
    \item
      Ist beispielsweise $C = S(X)$ der singuläre Kettenkomplex eines topologischen Raumes, $S_k(X) = \mathbb{F} (\Sigma_k(X))$, dann folgt
      \begin{equation*}
        S_k(X) \tensor G = \bigoplus_{\Sigma_k(X)} \Z \tensor G \cong \bigoplus_{\Sigma_k(X)} (\underbrace{\Z \tensor G}_{=G}) = \bigoplus_{\Sigma_k(X)} G,
      \end{equation*}
      das heißt jedes Element $\tau \in S_k(X) \tensor G$ hat eindeutige Darstellung
      \begin{equation*}
        \overline c = g_1 \sigma_1 + \dotsb + g_r \sigma_r,
      \end{equation*}
      wobei $g \sigma \coloneqq g \tensor \sigma$ abkürzt.
    \item
      \emph{Warnung}: Für
      \begin{equation*}
        \overline c = g_1 \sigma_1 + \dotsb + g_r \sigma_r
      \end{equation*}
      ist also
      \begin{equation*}
        \del \overline c = g_1 \del \sigma_1 + \dotsb + g_r \del \sigma_r
      \end{equation*}
      mit
      \begin{equation*}
        \del \sigma = \sum_{i = 0}^k {(-1)}^{i} \sigma \circ \delta_k^i,
      \end{equation*}
      die Elemente $\sigma$ und $\del \sigma$ liegen aber nicht in $S_k(X) \tensor G$, da $G$ im Allgemeinen kein Einselement hat.
    \item
      Die Homologie von $C \tensor G$ notieren wir so:
      \begin{equation*}
        H(C;G) \coloneqq (H_k(C \tensor G))
      \end{equation*}
      und nenne sie die \emph{Homologie von $C$ mit Koeffizienten in $G$}.
  \end{enumerate}
\end{kommentar}

\begin{defn}
  Seien $C$ und $C'$ Kettenkomplexe und $f \colon C \to C'$ eine Kettenabbildung.
  Ist $G$ abelsche Gruppe, so nennt man $f \tensor \id = (f_k \tensor id)$, $f_k \tensor \id \colon C_k \tensor G \to C_k' \tensor G$ die \emph{von $f$ induzierte Kettenabbildung}.
\end{defn}

\begin{kommentar}
  \begin{enumerate}
    \item
      $f$ ist tatsächlich Kettenabbildung, denn:
      \begin{equation*}
        (\del_k' \tensor \id) \circ (f_k \tensor \id) = (\del_k' \circ f_k) \tensor \id = (f_{k1} \circ \del_k) \tensor \id = (f_{k-1} \tensor \id) \circ (\del_k \tensor \id)
      \end{equation*}
    \item
      Das Tensorieren mit $G$ ergibt also einen Funktor
      \begin{equation*}
        F = - \tensor G \colon \KK \to \KK.
      \end{equation*}
  \end{enumerate}
\end{kommentar}

\emph{Frage:} Wie hängen nun $H(C;G)$ und $H(C) = H(C;\Z)$ zusammen?

\begin{vorbereitung}
  Sei $C$ Kettenkomplex, $G$ abelsche Gruppe und $k \in \N_0$.
  Dann induziert das Tensorprodukt $C_k \times G$ ein bilineares
  \begin{align*}
    Z_k \times G &  \xrightarrow{\tensor} Z_k (C \tensor G) \\
    (z,g) & \mapsto z \tensor g,
  \end{align*}
  denn
  \begin{equation*}
    (\del_k \tensor \id) (z \tensor g) = \underbrace{\del_k z}_{=0} \tensor g = 0
  \end{equation*}
  und überführt dabei $B_k \times G$ nach $B_k(C \tensor G)$, denn ist $z = \del w$ ($w \in C_{k+1}$), so ist
  \begin{equation*}
    z \tensor g = \del w \tensor g = \del \tensor \id (w \tensor g)
  \end{equation*}
  Deshalt induziert $\tensor$ ein (eindeutig bestimmtes)
  \begin{align*}
    \Lambda \colon H_k \times G & \to H_k(C;G) \\
    \Lambda([z], g) & = [ z \tensor g]
  \end{align*}
  Da $\Lambda$ offenbar bilinear ist, existiert ein eindeutig bestimmter Homomorphismus
  \begin{align*}
    \lambda \colon H_k \tensor G & \to H_k(C;G) \\
    \intertext{mit}
    \lambda([z] \tensor g) & = [ z \tensor g ]
  \end{align*}
  für alle $z \in Z_k, g \in G$.
  \begin{cd*}
    Z_k \times G \arrow[r, "\tensor"]
      \ar[d, "\pi_k \times \id"]
    & Z_k (C \tensor G)
      \ar[d, "\pi_k"]
    \\
    H_k \times G \ar[r, "\Lambda"]
    & H_k(C;G)
  \end{cd*}
\end{vorbereitung}

\emph{Ab jetzt}: $C$ sei frei.

\begin{lemma}
\label{lem:quux}
  Es gilt mit $\del_k' \colon C_k \to B_{k-1}$ und $i_k \colon B_k \xhookrightarrow{} Z_k$, $j_k \colon Z_k \xhookrightarrow{} C_k$ (also $\del_k = j_{k-1} \circ i_{k-1} \circ \del_k'$):
  \begin{enumerate}
    \item
      \begin{equation*}
        \del_k' \tensor \id C_k \tensor B_{k-1} \tensor \id
      \end{equation*}
      ist surjektiv und
      \begin{equation*}
        \overline{Z_k} \coloneqq Z_k(C \tensor G) = {(\del_k' \tensor \id)}^{-1} (\ker (i_{k-1} \tensor \id))
      \end{equation*}
    \item
      \begin{equation*}
        \del_k' \tensor \id (\overline{B_k}) = 0,
      \end{equation*}
      wobei $\overline{B_k} \coloneqq B_k(C \tensor G)$.
  \end{enumerate}
\end{lemma}

\begin{kommentar}
  \begin{enumerate}
    \item
      Deshalb induziert nun $\del_k' \tensor \id$ einen Homomorphismus
      \begin{align*}
        h_k \colon H_k(C;G) = \overline{Z_k} / \overline{B_k} & \to \ker (i_{k-1} \tensor \id) \\
        [ \overline z ] & \mapsto \del_k' \tensor \id (\overline z).
      \end{align*}
    \item
      Betrachte nun die freie Auflösung
      \begin{cd*}
        S \colon 0 \ar[r]
        & B_{k-1} \ar[r, "i_{k-1}"]
        & Z_{k-1} \ar[r, "\pi_k"]
        & H_{k-1}(C) \ar[r]
        & 0,
      \end{cd*}
      da mit $C_{k-1}$ auch $Z_{k-1}$ und $B_{k-1}$ frei sind.
      Sei dann
      \begin{equation*}
        \Phi = \Phi(\id; S, S(H_{k-1}(C))) \colon \ker(i_{k-1} \tensor \id) \to \Tor(H_{k-1}(C), G)
      \end{equation*}
      der induzierte Isomorphismus.
      Setze schließlich
      \begin{equation*}
        \mu \coloneqq \Phi \circ h \colon H_k(C;G) \to \Tor(H_{k-1}(C), G)
      \end{equation*}
  \end{enumerate}
\end{kommentar}

\begin{proof}[Beweis von Lemma~\ref{lem:quux}]
  \begin{enumerate}
    \item
      \begin{equation*}
        \del_k' \tensor \id \colon C_k \tensor G \to B_{k-1} \tensor G
      \end{equation*}
      surjektiv, $\overline{Z_k}= {(\del_k' \tensor \id)}^{-1} (\ker (i_{k-1} \tensor \id))$.
      Dazu: Da $\del_k'$ surjektiv ist, ist es auch $\del_k' \tensor \id$:
      Ist $z \in B_{k-1}$, $g \in G$, also $z = \del w$ für ein $w \in C_k$, so ist $z \tensor g = \del \tensor \id (w \tensor g)$.
      Da ($z \tensor g : z \in B_{k-1}, g \in G$) $B_{k-1} \tensor G$ erzeugt, folgt, dass $\del_k' \tensor \id$ surjektiv ist.

      Erinnere:
      \begin{equation*}
        \del_k = j_{k-1} \circ i_{k-1} \circ \del_k'.
      \end{equation*}
      Dann gilt
      \begin{equation*}
        \del_k \tensor \id = (j_{k-1} \tensor \id) \circ (i_{k-1} \tensor \id) \circ (\del_{k}' \tensor \id).
      \end{equation*}
      Weil nun die folgende Sequenz spaltet
      \begin{cd*}
        0 \ar[r]
        & Z_{k-1} \ar[r, "j_{k-1}"]
        & C_{k-1} \ar[r, "\del_{k-1}'"]
          \ar[l, bend left, "l_{k-1}"]
        & B_{k-2} \ar[r]
        & 0
      \end{cd*}
      ist auch
      \begin{cd*}
        0 \ar[r]
        & Z_{k-1} \tensor G \ar[r, "j_{k-1} \tensor \id"]
        & C_{k-1} \tensor G \ar[r]
        & B_{k-2} \tensor G \ar[r]
        & 0
      \end{cd*}
      exakt, insbesondere ist $j_{k-1} \tensor \id$ injektiv.
      Somit gilt
      \begin{align*}
        \overline{Z_k}
        & = Z_k(C \tensor G) \\
        & = \ker(\del_k \tensor \id) \\
        & \underset{j_k \tensor \id \text{ injektiv}}{=} \ker( (i_{k-1} \tensor \id) \circ (\del_k' \tensor \id)) \\
        & = {(\del_k' \tensor \id)}^{-1} (\ker (i_{k-1} \tensor \id))
      \end{align*}
    \item
      $\del_k' \tensor \id (\overline{B_k}) = 0$.

      Dazu:
      Sei $\overline z \in \im(\del_{k+1} \tensor \id) = \overline{B_k}$ und ohne Einschränkung $\overline z = (\del_{k+1} \tensor \id) (c \tensor g)$ mit $c \in C_{k+1}, g \in G$ (jedes andere Element ist Summe von solchen).
      Dann ist
      \begin{equation*}
        \del_k' \tensor \id (\overline z) = (\del_k' \tensor \id) \circ (\del_{k+1} \tensor \id) (c \tensor g) = \underbrace{\del_k' \circ \del_{k+1} (c)}_{= 0} {} \tensor g = 0
      \end{equation*}
  \end{enumerate}
\end{proof}

\begin{satz}[Universelles Koeffiziententheorem für Kettenkomplexe]
  Sei $C$ freier Kettenkomplex und $k \in \Z$.
  Dann ist mit $\lambda$ und $\mu$ wie oben beschrieben die folgende Sequenz exakt und spaltet:
  \begin{cd*}
    0 \ar[r]
    & H_k(C) \tensor G \ar[r, "\lambda"]
    & H_k(C;G) \ar[r, "\mu"]
    & \Tor(H_{k-1}(C),G) \ar[r]
    & 0
  \end{cd*}
\end{satz}

\begin{proof}
  \begin{enumerate}
    \item
      Exaktheit bei $\Tor(H_k(C), G)$:
      Nach dem Lemma ist $\del_k' \tensor \id \colon \overline{Z_k} \to \ker (i_{k-1} \tensor \id)$ surjektiv und dann auch $h \colon \overline{H_k} \coloneqq H_k(C;G) \to \ker(i_{k-1} \tensor \id)$ und damit auch $\mu = \Phi \circ h \colon \overline{H_k} \to \Tor(H_{k-1}(C), G)$, da $\Phi$ surjektiv ist.
    \item
      Bei $H_k(C;G)$:
      \begin{enumerate}
        \item
          $\im(\lambda) \subseteq \ker(\mu)$:
          Für $z \in Z_k, g \in G$ ist:
          \begin{align*}
            \mu \circ \lambda ([z] \tensor g) & = \Phi \circ h ( [z \tensor g]) & \text{da $\lambda([z] \tensor g) = [z \tensor g]$} \\
            & = \Phi(\del_k' \tensor \id (z \tensor g)) & \text{da $h([z \tensor g]) = \del_k' z \tensor g \in \ker(i_{k-1} \tensor \id)$} \\
            & = \Phi(\underbrace{\del_k' z}_{=0} {} \tensor g) = 0
          \end{align*}
          Also ist $\mu \circ \lambda = 0$.
        \item
          $\ker \mu \subseteq \im \lambda$:
          Sei $[\overline z] \in \overline{H_k}$ (mit $\overline{z_k} \in \overline{Z_k}$) mit $\mu([\overline z]) = 0$, also
          \begin{equation*}
            0 = \del_k' \tensor \id (\overline z)
          \end{equation*}
          (da $\Phi$ injektiv ist).
          Da
          \begin{cd*}
            0 \ar[r]
            & Z_k \tensor G \ar[r, "j_k \tensor \id"]
            & C_k \tensor G \ar[r, "\del_k' \tensor \id"]
            & B_k \tensor G \ar[r]
            & 0
          \end{cd*}
          bei $C_k \tensor G$ exakt ist, existiert ein $\overline c \in Z_k \tensor G$ mit:
          \begin{equation*}
            j_k \tensor \id (\overline c) = \overline z.
          \end{equation*}
          Es gibt also $r \in \N_0$, $g_1, \dotsc, g_r \in G$ und $z_1, \dotsc, z_r \in Z_k$ mit
          \begin{equation*}
            \overline z = g_1 z_1 + \dotsb + g_r z_r.
          \end{equation*}
          Dann folgt
          \begin{equation*}
            [\overline z] = [ z_1 \tensor g_1 ] + \dotsm + [ z_r \tensor g_r ] = \lambda( [z_1] \tensor g_1) + \dotsm + \lambda([z_r] \tensor g_r) = \lambda([z_1] \tensor g_1 + \dotsm + [z_r] \tensor g_r) \in \im \lambda.
          \end{equation*}
      \end{enumerate}
    \item
      Bei $H_k(C) \tensor G$:
      Da $B_{k-1}$ frei ist, spaltet die Sequenz
      \begin{cd*}
        0 \ar[r]
        & Z_k \ar[r, "j_k"]
        & C_k \ar[r, "\del_k'"]
          \ar[l, bend left, "l_k"]
        & B_{k-1} \ar[r]
        & 0
      \end{cd*}
      Sei $l_k \colon C_k \to Z_k$ eine Spaltung,
      \begin{equation*}
        l_k \circ j_k = \id.
      \end{equation*}
      Betrachte dann die Komposition
      \begin{cd*}
        \label{seq:komposition}
        \tag{$\ast$}
        \overline{Z_k} \subseteq C_k \tensor G \ar[r, "l_k \tensor \id"]
        & Z_k \tensor G \ar[r, "\pi_k \tensor \id"]
        & H_k \tensor G.
      \end{cd*}
      Da jedes Element in $\overline{B_k}$ Summe von Elementen der Form
      \begin{equation*}
        \del c \tensor g = (\del_{k+1} \tensor \id) (c \tensor g)
      \end{equation*}
      ist, ist
      \begin{equation*}
        (\pi_k \tensor \id) \circ (l_k \tensor \id) (\del_k c \tensor g)
        = \pi_k \circ l_k \circ \underbrace{\del_k (c)}_{= j_k (\del_k'(c))} {} \tensor g
        \underset{l_k \circ j_k = \id}{=} \underbrace{\pi \circ \del_k'}_{= 0} (c) \tensor g = 0.
      \end{equation*}
      Deshalb induziert
      \begin{equation*}
        (\pi_k \tensor \id) \circ (l_k \tensor \id) |_{\overline{Z_k}}
      \end{equation*}
      ein Homomorphismus
      \begin{align*}
        \lambda' \colon \overline{H_k} & \to H_k \tensor G \\
        \intertext{mit}
        [\overline z] & \mapsto (\pi_k \tensor \id) \circ (l_k \tensor \id) (\overline z)
      \end{align*}
      also:
      \begin{equation*}
        \lambda' \circ \lambda ([z] \tensor g) = \lambda' ([z \tensor g]) = (\pi_k \tensor \id) \circ (l_k \tensor \id) (z \tensor g) = \pi_k \circ \underbrace{l_k (z)}_{= z} {} \tensor g = [z] \tensor g
      \end{equation*}
      \todo[]{ersetze $l_k$ durch $l_k \circ j_k$}
      und damit $\lambda' \circ \lambda = \id$.
      Somit ist $\lambda$ injektiv und~\eqref{seq:komposition} spaltet.
  \end{enumerate}
\end{proof}
